% Requerimientos no Funcionales del sistema.
% Autor(es): Landa Aguirre Rafael
% Fecha: 19 - Mar - 2018
%
%%==============================================================================
% Requerimientos no Funcionales
%

\subsection{Requerimientos no Funcionales}

\begin{itemize}
	\item[RNF1:] Interfaces de usuario.
	\begin{itemize}
		\item[RNF1.1:] El diseño de las pantallas deberán ser adecuadas 
		para cualquier tipo de usuario, independientemente de la edad.
		\item[RNF1.2:] Cada pantalla tendrá íconos descriptivos.
		\item[RNF1.3:] La pantalla principal mostrará una interfaz al 
		usuario haciendo petición de la ubicación del mismo.
	\end{itemize}
	\item[RNF2:] Mantenimiento.
	\begin{itemize}
		\item[RNF2.1:] El sistema tendrá en cuenta la incorporación de 
		mejoras, mientras éstas no interfieran con el funcionamiento del 
		mismo.
		\item[RNF2.2:] En caso de que se requieran cambios en el sistema 
		y éstos interfieran con el funcionamiento del mismo, dependerá 
		de retroalimentación de los usuarios.
	\end{itemize}
	\item[RNF3:] Portabilidad.
	\begin{itemize}
		\item[RNF3.1:] El sistema operará en diferentes navegadores web 
		(Mozilla Firefox, Google Chrome, Internet Explorer Edge).
	\end{itemize}
	\item[RNF4:] Restricciones de diseño y construcción.
	\begin{itemize}
		\item[RNF4.1:] El sistema operará sobre HTTP.
		\item[RNF4.2:] El sistema estará construido con el framework de 
		desarrollo web para python: DJANGO.
		\item[RNF4.3:] El servidor estará brindando soporte WSGI para 
		montar aplicaciones desarrolladas en Python.
		\item[RNF4.4:] El sistema deberá ser desarrollado de tal manera 
		que cualquier persona involucrada en el área de desarrollo de 
		software sea capaz de agregar o modificar funciones de la 
		aplicación diseñada.
	\end{itemize}
	\item[RNF5:] Legales y reglamentarias.
	\begin{itemize}
		\item[RNF5.1:] La aplicación está protegida bajo las normas del 
		IPN.
	\end{itemize}
\end{itemize}

%===============================================================================
% Sección: Metodología de desarrollo.
%

\subsection{Metodología de desarrollo}

Para el desarrollo de este proyecto se usará la metodología de Prototipado 
Evolutivo, ya que permite desarrollar el sistema a medida que avanza el 
proyecto, facilita la retroalimentación de cada una de sus fases y permite 
realizar una re-evaluación tanto de diseño como de análisis en caso de ser 
necesario. Usar esta metodología permite desarrollar los aspectos más visibles 
al inicio del desarrollo, e ir complementando hasta llegar al prototipo final 
\cite{ingSoftware:Sommerville}. \\
Algunas ventajas con las que cuenta, son:
\begin{itemize}
	\item Permite realizar cambios en el proyecto con base al cambio de 
		requerimientos.
	\item Los cambios se adaptan fácilmente a medida que avanza el proyecto.
	\item La retroalimentación de pruebas permite llevar a cabo un rediseño 
		en caso de presentarse algún error.
\end{itemize}

La metodología también permite establecer un rango de error para poder realizar 
las pruebas que sean necesarias en el sistema hasta obtener los resultados 
deseados. \\

Las etapas que maneja la metodología de prototipado evolutivo se muestran en la 
figura \ref{fig:diagrama:prototipo:evolutivo}.

\begin{figure}[H]
	\begin{center}
		\includegraphics[scale=0.9]
		{images/doc/img_diag_prototipo_evolutivo}
		\caption{Diagrama de las etapas de Prototipado Evolutivo.}
		\label{fig:diagrama:prototipo:evolutivo}
	\end{center}
\end{figure}

Los prototipos que se contemplaron para la realización de este proyecto son:

\begin{itemize}
	\item Comunicación entre el sensor de proximidad y el drone. 
	\item Aplicación web.
\end{itemize}

. \\

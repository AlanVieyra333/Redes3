%===================== Especificación del módulo GSM ===========================

\subsubsection{Especificación del módulo GSM}

Para realizar la comunicación entre el drone y el servidor web a través de internet, se usará el sistema GSM que permite a un cliente navegar por internet desde cualquier lugar con señal GSM.

La interfaz de comunicación que se usará será el GSM-800L ya que permite comunicar a 800MHz y su tamaño es pequeño y ligero comparado con otros modelos, como se muestra en la tabla \ref{tabla:modelo-gsm}.

\begin{table}[H]
	\centering
	\caption{Comparación de modelos de GSM.}
	\label{tabla:modelo-gsm}
	\begin{tabular}{|c|m{2cm}|c|m{2cm}|m{2cm}|c|}
		\hline
			\centering\textbf{Modelo} &
			\centering\textbf{Frecuencia de Banda (MHz)} &
			\centering\textbf{Protocolos} &
			\centering\textbf{Voltaje de operación (Volts)} &
			\centering\textbf{Velocidad de transmisión (Baudios)} &
			\textbf{Costo (MXN)}
		\\ \hline
			\rowcolor{colorGrisClaro}
			SIM-800L &
			\centering 850 / 900 / 1800 / 1900 &
			\centering HTTP/TCP/UDP &
			\centering 3.5 - 4.2 &
			\centering 115200 &
			Menor a  \$300
		\\ \hline
			SIM-900L &
			\centering 850 / 900 / 1800 / 1900 &
			\centering HTTP/TCP/UDP &
			\centering 4.5 - 5.5 &
			\centering 115200 &
			Arriba de \$500
		\\ \hline
			MG323-B &
			\centering 900 / 1800 &
			\centering HTTP/TCP/UDP &
			\centering 3.3  - 4.2 &
			\centering 115200 &
			Arriba de \$500
		\\ \hline
	\end{tabular}
\end{table}
%===================== Especificación de la comunicación serial ===========================

\subsubsection{Especificación de la comunicación serial}

Para comunicar al drone con el microcontrolador se decidió implementar una interfaz USB ya que ésta permitirá tanto el intercambio de datos o paquetes, como alimentar al microcontrolador a un voltaje de 5v.

Debido a que el microcontrolador tiene una terminal USART o UART, es necesario un convertidor estándar de comunicación UART a USB. \\

La interfaz que se usará será la FT232RL ya que su implementación es sencilla y es más accesible comercialmente en comparación con otras, como se muestra en la tabla \ref{tabla:modelo-convertidores-usb}.

\begin{table}[H]
	\centering
	\caption{Comparación de convertidores UART a USB.}
	\label{tabla:modelo-convertidores-usb}
	\begin{tabular}{|m{2cm}|m{2cm}|c|m{3cm}|c|}
		\hline
			\centering\textbf{Modelo} &
			\centering\textbf{Rango de frecuencia} &
			\centering\textbf{Popularidad} &
			\centering\textbf{Facilidad de implementación} &
			\textbf{Costo}
		\\ \hline
			\rowcolor{colorGrisClaro}
			FT232RL &
			\centering 300 baudios - 3 Mega baudios &
			\centering Media &
			\centering Alta &
			Menor a \$100 MXN
		\\ \hline
			USB-SER &
			\centering 300 baudios - 3 Mega baudios &
			\centering Media &
			\centering Baja &
			Arriba de \$100 MXN
		\\ \hline
			CP2102 &
			\centering 300 baudios - 3 Mega baudios &
			\centering Media &
			\centering Alta &
			Menor a \$100 MXN
		\\ \hline
			FTDI231XQ &
			\centering 300 baudios - 3 Mega baudios &
			\centering Media &
			\centering Alta &
			Menor a \$100 MXN
		\\ \hline
	\end{tabular}
\end{table}

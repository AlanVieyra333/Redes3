%===================== Especificación del sensor de proximidad ===========================

\subsubsection{Especificación del sensor de proximidad}

Debido a que el drone seleccionado no cuenta con sensores que le permita la detección de obstáculos cercanos, se optó por implementar en él un sensor de proximidad.

El tipo de sensor será ultrasónico ya que su costo/beneficio es menor comparado con otros como se observa en la tabla \ref{tabla:tipo-sensores}.

\begin{table}[H]
	\centering
	\caption{Comparación de tipos de sensores de proximidad.}
	\label{tabla:tipo-sensores}
	\begin{tabular}{|m{2cm}|m{2.5cm}|m{2cm}|m{2cm}|c|}
		\hline
			\centering\textbf{Tipo de sensor} &
			\centering\textbf{Distancia mínima} &
			\centering\textbf{Distancia máxima} &
			\centering\textbf{Factor de error} &
			\textbf{Costo}
		\\ \hline
			Capacitivo &
			\centering ~5mm &
			\centering ~20mm &
			\centering Tamaño del objeto &
			Medio
		\\ \hline
			Láser &
			\centering ~0mm &
			\centering ~100,000mm &
			\centering Precisión &
			Alto
		\\ \hline
			Inductivo &
			\centering ~0.8 mm &
			\centering ~250mm &
			\centering Tamaño del objeto &
			Alto
		\\ \hline
			\rowcolor{colorGrisClaro}
			Ultrasónico &
			\centering ~20mm &
			\centering ~4,000mm &
			\centering Falsos ecos &
			Bajo
		\\ \hline
	\end{tabular}
\end{table}

El modelo que se usará será el HC-SR04 debido a que nos permite detectar distancia de a lo más 400cm, suficiente para el cumplimiento del objetivo del presente prototipo, y su costo es menor que el de otros sensores del mismo tipo, como de observa en la tabla \ref{tabla:modelo-sensores}.

\begin{table}[H]
	\centering
	\caption{Comparación de modelos de sensores ultrasónicos.}
	\label{tabla:modelo-sensores}
	\begin{tabular}{|m{2cm}|m{2cm}|m{2cm}|c|}
		\hline
			\centering\textbf{Modelo} &
			\centering\textbf{Distancia mínima} &
			\centering\textbf{Distancia máxima} &
			\textbf{Costo}
		\\ \hline
			\rowcolor{colorGrisClaro}
			HC-SRF04 &
			\centering 2 cm &
			\centering 400 cm &
			Bajo
		\\ \hline
			HC-SRF08 &
			\centering 3 cm &
			\centering 600 cm &
			Medio
		\\ \hline
			HC-SRF10 &
			\centering 6 cm &
			\centering 600 cm &
			Medio
		\\ \hline
	\end{tabular}
\end{table}
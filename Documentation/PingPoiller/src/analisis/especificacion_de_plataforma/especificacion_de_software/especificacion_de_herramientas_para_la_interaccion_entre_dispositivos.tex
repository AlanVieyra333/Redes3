%===================== Especificación de herramientas para la interacción entre dispositivos ===========================

\subsubsection{Especificación de herramientas para la interacción entre dispositivos}

Para permitir la interacción entre los diferentes dispositivos utilizados (drone, microcontrolador y servidor web), se hace uso de varios protocolos de comunicación tanto de terceros como propios, los cuales se describen a continuación. \\

%\textbf{Comunicación entre el drone y microcontrolador} \\


\textbf{Comunicación entre la aplicación web (servidor) y microcontrolador} \\
%La comunicación entre el servidor web y el microcontrolador debe realizarse a través de internet e inalámbricamente debido a que el microcontrolador estará ubicado sobre el drone. Los componentes que permiten dicha comunicación son los módulos WI-FI o bien los módulos GSM. \\
%El uso de un módulo WI-FI implicaría estar siempre serca de de dicha señal, por lo que el rango de operabilidad del drone sería muy limitado. Por e contrario, el uso de un módulo GSM tiene un mayor rango de señal, permitiendo usarlo en cualquier sona abierta con dicha señal.
Como se mencionó anteriormente, la comunicación entre el drone y la aplicación web será a través de internet usando como intermediario un módulo GSM conectado a un microcontrolador. \\
El protocolo de comunicación utilizado se diseñó de tal manera que permita enviar todos los datos necesarios para el correcto monitoreo del drone. \\
En el código \ref{code:estructura_datagrama} se muestra la estructura del datagrama utilizado.\\

\begin{lstlisting}[caption={Estructura del protocolo de comunicación entre la aplicación web y el microcontrolador.},captionpos=b,label=code:estructura_datagrama]
struct DatagramGSM{
  double latitude;		// Latitud geografica.
  double longitude;		// Longitud geografica.
  float altitude;			// Altitud relativa (cuya referencia es la altura inicial).
  float battery_level;	// Nivel de bateria en voltaje.
  float speed;				// Suma de los vetores de velocidad vertical y horizontal.
  bool state;				// Estado del drone (en vuelo, aterrizado).
  float giros_x;			// Valor del giroscopio en el eje x.
  float giros_y;			// Valor del giroscopio en el eje y.
  float giros_z;			// Valor del giroscopio en el eje z.
};
\end{lstlisting}

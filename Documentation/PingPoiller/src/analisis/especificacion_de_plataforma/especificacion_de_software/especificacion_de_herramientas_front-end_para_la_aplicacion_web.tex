%===================== Especificación de herramientas front-end para la aplicación web ===========================

\subsubsection{Especificación de herramientas front-end para la aplicación web}

Para el desarrollo de la interfaz de usuario se hará uso del framework Angular ya que tiene algunas ventajas respecto a otras como se muestra en la tabla \ref{tabla:frameworks-frontend}. Cuenta con una alta popularidad en el mercado, permite el desarrollo basado en componentes, su velocidad de respuesta es rápida y aunque su curva de aprendizaje es alta, lo compensa la rapidez con la que se desarrolla la aplicación.

\begin{table}[H]
	\centering
	\caption{Comparación de frameworks frontend.}
	\label{tabla:frameworks-frontend}
	\begin{tabular}{|l|m{2.5cm}|c|c|c|}
		\hline
		\centering\textbf{Framework} &
		\centering\textbf{Basado en componentes} &
		\centering\textbf{Respuesta} &
		\centering\textbf{Popularidad} &
		\textbf{Curva de aprendizaje}
		\\ \hline
		React &
		\centering Si &
		Muy rápida &
		Muy alta &
		Alta
		\\ \hline
		\rowcolor{colorGrisClaro}
		Angular &
		\centering Si &
		Rápida &
		Alta &
		Alta
		\\ \hline
		Vue &
		\centering Si &
		Rápida &
		Media &
		Baja
		\\ \hline
		Ember &
		\centering Si &
		Muy rápida &
		Baja &
		Alta
		\\ \hline
	\end{tabular}
\end{table}
%===================== Especificación de herramientas back-end para la aplicación web ===========================

\subsubsection{Especificación de herramientas back-end para la aplicación web}

Para el desarrollo back-end de la aplicación web se usará el framework DJango ya que su velocidad de respuesta y popularidad es alta como se muestra en la tabla \ref{tabla:frameworks-backend}, Además tiene un conjunto de herramientas gratuitas y de código abierto para el desarrollo de aplicaciones web basadas en Python. También este framework de desarrollo maneja el paradigma orientado a objetos y esto permitirá que el código pueda ser escalable y modelado por medio de UML

\begin{table}[H]
	\centering
	\caption{Comparación de frameworks backend.}
	\label{tabla:frameworks-backend}
	\begin{tabular}{|l|c|c|c|c|}
		\hline
		\centering\textbf{Framework} &
		\textbf{Lenguaje} &
		\textbf{Respuesta} &
		\textbf{Popularidad} &
		\textbf{Curva de aprendizaje}
		\\ \hline
		Spring &
		Java &
		Lenta &
		Muy alta &
		Alta
		\\ \hline
		\rowcolor{colorGrisClaro}
		DJango &
		Python &
		Rápida &
		Alta &
		Media
		\\ \hline
		Laravel &
		PHP &
		Media &
		Alta &
		Media
		\\ \hline
		Hibernate &
		Java &
		Lenta &
		Alta &
		Media
		\\ \hline
	\end{tabular}
\end{table}
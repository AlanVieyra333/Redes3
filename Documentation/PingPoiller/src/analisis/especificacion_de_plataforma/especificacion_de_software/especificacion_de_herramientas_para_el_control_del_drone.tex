%===================== Especificación de herramientas para el control del drone ===========================

\subsubsection{Especificación de herramientas para el control del drone}

El drone a utilizar cuenta con sistema operativo linux, que se opera como 
cualquier otro, por ejemplo, Debian, Ubuntu, Fedora, entre otros. Sobre este 
sistema se ejecuta el controlador de vuelo original de Parrot, el cual se 
comunica con el cliente a través de una aplicación móvil usando la red WI-FI que 
genera el drone. \\

Para controlar el vuelo autónomamente es necesario controlar sus movimientos 
desde el mismo drone por lo que usar el SDK de Parrot o librerías de node como 
las mostradas en el capítulo 2 no son una opción. Para lograr este objetivo se 
usará el controlador de vuelo Ardupilot ya que cuenta con las características 
necesarias como se muestra en la tabla \ref{tabla:herramientas-control-drone}.\\

Ardupilot es de código abierto y gratuito, y permite controlar el drone desde el 
propio drone, desde el proceso de pre-armado, armado, estado de vuelo, modos de 
vuelo, aterrizaje, y protocolos de comunicación entre dispositivos. cupliendo.\\

Ardupilot se puede instalar en el drone seleccionado (Bebop 2) siguiendo los 
pasos que se muestran en la página principal de ardupilot.

\begin{table}[H]
	\centering
	\caption{Comparación de herramientas de control de drone.}
	\label{tabla:herramientas-control-drone}
	\begin{tabular}{|l|c|c|c|}
		\hline
		\centering\textbf{Nombre} &
		\textbf{Control desde el drone} &
		\textbf{Facilidad de uso} &
		\textbf{Costo}
		\\ \hline
		SDK Parrot &
		No &
		Media &
		Gratuito
		\\ \hline
		Librerías NodeJS &
		No &
		Alta &
		Gratuito
		\\ \hline
		\rowcolor{colorGrisClaro}
		Ardupilot &
		Si &
		Baja &
		Gratuito
		\\ \hline
	\end{tabular}
\end{table}

%===================== Especificación de herramientas para bases de datos ===========================

\subsubsection{Especificación de herramientas para bases de datos}

Se elige MySQL ya que permite una instalación sencilla, es compatible con sistemas Windows, Linux (y sus distribuciones) y OSX; además de ofrecer una licencia pública y es considerada como la base de datos de código abierto más popular en el mundo, como se muestra en la tabla \ref{tabla:bases-de-datos}.

\begin{table}[H]
	\centering
	\caption{Comparación de manejadores de bases de datos.}
	\label{tabla:bases-de-datos}
	\begin{tabular}{|l|c|c|c|}
		\hline
		\centering\textbf{Nombre} &
		\textbf{Lenguaje} &
		\textbf{SO compatibles} &
		\textbf{Costo}
		\\ \hline		
		PostgreSQL &
		PL/SQL &
		Windows/Linux/OSx &
		Ninguno
		\\ \hline
		\rowcolor{colorGrisClaro}
		MySQL &
		SQL &
		Windows/Linux/OSx &
		Ninguno
		\\ \hline
		MongoDB &
		JSON &
		Windows/Linux/OSx &
		Ninguno
		\\ \hline
	\end{tabular}
\end{table}

%===============================================================================
% Sección: Reglas del negocio.
%

%===============================================================================
%% Documentación de reglas de negocio:
% {\bcIntegridad}    % Clase: \bcCondition,   \bcIntegridad, \bcAutorization, \bcDerivation.
% {\btEnabler}     % Tipo:  \btEnabler,     \btTimer,      \btExecutive.
% {\blControlling}    % Nivel: \blControlling, \blInfluencing.
% \BRItem[Versión] <<Versión de la regla de negocio>>.
% \BRItem[Estado] <<Estado de la regla de negocio>>. % Aprobado, Por aprobar. 
% \BRItem[Propuesta por] <<Nombre de la persona que elaboró la RN>>.
% \BRItem[Revisada por] \cdtEmpty
%	\begin{itemize}
%		\item <<Nombre de colaboradores o revisores>>.
%	\end{itemize}
% \BRItem[Descripción] Descripción de la regla de negocio.
% \BRItem[Sentencia] Sentencia .
% \BRItem[Motivación] Razón por la existencia de la regla de negocio.
% \BRItem[Ejemplo positivo] Cumplen la regla:
%\begin{itemize}
%	\item Lista de ejemplos que cumplen la RN.
%\end{itemize}
%\BRItem[Ejemplo negativo] No cumplen con la regla:
%\begin{itemize}
%	\item Lista de ejemplos que no cumplen la RN.
%\end{itemize}

%===============================================================================
% RN: RN-VA-N01 Datos de usuario necesarios
%
\begin{BusinessRule}{RN-VA-N01}{Cálculo de distancia sensor de proximidad}
	{\bcDerivation}
	{\btEnabler}
	{\blControlling}
	\BRItem[Versión] 1.0.
	\BRItem[Estado] \ Aprobado.
	\BRItem[Propuesta por] Rafael Landa Aguirre.
	\BRItem[Revisada por] \cdtEmpty
	\begin{itemize}
		\item \ Victor David Olvera Perez
	\end{itemize}
	\BRItem[Descripción] El cálculo de distancia del sensor de proximidad 
	hace referencia a detectar la presencia de un obstáculo en un rango que 
	nos proporciona el fabricante del sensor y además, un rango que el 
	desarrollador puede programar.
	
	\BRItem[Sentencia] \cdtEmpty

	\begin{equation*}
	Distancia(Duracion\_de\_Pulso) = Duracion\_de\_Pulso * 0.017
	\end{equation*}

	Donde:
	\begin{itemize}
		\item \textit{$Duracion\_de\_Pulso:$} Tiempo respuesta de 
		duración de un pulso enviado por el sensor de proximidad.
		\item \textit{$0.017:$} Constante de tiempo igual al recíproco 
		de la velocidad del sonido en condiciones normales de 
		temperatura.
		\item \textit{$Distancia:$} Distancia aproximada (cm) a un 
		objeto detectado.
	\end{itemize}

	\BRItem[Motivación] Calcular la distancia próxima a un objeto para 
	ayudar a esquivar obstáculos.
\end{BusinessRule}

%===============================================================================
% RN: RN-MON-N01 Datos de usuario necesarios
%
\begin{BusinessRule}{RN-MON-N01}{Datos necesarios para el registro de usuarios}
	{\bcIntegridad}
	{\btEnabler}
	{\blControlling}
	\BRItem[Versión] 1.0.
	\BRItem[Estado] Aprobado.
	\BRItem[Propuesta por] Rafael Landa Aguirre.
	\BRItem[Revisada por] \cdtEmpty
	\begin{itemize}
		\item Alan Fernando Rincón Vieyra.
		\item Víctor David Olvera Pérez.
	\end{itemize}
	\BRItem[Descripción] Un usuario deberá registrarse para el ingreso en el 
	sistema. Debe introducir la siguiente información:
		\begin{itemize}
			\item Nombre. (Obligatorio)
			\item Nombre de usuario. (Obligatorio)
			\item Correo electrónico. (Obligatorio)
			\item Contraseña. (Obligatorio)
		\end{itemize}
	\BRItem[Motivación] El nombre de usuario permite identificar a cada 
	usuario en el sistema, el correo electrónico permite al usuario 
	restablecer su contraseña, la contraseña restringe el acceso al perfil 
	de cada usuario.
\end{BusinessRule}

%===============================================================================
% RN: RN-MON-N02 Nombre válido
%
\begin{BusinessRule}{RN-MON-N02}{Nombre válido}
	{\bcIntegridad}
	{\btEnabler}
	{\blControlling}
	\BRItem[Versión] 1.0.
	\BRItem[Estado] Aprobado.
	\BRItem[Propuesta por] Alan Fernando Rincón Vieyra.
	\BRItem[Revisada por] \cdtEmpty
	\begin{itemize}
		\item Rafael Landa Aguirre.
		\item Víctor David Olvera Pérez.
	\end{itemize}
	\BRItem[Descripción] El formato de nombre deberá ser entre 2 y 50 
	caracteres alfabéticos incluyendo la “ñ” y acentos.
	\BRItem[Sentencia] Expresión regular:
	\begin{center}
		\^\ [a-zA-ZáéíóúüÁÉÍÓÚÜñÑ]\{8,20\}\$.
	\end{center}
	\BRItem[Motivación] Evitar uso de caracteres inválidos.
	\BRItem[Ejemplo positivo] Cumplen la regla:
	\begin{itemize}
		\item Alan Fernando Rincón Vieyra
		\item Rafael Landa Aguirre
		\item Víctor David Olvera Pérez
	\end{itemize}
	\BRItem[Ejemplo negativo] No cumplen con la regla:
	\begin{itemize}
		\item “O“.
		\item “H0l4 Mund0”.
		\item “Esto es un texto con mas de cincuenta caracteres alfabéticos“.
	\end{itemize}
\end{BusinessRule}

%===============================================================================
% RN: RN-MON-N03 Nombre de usuario válido
%
\begin{BusinessRule}{RN-MON-N03}{Nombre de usuario válido}
	{\bcIntegridad}
	{\btEnabler}
	{\blControlling}
	\BRItem[Versión] 1.0.
	\BRItem[Estado] Aprobado.
	\BRItem[Propuesta por] Rafael Landa Aguirre.
	\BRItem[Revisada por] \cdtEmpty
	\begin{itemize}
		\item Alan Fernando Rincón Vieyra.
		\item Víctor David Olvera Pérez.
	\end{itemize}
	\BRItem[Descripción] El formato de nombre de usuario deberá ser entre 8 
	y 20 caracteres alfanuméricos.
	\BRItem[Sentencia] Expresión regular:
	\begin{center}
		\^ \ ([a-zA-Z]$|$[0-9])\{8,20\}\$
	\end{center}
	\BRItem[Motivación] Evitar uso de caracteres inválidos.
	\BRItem[Ejemplo positivo] Cumplen la regla:
	\begin{itemize}
		\item $AlanVieyra$.
		\item $NombreDeUsuario12345$.
	\end{itemize}
	\BRItem[Ejemplo negativo] No cumplen con la regla:
	\begin{itemize}
		\item $Usuario$.
		\item $Nombre de Usuario$.
	\end{itemize}
\end{BusinessRule}

%===============================================================================
% RN: RN-MON-N04 Correo válido
%
\begin{BusinessRule}{RN-MON-N04}{Correo válido}
	{\bcIntegridad}
	{\btEnabler}
	{\blControlling}
	\BRItem[Versión] 1.0.
	\BRItem[Estado] Aprobado.
	\BRItem[Propuesta por] Rafael Landa Aguirre.
	\BRItem[Revisada por] \cdtEmpty
	\begin{itemize}
		\item Alan Fernando Rincón Vieyra.
		\item Víctor David Olvera Pérez.
	\end{itemize}
	\BRItem[Descripción] El formato de un correo electrónico deberá estar 
	definido por caracteres alfanuméricos, incluyendo “.” y “@”, cumpliendo 
	con el estándar que plantea la organización W3C.
	\BRItem[Sentencia] Expresión regular:
	\begin{center}
		\^ \ [a-zA-Z0-9.!\#\$\%\&'*+/=? \^ \ \_ \` \ \{$|$\}\textasciitilde-]+
		@[a-zA-Z0-9-]+(?:\textbackslash .[a-zA-Z0-9-]+)*\$
	\end{center}
	\BRItem[Motivación] Evitar uso de caracteres inválidos.
	\BRItem[Ejemplo positivo] Cumplen la regla:
	\begin{itemize}
		\item $user.name34@gmail.com$.
		\item $alguien\_otro@outlook.com$.
	\end{itemize}
	\BRItem[Ejemplo negativo] No cumplen con la regla:
	\begin{itemize}
		\item $Usuario$.
		\item $usuariogmail.com$.
	\end{itemize}
\end{BusinessRule}

%===============================================================================
% RN: RN-MON-N05 Contraseña válida
%
\begin{BusinessRule}{RN-MON-N05}{Contraseña válida}
	{\bcIntegridad}
	{\btEnabler}
	{\blControlling}
	\BRItem[Versión] 1.0.
	\BRItem[Estado] Aprobado.
	\BRItem[Propuesta por] Rafael Landa Aguirre.
	\BRItem[Revisada por] \cdtEmpty
	\begin{itemize}
		\item Alan Fernando Rincón Vieyra.
		\item Víctor David Olvera Pérez.
	\end{itemize}
	\BRItem[Descripción] La longitud de la contraseña deberá ser mínimo de 8 
	caracteres y máximo 50 caracteres alfanuméricos.
	\BRItem[Sentencia] Expresión regular:
	%%^[a-zA-Z0-9.!#$%&'*+/=?^_`{|}~-]{8,50}$. 
	\begin{center}
		\$\^ \ [a-zA-Z0-9.! \# \$ \% \& '*+/=? \^ \ \_ \ \` \ \{$|$\} 
		\textbackslash -]\{8,50\}\$\$
	\end{center}
	\BRItem[Motivación] Evitar uso de caracteres inválidos, y manejar 
	contraseñas con seguridad alta.
	\BRItem[Ejemplo positivo] Cumplen la regla:
	\begin{itemize}
		\item $CoNtRaSeNa123$.
		\item $C0NtRa\$3n4$.
	\end{itemize}
	\BRItem[Ejemplo negativo] No cumplen con la regla:
	\begin{itemize}
		\item $my pass$.
		\item $ContrasenaContrasenaContrasenaContrasenaContrasena1$.
	\end{itemize}
\end{BusinessRule}

%===============================================================================
% RN: RN-MON-N06 Coordenadas de vuelo válidas
%
\begin{BusinessRule}{RN-MON-N06}{Coordenadas de vuelo válidas}
	{\bcIntegridad}
	{\btEnabler}
	{\blControlling}
	\BRItem[Versión] 1.0.
	\BRItem[Estado] Aprobado.
	\BRItem[Propuesta por] Rafael Landa Aguirre.
	\BRItem[Revisada por] \cdtEmpty
	\begin{itemize}
		\item Víctor David Olvera Pérez.
	\end{itemize}
	\BRItem[Descripción] Se debe seleccionar las coordenadas destino del 
	vuelo, teniendo en consideración entre la distancia de ubicación inicial 
	y final debe ser máximo de 50 metros, para que el drone pueda realizar 
	su vuelo autónomo, por motivos de duración de batería.
	\BRItem[Motivación] Evitar que las coordenadas sean las mismas en inicio 
	y final, y la distancia entre ellas permita al drone realizar su vuelo 
	antes de que su batería se agote.
\end{BusinessRule}


%===============================================================================
% RN: RN-MON-N07 Información del drone
%
\begin{BusinessRule}{RN-MON-N07}{Información del drone}
	{\bcIntegridad}
	{\btEnabler}
	{\blControlling}
	\BRItem[Versión] 1.0.
	\BRItem[Estado] Aprobado.
	\BRItem[Propuesta por] Víctor David Olvera Pérez.
	\BRItem[Revisada por] \cdtEmpty
	\begin{itemize}
		\item Rafael Landa Aguirre.
	\end{itemize}
	\BRItem[Descripción] En la aplicación se debe recibir información sobre 
	el sistema interno del drone, tomando en cuenta los siguientes puntos.
		\begin{itemize}
			\item Posición geográfica (se mostrará en el mapa).
			\item Altura sobre el nivel del mar.
			\item Nivel de batería.
			\item Velocidad vertical y horizontal.
			\item Orientación magnética.
			\item Estado del drone.
		\end{itemize}
	\BRItem[Motivación] Tener información sobre el sistema interno del drone.
\end{BusinessRule}

%===============================================================================
% RN: RN-MON-N08 Datos de historial de vuelo
%
\begin{BusinessRule}{RN-MON-N08}{Datos de historial de vuelo}
	{\bcIntegridad}
	{\btEnabler}
	{\blControlling}
	\BRItem[Versión] 1.0.
	\BRItem[Estado] Aprobado.
	\BRItem[Propuesta por] Víctor David Olvera Pérez.
	\BRItem[Revisada por] \cdtEmpty
	\begin{itemize}
		\item Rafael Landa Aguirre.
	\end{itemize}
	\BRItem[Descripción] Debe llevarse un registro por cada vuelo realizado 
	por el drone. Tomando en cuenta lo siguiente.
	\begin{itemize}
		\item Fecha de solicitud.
		\item Hora de comienzo de vuelo.
		\item Ruta o trayectoria de vuelo en coordenadas geográficas.
	\end{itemize}
	\BRItem[Motivación] Tener información sobre el sistema interno del 
	drone.
\end{BusinessRule}

%===============================================================================
% RN: RN-MON-N09 Calcular distancia a recorrer
%
\begin{BusinessRule}{RN-MON-N09}{Calcular distancia a recorrer}
	{\bcDerivation}
	{\btTimer}
	{\blControlling}
	\BRItem[Versión] 1.0.
	\BRItem[Estado] Aprobado.
	\BRItem[Propuesta por] Víctor David Olvera Pérez.
	\BRItem[Revisada por] \cdtEmpty
	\begin{itemize}
		\item Alan Fernando Rincón Vieyra.
		\item Rafael Landa Aguirre.
	\end{itemize}
	\BRItem[Descripción] La distancia a recorrer es la distancia que existe 
	entre la coordenada de la ubicación inicial y la final. Se calcula 
	aplicando la fórmula de la distancia entre dos puntos.
	\BRItem[Sentencia] .
	\begin{equation*}
	Distancia(p_{1}, p_{2}) = \sqrt{\left ( p_{1}.x - p_{2}.x \right )^2 + 
	\left ( p_{1}.y - p_{2}.y \right )^2}
	\end{equation*}
	Donde:
		\begin{itemize}
			\item \texttt{$p_{1}:$} Coordenada inicial.
			\item \texttt{$p_{2}:$} Coordenada final.
			\item \texttt{$x:$} Latitud geográfica.
			\item \texttt{$y:$} Longitud geográfica.
			\item \texttt{$Distancia:$} La distancia a encontrar.
		\end{itemize}
	\BRItem[Motivación] Saber cuánta distancia se recorre en un vuelo.
\end{BusinessRule}

%===============================================================================
% RN: RN-MON-N10 Calcular tiempo en vuelo
%
\begin{BusinessRule}{RN-MON-N10}{Calcular tiempo en vuelo}
	{\bcDerivation}
	{\btTimer}
	{\blControlling}
	\BRItem[Versión] 1.0.
	\BRItem[Estado] Aprobado.
	\BRItem[Propuesta por] Víctor David Olvera Pérez.
	\BRItem[Revisada por] \cdtEmpty
	\begin{itemize}
		\item Alan Fernando Rincón Vieyra.
		\item Rafael Landa Aguirre.
	\end{itemize}
	\BRItem[Descripción] El tiempo en vuelo se refiere al tiempo que lleva 
volando el drone desde que empieza el recorrido hasta el momento actual. Se 
calcula haciendo una diferencia del tiempo actual menos el tiempo inicial.
	\BRItem[Sentencia]
	\begin{equation*}
	TiempoEnVuelo = T_{act} - T_{ini}
	\end{equation*}
	Donde:
		\begin{itemize}
		\item \texttt{$T_{ini}:$} Tiempo inicial registrado al momento 
		de iniciar el vuelo.
		\item \texttt{$T_{act}:$} Tiempo actual del sistema.
		\item \texttt{$TiempoEnVuelo:$} Tiempo recorrido hasta ese 
		momento.
		\end{itemize}
	\BRItem[Motivación] Saber el tiempo que tardó en trasladarse.
\end{BusinessRule}

%===============================================================================
% RN: RN-MON-N11 Correo electrónico único
%
\begin{BusinessRule}{RN-MON-N11}{Correo electrónico único}
	{\bcIntegridad}
	{\btEnabler}
	{\blControlling}
	\BRItem[Versión] 1.0.
	\BRItem[Estado] \ Aprobado
	\BRItem[Propuesta por] Rafael Landa Aguirre.
	\BRItem[Revisada por] \cdtEmpty
	\begin{itemize}
		\item \ Victor David Olvera Perez
	\end{itemize}
	\BRItem[Descripción] El correo electrónico que se registre por usuario
	deberá ser único, es decir, no habrá duplicados.
	\BRItem[Motivación] Evitar duplicados en datos registrados por usuario, 
	con el fin de identificar un usuario en particular.
\end{BusinessRule}

%===============================================================================
% RN: RN-MON-N12 Nombre de usuario único
%
\begin{BusinessRule}{RN-MON-N12}{Nombre de usuario único}
	{\bcIntegridad}
	{\btEnabler}
	{\blControlling}
	\BRItem[Versión] 1.0.
	\BRItem[Estado] \ Aprobado
	\BRItem[Propuesta por] Rafael Landa Aguirre.
	\BRItem[Revisada por] \cdtEmpty
	\begin{itemize}
		\item \ Victor David Olvera Perez
	\end{itemize}
	\BRItem[Descripción] El nombre de usuario que se registre por usuario
	deberá ser único, es decir, no habrá duplicados.
	\BRItem[Motivación] Evitar duplicados en datos registrados por usuario, 
	con el fin de identificar un usuario en particular.
\end{BusinessRule}

%===============================================================================
% RN: RN-MON-N13 Confirmación de contraseña
%
\begin{BusinessRule}{RN-MON-N13}{Confirmación de contraseña}
	{\bcIntegridad}
	{\btEnabler}
	{\blControlling}
	\BRItem[Versión] 1.0.
	\BRItem[Estado] \ Aprobado
	\BRItem[Propuesta por] Rafael Landa Aguirre.
	\BRItem[Revisada por] \cdtEmpty
	\begin{itemize}
		\item Alan Fernando Rincón Vieyra.
		\item Víctor David Olvera Pérez.
	\end{itemize}
	\BRItem[Descripción] La longitud de la contraseña deberá ser mínimo de 8 
	caracteres y máximo 50 caracteres alfanuméricos. La confirmación de 
	contraseña deberá ser idéntica a la contraseña original.
	\BRItem[Sentencia] Expresión regular:
	%%^[a-zA-Z0-9.!#$%&'*+/=?^_`{|}~-]{8,50}$. 
	\begin{center}
		\$\^ \ [a-zA-Z0-9.! \# \$ \% \& '*+/=? \^ \ \_ \ \` \ \{$|$\} 
		\textbackslash -]\{8,50\}\$\$
	\end{center}
	\BRItem[Motivación] La confirmación de la contraseña deberá ser idéntica
	con el objetivo de confirmar la contraseña previamente introducida.
	\BRItem[Ejemplo positivo] Cumplen la regla:
	\begin{itemize}
		\item $CoNtRaSeNa123$.
		\item $C0NtRa\$3n4$.
	\end{itemize}
	\BRItem[Ejemplo negativo] No cumplen con la regla:
	\begin{itemize}
		\item $my pass$.
		\item $ContrasenaContrasenaContrasenaContrasenaContrasena1$.
	\end{itemize}
\end{BusinessRule}

%===============================================================================
% RN: RN-MON-N14 Datos de usuario necesarios para el inicio de sesión
%
\begin{BusinessRule}{RN-MON-N14}{Datos necesarios para el inicio de sesión}
	{\bcIntegridad}
	{\btEnabler}
	{\blControlling}
	\BRItem[Versión] 1.0.
	\BRItem[Estado] Aprobado.
	\BRItem[Propuesta por] Alan Fernando Rincón Vieyra.
	\BRItem[Revisada por] \cdtEmpty
	\begin{itemize}
		\item Rafael Landa Aguirre.
	\end{itemize}
	\BRItem[Descripción] Para que un usuario acceda al sistema, debe 
	introducir la siguiente información:
	\begin{itemize}
		\item Nombre de usuario. (Obligatorio)
		\item Contraseña. (Obligatorio)
	\end{itemize}
	\BRItem[Motivación] El nombre de usuario permite identificar a cada 
	usuario en el sistema ya que es único, la contraseña restringe el acceso 
	de su perfil a cualquier persona.
\end{BusinessRule}

%===============================================================================
% Sección: Modelo de requerimientos.
%

\subsection{Modelado de requerimientos}

Los requerimientos de este proyecto se encuentran divididos en requerimientos de 
usuario y de sistema.
\begin{itemize}
	\item Los requerimientos del usuario los proporciona el cliente 
	especificando qué es lo que desea que el sistema haga, así como las 
	limitaciones operacionales con las que cuenta.
	\item Los requerimientos del sistema describen los servicios del sistema 
	de una forma más detallada y técnica, donde en conjunto el cliente y el 
	contratante determinan estos requerimientos.
\end{itemize}

A su vez, en cada uno de los requerimientos anteriores podemos encontrar una 
subdivisión de requerimientos en funcionales y no funcionales, como se listan a 
continuación:

\begin{itemize}
	\item Los requerimientos funcionales son aquellos que definen cómo debe 
	funcionar el sistema.
	\item Los requerimientos no funcionales son aquellos que especifican 
	criterios de evaluación  de la operación de un servicio informático con 
	fines de uso.
\end{itemize}

Para los requerimientos funcionales se usará la clave RFXY, donde:
\begin{itemize}
	\item RF: Es la clave para todos los Requerimientos Funcionales.
	\item X: Es la letra correspondiente al tipo de requerimiento (“H” - 
	Hardware, “S” - Software).
	\item Y: Es un número consecutivo: 1, 2, 3, $\cdots$
\end{itemize}

Para los requerimientos no funcionales se usará la clave RNFXY, donde:
\begin{itemize}
	\item RNF: Es la clave para todos los Requerimientos no Funcionales.
	\item X: Es la letra correspondiente al tipo de requerimiento (“H” - 
	Hardware, “S” - Software).
	\item Y: Es un número consecutivo: 1, 2, 3, $\cdots$
\end{itemize}

Para los requerimientos del usuario se usará la clave RUXY, donde:
\begin{itemize}
	\item RU: Es la clave para todos los Requerimientos del Usuario.
	\item X: Es la letra correspondiente al tipo de requerimiento (“H” - 
	Hardware, “S” - Software).
	\item Y: Es un número consecutivo: 1, 2, 3, $\cdots$
\end{itemize}

Además, se usan las abreviaciones que se muestran en la tabla 
\ref{tab:leyenda:RFs}. La prioridad describe el grado de importancia (relevancia 
y urgencia) que tiene para el negocio que el sistema cumpla con dicho 
requerimiento.

\begin{table}[H]
	\centering 
	\caption{Leyenda para los requerimientos funcionales.}
	\label{tab:leyenda:RFs}
	\begin{tabular}{ | l | l |}
	\hline
	$\textbf{Abreviación}$ & $\textbf{Descripción}$ \\\hline
	Id & Identificador del requerimiento. \\\hline
	Pri. & Prioridad \\\hline
	Ref. & Referencia a los Requerimientos de usuario. \\\hline
	MA & Prioridad Muy Alta. \\\hline
	A & Prioridad Alta. \\\hline
	M & Prioridad Media. \\\hline
	B & Prioridad Baja. \\\hline
	MB & Prioridad Muy Baja. \\\hline
	\end{tabular}
\end{table}

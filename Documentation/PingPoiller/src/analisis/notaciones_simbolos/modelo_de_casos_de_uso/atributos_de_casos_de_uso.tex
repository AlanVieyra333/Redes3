%===============================================================================
% Sección: Atributos de casos de uso.
%

\subsubsection{Atributos de casos de uso}

Para entender un caso de uso más allá de un diagrama se utilizan secciones 
en el documento para cada uno, dividido por capítulos de acuerdo al subsistema y 
una tabla con los siguientes datos:

\begin{itemize}
	\item \textbf{Id:} Identificador.
	\item \textbf{Nombre:} Nombre del caso de uso el cual es descriptivo 
	basándose en la tarea que se realiza.
	\item \textbf{Actores:} Lista de los actores relacionados o que utilizan 
	el caso de uso.
	\item \textbf{Descripción:} Es un resumen en el que se especifica la 
	transacción.
	\item \textbf{Propósito:} Indica el objetivo por el cual el caso de uso 
	existe.
	\item \textbf{Entradas:} Lista los datos de entrada que el caso de uso 
	recibe.
	\item \textbf{Salidas:} Lista los datos de salida que genera el caso de 
	uso.
	\item \textbf{Origen:} Indica de donde provienen los datos de entrada.
	\item \textbf{Destino:} Indica a dónde se dirigen los datos de salida.
	\item \textbf{Precondiciones:} Enlista las cosas que deben haber 
	sucedido para que el caso de uso se lleve a cabo.
	\item \textbf{Postcondiciones:} Enlista las cosas que suceden en el 
	sistema o negocio de forma inmediata o a corto plazo.
	\item \textbf{Errores:} Enlista las situaciones en las que el caso de 
	uso concluye sin éxito.
	\item \textbf{Reglas de Negocio:} Lista las reglas de negocio que están 
	asociadas al caso de uso.
	\item \textbf{Disparadores:} Aquellas condiciones o motivaciones que el 
	actor tiene para ejecutar el caso de uso.
	\item \textbf{Condición de Término:} Aquellos resultados que garantizan 
	que el caso de uso se ha ejecutado con éxito.
	\item \textbf{Efectos Colaterales:} Afectaciones en otras partes del 
	sistema.
	\item \textbf{Viene de:} Indica cuando el caso de uso se extiende de 
	otro o se incluye en otro.
\end{itemize}

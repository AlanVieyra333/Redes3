%===============================================================================
% Sección: Identificadores de casos de uso.
%

\subsubsection{Identificadores de casos de uso} 

Los casos de uso se identificarán de acuerdo con la siguiente nomenclatura:

\begin{equation*}
Subsistema - \underset{Modulos}{Division} - CU + \underset{Identificador}{Id} + 
Nombre
\end{equation*}

En donde subsistema puede ser:
\begin{itemize}
	\item \textbf{VA:} Vuelo Autónomo. El drone será capaz de controlar la 
	ruta establecida por el cliente, usando el sensor ultrasónico que le 
	permitirá evadir obstáculos y llegar a su destino. \\
	Para el subsistema Vuelo Autónomo existe la división de módulos 
	siguiente:
	\begin{itemize}
		\item \textbf{PR:} Proximidad.
		\item \textbf{CM:} Control de Movimientos.
	\end{itemize}
	\item \textbf{MON:} Monitoreo. El cliente podrá visualizar información 
	sobre el vuelo realizado, teniendo comunicación con el drone a través de 
	internet. Donde división puede ser únicamente el módulo, o bien, el 
	módulo puede tener submódulos. \\
	Para el subsistema Monitoreo existe la división de módulos siguiente:
	\begin{itemize}
		\item \textbf{AS:} Acceso al Sistema.
		\item \textbf{CD:} Comunicación con Drone.
		\item \textbf{GV:} Gestión de Vuelo.
		\item \textbf{GH:} Gestión de Historial.
	\end{itemize}
\end{itemize}

Ejemplo que cumple con la nomenclatura:
\begin{itemize}
	\item \texttt{VA-PR-CU1} Obtener distancia de objeto.
	\item \texttt{MON-GV-CU4.1} Asignar ubicación.
\end{itemize}

Ejemplo que no cumple con la nomenclatura:
\begin{itemize}
	\item \texttt{CU7-VA-CM} Obtener distancia de objeto.
	\item \texttt{MON-CU-IS-3.1} Asignar ubicación.
\end{itemize}

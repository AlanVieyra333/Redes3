%===============================================================================
% Sección: Introducción.
%

Los diagramas de casos de uso son una herramienta usada para representar las 
transacciones entre un actor y el sistema, las cuales siempre tendrán un valor 
agregado o un propósito para que el actor las realice. En estos diagramas se 
podrán observar los siguientes elementos:

\begin{itemize}
	\item \UCsist Representa al sistema. Es un óvalo e indica las acciones 
	definidas a realizar por el sistema dentro de las trayectorias de casos 
	de uso.
	\item \UCactor Representa al actor.
	\item Relación $<- - -<< extends >>- - -$. Indica que un caso de uso 
	puede ejecutarse a partir de otro.
	\item Relación $- - -<< include >>- - ->$. Indica que un caso de uso 
	debe ejecutarse a partir de otro.
\end{itemize}

La conexión entre un actor y un caso de uso es por medio de una línea como se 
muestra en la figura \ref{fig:actor:cu}.

\begin{figure}[H]
	\begin{center} 
		\includegraphics[width=.4\textwidth]
		{images/doc/img_actor_con_caso_de_uso}
		\caption{Un Actor y un caso de uso.}
		\label{fig:actor:cu}
	\end{center}
\end{figure}

Los casos de uso se encontrarán dentro de paquetes (representados por carpetas) 
indicando así que pertenecen a un mismo módulo, por ejemplo, el módulo del 
Restaurante como se muestra en la figura \ref{fig:mod:restaurante}.

\begin{figure}[H]
	\begin{center} 
		\includegraphics[width=.6\textwidth]
		{images/doc/img_modulo_restaurante}
		\caption{Un Actor con varios caso de uso dentro de un módulo.}
		\label{fig:mod:restaurante}
	\end{center}
\end{figure}

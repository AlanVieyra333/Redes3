%===============================================================================
% Sección: Modelo de reglas de negocio.
%

\subsection{Modelado de reglas de negocio}

Las reglas de negocio son directivas que tienen como fundamento la misión de un 
negocio y como objetivo regir estrategias para poder conseguir esta misión. Una 
regla de negocio no requiere de una interpretación adicional. Estas reglas 
también son una fuente de información muy relevante ya que generalmente 
establecen las relaciones entre dos o más términos del negocio.
En el documento se presentarán estas reglas en forma de secciones indicando los 
siguientes atributos:

\begin{itemize}
	\item \textbf{Id:} Es el identificador de la regla de negocio con la 
	cual se podrá referenciar a lo largo del documento.
	\item \textbf{Nombre:} Indica el nombre de la regla de negocio el cual 
	debe describir de forma concisa en qué consiste la regla.
	\item \textbf{Tipo:} Indica el tipo de regla de negocio de acuerdo a 
	cómo se aplica.
	\begin{itemize}
		\item \textbf{Habilitadora:} Permite realizar el proceso en el 
		que la regla se ve involucrada.
		\item \textbf{Cronometrada:} Recibe parámetros y con respecto a 
		eso realiza el proceso.
		\item \textbf{Ejecutiva:} Es aquella que se debe llevar a cabo 
		cuando una autoridad se ve involucrada para que el proceso 
		concluya.
	\end{itemize}
	\item Clase: Indica la naturaleza de la regla de negocio.
		\begin{itemize}
		\item \textbf{Condición:} Es una regla que cumple una condición 
		para llevarse a cabo.
		\item \textbf{Integridad Referencial:} Es una regla que indica 
		validaciones que, de no ser tomadas en cuenta se pondría en 
		peligro la integridad de la información.
		\item \textbf{Autorización:} Son restricciones en las que se ven 
		involucradas palabras como, al menos uno.
		\end{itemize}
	\item \textbf{Nivel:} Indica cómo es que la regla se toma en cuenta para 
	el desarrollo del sistema.
		\begin{itemize}
		\item \textbf{Controla:} Define que el sistema se encargará de 
		vigilar el cumplimiento de la regla en todo momento.
		\item \textbf{Influencia:} Sugiere formas en las que debería 
		realizarse la operación, pero no la limita. De tal forma que el 
		sistema dará facilidades para evitar esas situaciones o 
		advertirá cada vez que se detecte que la regla no es tomada en 
		cuenta.
		\end{itemize}
	\item \textbf{Descripción:} Es un pequeño resumen que ayuda a entender 
	la regla de negocio.
	\item \textbf{Motivación:} Razón por la que existe la regla de negocio.
	\item \textbf{Sentencia:} Descripción formal o matemática de la regla de 
	negocio.
\end{itemize}

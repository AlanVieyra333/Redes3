%===============================================================================
% Sección: Introduccion al modelo de mensajes.
%

Los mensajes son elementos en el sistema que ayudan a mantener pendiente al 
actor, ya sea para indicarle que la operación se llevó a cabo correctamente, si 
existe un error, o si es solo para informar de alguna acción que ejecuta el 
sistema internamente. \\

Cuando un mensaje es recurrente se parametrizan sus elementos, por ejemplo los mensajes: ``Aún no existen registros de \textit{drones} en el sistema.'', ``Aún no existen registros de \textit{vuelos} en el sistema.'', tienen una estructura similar 
por lo que, con el objetivo de que el mensaje sea genérico y pueda utilizarse en todos los casos de uso que se considere necesario, se utilizan parámetros para definir el mensaje.\\

Los parámetros también se utilizan cuando la redacción del mensaje tiene datos que son ingresados por el actor o que dependen del resultado de la operación, por ejemplo: 
``El \textit{usuario} ha sido \textit{registrado} exitosamente.''. En este caso la redacción se presenta parametrizada de la forma: ``$<ARTICULO> + <ENTIDAD>$ ha sido $<OPERACION>$ exitosamente.'' y los parámetros se describen de la siguiente forma:

\begin{itemize}
	\item ARTICULO: Es un artículo determinado.
	\item ENTIDAD: Es el valor de la entidad sobre la cual se realizó la acción.
	\item OPERACIÓN: Es la acción que el actor solicitó realizar.
\end{itemize}

Cada mensaje enlista los parámetros  que utiliza, sin embargo aquí se definen los más comunes a fin de simplificar la descripción de los mensajes:

\begin{description}
	\item [$<ARTICULO>$:] Se refiere a un {\em artículo} el cual puede ser DETERMINADO (El $\mid$ La $\mid$ Lo $\mid$ Los $\mid$ Las) o INDETERMINADO (Un $\mid$ Una $\mid$ 
	Uno $\mid$ Unos $\mid$Unas) se aplica generalmente sobre una ENTIDAD, ATRIBUTO o VALOR.
	
	\item [$<CAMPO>$:] Se refiere a un campo del formulario. Por lo regular es el nombre de un atributo en una entidad.
	
	\item [$<CONDICION>$:] Define una expresión booleana cuyo resultado deriva en {\em falso} o {\em verdadero} y suele ser la causa del mensaje.
	
	\item [$<DATO>$:] Es un sustantivo y generalmente se refiere a un atributo de una entidad descrito en el modelo estructural del negocio, por ejemplo: nombrem, nombre de usuario, correo, etc. %ATRIBUTO
	
	\item [$<ENTIDAD>$:] Es un sustantivo y generalmente se refiere a una entidad del modelo estructural del negocio, por ejemplo: cliente, vuelo, etc.
	\item [$<OPERACION>$:] Se refiere a una acción que se debe realizar sobre los datos de una o varias entidades. Por ejemplo: registrar, eliminar, actualizar, por mencionar algunos. Comúnmente la OPERACIÓN va concatenada con el sustantivo, por ejemplo: Registro de un nuevo usuario, solicitar un vuelo, eliminar un vuelo del historial y demás.
	
	\item [$<VALOR>$:] Es un sustantivo concreto y generalmente se refiere a un valor en específico. Por ejemplo: \textit{Correo} es un \textbf{valor} concreto de la \textbf{entidad} \textit{Cliente}.
	
\end{description}
%===============================================================================
% Sección: Atributos de mensajes.
%

\subsubsection{Atributos de mensajes}

Un mensaje se conformará de:
\begin{itemize}
	\item \textbf{Tipo:} Indica a qué clase pertenece un mensaje, entre las cuales se 
	encuentran:
	\item \textbf{Informativo:} Notificar al actor que una operación concluyó o para 
	notificar al actor de una operación realizada en el sistema.
	\item \textbf{Confirmación:} Requiere la confirmación del actor para llevar a 
	cabo una operación en el sistema.
	\item \textbf{Error:} Notificar al actor que la operación no se llevó a cabo 
	adecuadamente o no se llevó a cabo.
	\item \textbf{Alerta:} Notificar al actor que una operación requiere de su 
	atención para llevarse a cabo.
	\item \textbf{Canal:} Indica el medio que el sistema utiliza para enviar un 
	mensaje.
	\item \textbf{Propósito:} Indica la razón por la cual el mensaje es necesario.
	\item \textbf{Redacción:} Indica el texto que el actor podrá visualizar cuando el 
	mensaje se muestre.
	\item \textbf{Parámetros:} Indica un elemento dentro del texto del mensaje que 
	puede cambiar dinámicamente de acuerdo a la operación que se esté 
	llevando a cabo.
	\item \textbf{Ejemplo:} Muestra una aplicación del mensaje.
\end{itemize}

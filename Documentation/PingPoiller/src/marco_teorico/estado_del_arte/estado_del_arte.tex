% Estado del arte.
% Autor(es): Landa Aguirre Rafael
%
% Fecha: 23 - Feb - 2018.
%

\section{Estado del Arte}

Usando herramientas como la internet se buscaron prototipos o sistemas r
ealizados por empresas comerciales ó universidades que han logrado dar a conocer 
al público los diseños de drones. Con esto se comprueba que hay, en otros 
países, diseños más avanzados debido a la gran inversión que se realiza, y 
además los diferentes dispositivos que adaptan al drone, tales como sensores, 
procesadores más rápidos, los cuáles hacen que la autonomía del drone sea más 
eficiente.

\begin{longtable}[c]{| p{4cm} | p{4cm} | p{6.5cm} |}
	\caption{Ejemplos de prototipos previamente diseñados en algún proyecto 
		para alguna organización. \label{tab:edo:arte:proyectos}} 
	\\\hline
	\textbf{Nombre} & \textbf{Tipo} & \textbf{Características}
	\\\hline \hline
	\endfirsthead

	\hline
	\endhead

	\hline
	\endfoot

	\hline
	\endlastfoot

	Drone autónomo de Boeing &
	Vehículo desarrollado y presentado para la 
	empresa Boeing. & 
	Prototipo desarrollado en octubre del año 
	2015. Se trata de un drone de ocho motores completamente 
	eléctrico y que utiliza baterías desarrolladas por las propia 
	Boeing y diseñado con la idea de construir plataformas de carga 
	autónoma de gran envergadura: con capacidad para cargar con 
	entre 120 y 230 kg, un alcance de entre 15 y 30 km, cambiar la 
	manera en el que se entregan mercancías.
	\\\hline
	Sistema de control que permite el vuelo autonomo 
	de drones. &
	Sistema de  control desarrollado por la 
	universidad de Alicante. &
	Este sistema permite que el vehículo puede 
	alternar entre distintos planes de vuelo o definir los 
	desplazamientos óptimos en función del entorno y la información 
	recibida por sus sensores. 
	Usa el protocolo MavLink para la comunicación entre el drone 
	y este sistema e incorpora sensores que son compatibles con USB, 
	I2C, o RS232. 
	Existe una comunicación con el exterior a una base externa, la 
	cual permite enviar datos de vuelo, e información recogida por 
	los sensores.
	Incorpora en su diseño misiones peligrosas en las que el 
	tiempo de respuesta es bastante mayor o entornos distantes en 
	que la comunicación se puede interrumpir.
	\\\hline
	Planeador de misión y módulo de vuelo guiado 
	para un Drone. &
	Planeador de misión y módulo de vuelo ralizado 
	en el Instituto Politécnico Nacional. &
	Software que mediante el uso de un mapa 
	permita introducir una serie de coordenadas geográficas las 
	cuales sirven como referencia para los sistemas GPS llamados 
	waypoints, de los que se obtiene información básica como 
	latitud, longitud y opcionalmente altitud, mediante estos 
	waypoints el usuario podrá diseñar rutas que posteormente serán 
	guía para recorrido del drone.
	\\\hline
	Sistema de control y operación a distancia de un vehículo aéreo no 
	tripulado &
	Control y operación a distancia de un vehículo aéreo no tripulado
	desarrollado en el Instituto Politécnico Nacional (UPIITA) &
	El sistema de comunicaciones está diseñado sobre la banda UHF (900MHz), 
	debido a las ventajas que ofrece trabajar en ella, el cual permite que 
	los movimientos del UAV puedan ser controlados remotamente por un 
	operador, a través de una palanca de mando conectada a una PC que 
	representa la Estación Terrena, que será capaz de enviar la información 
	necesaria para controlar la trayectoria del vehículo, la cual será 
	monitoreada a través de una interfaz gráfica amigable de dónde se podrán 
	visualizar distintas variables de interés como son los datos 
	proporcionados por el GPS, la Temperatura del motor del 
	mini-helicóptero.
	\\\hline

 \end{longtable}

%% Referencias:
%% https://sgitt-otri.ua.es/es/empresa/documentos/ot-1502-drones.pdf
%% http://www.microsiervos.com/archivo/drones/dron-autonomo-boeing-capaz-transportar-230-kg.html
%% http://www.elmundo.es/economia/2017/05/04/590b6961e2704e34398b4618.html
.
\\ \\

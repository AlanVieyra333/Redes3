%%==============================================================================
% Microcontroladores
%
\section{Microcontroladores} 
Los microcontroladores hoy en día ofrecen bastante funcionalidad, ya que se 
pueden agregar otros dispositivos externos, ahorrar espacio y tiempo ya que 
internamente tiene agregada una memoria e interfaces de puertos (ya sean 
seriales o de entrada y salida) para que se puedan recibir datos externos. \\
El microcontrolador es un integrado que contiene un microprocesador, memoria (de 
programa y datos) y unidades de entrada/salida (puertos paralelos, 
temporizadores, conversores Analógico/Digital (A/D), Digital/Analógico (D/A), 
puertos serie, etc.) \cite{comparativa_micros}.

%% Clasificación de microcontroladores
%
\subsection{Clasificación}
Los microcontroladores se pueden clasificar de acuerdo con 
\cite{comparativa_micros}:

\begin{itemize}
	\item Ancho de palabra: de 4, 8, 16, 32 bits.
	\item Periféricos: A/D, D/A, Entrada/Salida,
		timers, etc.
	\item Especialidad: procesamiento de señales, video, comunicaciones, 
		etc.
\end{itemize}

%% Arquitectura básica
%
\subsection{Arquitectura básica}
Inicialmente los microcontroladores adoptaron una arquitectura Von Neumann, pero 
en la actualidad la arquitectura usada es arquitectura Harvard. \\
La arquitectura Von Neumann se caracteriza principalmente por tener una sola 
memoria principal y datos en la cual se almacenan instrucciones de forma 
indistinta. A dicha memoria se accede con un único bus de direcciones, control y 
datos \cite{comparativa_micros, arqui_org_micros}. \\
La arquitectura Harvard se caracteriza por tener la memoria de datos y de 
instrucciones separadas, y tiene la ventaja de que existen buses respectivos 
para cada memoria y se puede acceder desde el procesador simultáneamente sin 
necesidad de esperar al siguiente ciclo de instrucción \cite{arqui_org_micros}. 
\\
El procesador interno del microcontrolador o CPU (Unit Control Processor) es el 
elemento más importante, y se encarga de direccionar la memoria de 
instrucciones, recibir la instrucción en curso y su decodificación para llevar a 
cabo la ejecución de la operación que implica dicha instrucción, para finalmente 
almacenar el resultado de toda la operación \cite{arqui_org_micros}. \\
Existen tres tipos de instrucciones que caracterizan a los microprocesadores o 
microcontroladores actuales \cite{arqui_org_micros}:

\begin{itemize}
	\item CISC (Conjunto conplejo de instrucciones): Se caracterizan por 
	disponer de más de 80 instrucciones máquina, las cuáles son potentes y 
	pueden requerir más ciclos de instrucción para su ejecución. Ofrecen al 
	programador de realizar instrucciones como macros.
	\item RISC (Conjunto reducido de instrucciones): Ofrecen conjunto de 
	reducido de instrucciones y son simples, se ejecutan en un solo ciclo de 
	instrucción. Permiten optimizar el hardware y software.
	\item SISC (Conjunto simple de instrucciones): Estas instrucciones se 
	destinan a aplicaciones bastantes concretas, el conjunto de 
	instrucciones es reducido, además de ser específico, y se adaptan a las 
	necesidades de la aplicación.
\end{itemize}

\textbf{Memoria} \\
En los microcontroladores la memoria está integrada para almacenar datos e 
instrucciones; ambas partes debería ser volátil y la otra no volátil. La parte 
no volátil debe estar compuesta por una memoria ROM (memoria de solo lectura) y 
la otra parte en la cual se almacenan variables, y datos, está integrada por una 
memoria RAM (memoria de lectura y escritura de datos) 
\cite{info_microcontroladores}. \\
Las diferentes memorias que integran al microcontrolador, son: 

\begin{itemize}
	\item Memoria ROM con máscara. Memoria de solo lectura. 
	\item OTP. Memoria programable en tiempo desde una computadora personal 
	o PC.
	\item EPROM. Memoria programable de sólo lectura borrable.
	\item EEPROM. Memoria eléctricamente/borrable programable de solo 
	lectura.
	\item Memoria Flash.
\end{itemize}

\textbf{Puertos de Entrada/Salida (E/S)} \\
Este tipo de interfaces se encargan de comunicar al microcontrolador con el 
exterior, según los controladores de periféricos que posea cada modelo de 
microcontrolador, las líneas de E/S se destinan para proporcionar soporte a 
entradas, salidas y control \cite{info_microcontroladores}. \\
Algunas interfaces de comunicación que posea cada modelo, entre ellas se 
encuentran:

\begin{itemize}
	\item USART. Bus de comunicación serial asíncrono y síncrono.
	\item USB. Bus universal serial de comunicación.
	\item IIC (Inter-Integrated Circuit por sus siglas en 
		inglés). Bus serie de datos.
	\item CAN. Protocolo de comunicaciones basado en una topología bus para 
		la transmisión de mensajes en entornos distribuidos.
\end{itemize}

\textbf{Recursos del microcontrolador} \\
En referencia a \cite{info_microcontroladores}, algunos de los recursos 
específicos con los que cuenta un microcontrolador son:

\begin{itemize}
	\item Temporizadores o Timers. \\
		Llevan a cabo el control de periodos de tiempo y para llevar la 
		cuenta de los acontecimientos que ocurren en el exterior. 
	\item Perro guardián o Watch Dog Timer. \\
		Es un temporizador cuando se bloquea automáticamente, provoca un 
		reinicio. 
	\item Protección ante fallo de alimentación. \\
		Es un circuito cuyo objetivo es hacer reset en el 
		microcontrolador el voltaje es inferior al voltaje de 
		alimentación. 
	\item Estado de bajo consumo. \\
		Cuando el microcontrolador no está funcionando correctamente, 
		existe proceso de bajo consumo para ahorrar energía y pasar al 
		estado de bajo consumo o de reposo. 
	\item Conversor Analógico/Digital. \\
		En los microcontroladores se incorpora un circuito convertidor 
		para lograr convertir señales
		analógicas a digitales.
	\item Convertidor Digital/Analógico. \\
		Además, internamente en el microcontrolador se agrega un módulo 
		convertidor de señales	digitales a analógicas para comunicarse 
		con el exterior. 
	\item Modulador de ancho de pulsos (PWM). \\
		Son circuitos físicos que proporcionan salida de señales de 
		amplitud variable. 
\end{itemize}

%% Aplicaciones
%
\subsection{Aplicaciones de los microcontroladores}
El uso de los microcontroladores es cuando la aplicación no requiere tanta 
potencia de cálculo. Algunas aplicaciones que se les brinda a los 
microcontroladores:

\begin{itemize}
	\item Robótica.
	\item Equipamiento informático.
	\item Sistemas portátiles e informáticos.
	\item Sector doméstico.
\end{itemize}

%% Fabricantes
%
\subsection{Fabricantes de microcontroladores}
Desde el principio del microprocesador hasta la aparición de los primeros 
microcontroladores ha habido diversos fabricantes, ya sea de Japón, EEUU, 
entre otros, como son: Intel, Motorola, Hitachi, Philips, SGS-Thompson, National 
Semiconductor, Zilog, Texas Instruments, Toshiba, Microchip, Atmel 
\cite{comparativa_micros}. \\

\begin{figure}[H]
	\begin{center} 
		\includegraphics[scale=0.5]
		{images/doc/img_fabricantes_microcontroladores}
		\caption{Fabricantes de microcontroladores.}
	\end{center}
\end{figure}

De las cuales los principales fabricantes actuales se destacan los siguientes, 
de acuerdo con el volumen de producción, y diversidad de modelos:

\begin{itemize}
	\item Microchip Technology Corp.
	\item STMicroelectronics.
	\item Atmel Corp.
	\item Motorola Semiconductors Corp.
\end{itemize}

\begin{table}[H]
	\centering 
	\caption{Tabla de fabricantes con los microcontroladores respectivos
		más comúnes que se pueden encontrar en diversas aplicaciones 
		\cite{comparativa_micros}.}
	\begin{tabular*}{0.75\textwidth}{@{\extracolsep{\fill}} |c|l|l|}
		\hline
		$\textbf{Empresa}$ & $\textbf{8 bits}$ & $\textbf{16 bits}$
		\\\hline \hline
		Atmel & \parbox[t]{3.5cm}{AVR (Mega, tiny), 89Sxxxx, 8051} & \\
		\hline
		Microchip & \parbox[t]{3.5cm}{10F2xx, 12Cxx, 12Fxx, 16Cxx, 16Fxx,
		18Cxx, 18Fxx} &
		\parbox[t]{2.8cm}{PIC24F, PIC24H, dsPIC30Fxx, dsPIC33Fxx} \\
		\hline
		STMicroelectronics & ST-62, ST-7 & \\
		\hline
		Freescale(Antes Motorola) & \parbox[t]{3.5cm}{68HC05, 68HC08,
		68HC11,	HCS08} &
		\parbox[t]{2.8cm}{68HC12, 68HCS12, 68HCSX12, 68HC16} \\\hline
	\end{tabular*}
\end{table}

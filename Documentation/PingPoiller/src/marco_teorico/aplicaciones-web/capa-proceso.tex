%===============================================================================
% Capa de servidor.
%

\subsubsection{Capa de servidor}
\label{sec:capaProceso}

Es el conjunto de funcionalidades que ofrecerá la aplicación web, como el 
procesamiento de los datos introducidos por el usuario, el cálculo de resultados 
a partir de distintos datos de entrada, el diseño y ejecución de algoritmos, la 
manipulación de la información almacenada en una base de datos, la ejecución de 
diversas acciones como consecuencia del cumplimiento de diversas condiciones o 
del disparo de algún evento, etc. Es decir, planear y diseñar lo que luego se 
llevará a cabo mediante el uso de uno o varios lenguajes de programación.

% Ref: http://www.daniloaz.com/es/que-es-la-arquitectura-web/

%---------- Java ----------
\subsubsection{Java}
\label{sub:java}

El lenguaje de programación Java es un lenguaje de propósito general, además, es 
válido para realizar todo tipo de aplicaciones profesionales o empresariales 
\cite{lenguaje_java}. \\
Basado en el paradigma orientado a objetos, el cual engloba conceptos como: 
encapsulación, herencia, polimorfismo. \\

\textbf{Características} \\
Java ofrece las siguientes características \cite{lenguaje_java}:

\begin{itemize}
	\item Ofrece un Kit para el  desarrollo de aplicaciones gratis.
	\item Permite escribir applets (los cuales son programas que se insertan 
		en una página HTML), se ejecutan en el ordenador local.
	\item Las aplicaciones con las que se pueden escribir con al lenguaje, 
		entre ellas podemos encontrar las siguientes: intra-redes, 
		cliente/servidor, sistemas distribuidos en redes locales y en 
		internet.
	\item Las aplicaciones son fiables. Se puede controlar su seguridad 
		frente al acceso de los recursos del sistema, además de 
		gestionar permisos y criptografía.
	\item Aprovecha la mayoría de las características de los lenguajes 
		modernos, pero particularmente del lenguaje de programación C++.
	\item Incorporación del concepto de Multi-Threading (o también llamado 
		multihilos), ya que permite la ejecución de múltiples tareas 
		concurrentes.
	\item Aplicaciones independientes (para propósito general).
	\item Applets. Pequeñas aplicaciones que se insertan en una página HTML, 
		siempre y cuando el navegador soporte Java. Hoy en día su uso ya 
		no es tan común.
\end{itemize}

%% Referencias: 
%%

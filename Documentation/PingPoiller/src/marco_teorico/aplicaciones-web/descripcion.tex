%===============================================================================
% Descripción general de una arquitectura web.
%

\subsection{Descripción general}

En el campo de la ingeniería de software se denomina aplicación web a una 
aplicación basada en el modelo cliente-servidor, en el cual el cliente o 
usuario, accede a la aplicación mediante la dirección en la que está ubicado el 
respectivo servidor web a través de internet o de una intranet mediante un 
navegador, es decir, haciendo uso del protocolo HTTP para comunicarse con el 
usuario o con otras aplicaciones web. El protocolo de comunicación HTTP es el 
método más usado para el intercambio de información en la World Wide Web. \\

Estas aplicaciones son populares debido al práctico uso del navegador como 
cliente, a la independencia del sistema operativo y a la facilidad para 
actualizar y mantener dichas aplicaciones sin la necesidad de distribuir o 
instalar software en miles de usuarios potenciales. En la figura
\ref{fig:aplicacion-web:arquitectura} se puede apreciar de manera general la 
arquitectura de una aplicación web 
\cite{the:USB:puntos_venta, the:UDC:webapp_distributivo}.\\

\fig{img_aplicacion_web_arquitectura}{aplicacion-web:arquitectura}{Esquema 
general de una aplicación web}

\pagebreak

Las aplicaciones web se modelan mediante lo que se conoce como modelo de capas. 
Una capa representa un elemento que procesa o trata información. Entre los 
principales tipos se encuentran: \\

\begin{itemize}
	\item \textbf{Modelo de dos capas.} La información atraviesa dos capas 
	entre la interfaz y la administración de los datos. Gran parte de la 
	aplicación corre en el lado del cliente como se muestra en la figura 
	\ref{fig:aplicacion-web:dosCapas}.

\fig{img_aplicacion_web_dos_capas}{aplicacion-web:dosCapas}{Modelo de dos capas}

	\item \textbf{Modelo de n-capas.} La información atraviesa varias capas. 
	El más habitual es el modelo de tres capas, la cual está diseñada para 
	superar las limitaciones de las arquitecturas ajustadas al modelo de dos 
	capas, introduce una capa intermedia (la capa de proceso) entre 
	presentación y los datos como se muestra en la figura 
	\ref{fig:aplicacion-web:tresCapas}. Los procesos pueden ser manejados de 
	forma separada a la interfaz de usuario y a los datos, esta capa 
	intermedia centraliza la lógica de negocio, haciendo la administración 
	más sencilla, los datos se pueden integrar de múltiples fuentes, las 
	aplicaciones web actuales se ajustan a este modelo.

\fig{img_aplicacion_web_tres_capas}{aplicacion-web:tresCapas}{Modelo de tres 
capas.}

\end{itemize}

\newpage

%===============================================================================
% Capa de persistencia.
%

\subsubsection{Capa de persistencia}
\label{sec:capaPersistencia}

Determina la información del mundo real que tendrá que tratar una aplicación, 
diseñar un modelo conceptual que sea un fiel reflejo de dicho mundo real con sus 
distintas entidades y relaciones entre dichas entidades, determinar el modelo de 
datos que mejor se adapte a dicho modelo conceptual, implementar ese modelo de 
datos sobre un motor de bases de datos concreto y trasladar a él la información 
necesaria para el correcto funcionamiento de nuestra aplicación.  \\

% Ref: http://www.daniloaz.com/es/que-es-la-arquitectura-web/

%% Herramientas
\textbf{Herramientas}
Las herramientas fundamentales que nos ofrece un sistema manejador de base de 
datos independientemente de qué modelo maneje son las siguientes 
\cite{db:masters:oficiales:doc}:

\begin{itemize}
	\item Estructura de datos. Son aquellas con las que se puede construir 
		una base de datos, tales como: tablas, grafos (árboles binarios 
		principalmente), estructura de directorios, entre otros.
	\item Integridad de la información. Se tienen que cumplir las siguientes 
		reglas: dominios, claves, llaves foráneas que referencian a 
		otras estructuras de almacenamiento de información, etc.
	\item Operaciones sobre datos. Visualizar, insertar, actualizar, borrar 
		información en un determinado tiempo o momento.
\end{itemize}

%---------- Base de datos relacionales ----------
%
\textbf{Base de datos basadas en modelo relacional}
\label{sub:bdRelacionales}

%% Información general
Término introducido por primera vez por el científico inglés Edgar Frank Codd 
(1970), las bases de datos son una representación de la realidad. Se puede 
considerar como un modelo de la realidad, y el componente fundamental para poder 
representar ese modelo de la realidad en un sistema gestor de base de datos son 
las tablas \cite{bases_relacionales_prin_dis}. \\

Las bases de datos relacionales están basadas en el modelo  entidad-relación, ya 
que este modelo hace referencia a una representación conceptual y abstracta de 
datos, metodología del modelado de la base de datos, y principalmente se 
distinguen tres objetos básicos para poder entender y manejar una base de datos 
bajo este modelo \cite{db:ibm:center}:

\begin{itemize}
	\item Relaciones entre entidades.
	\item Atributos o características propias de las entidades.
	\item Entidades u objeto abstracto que hace representación de la vida 
		real.
\end{itemize}

%% Características
Tiene por objetivo \cite{bases_relacionales_prin_dis}:

\begin{itemize}
	\item Borrado de datos.
	\item Edición de datos.
	\item Inserción de datos.
	\item Los tipos de datos que se tienen que manejar en cualquier sistema 
		gestor de base de datos son: tipos numéricos, binarios, fechas, 
		booleanos, textos, caracteres, tipos geométricos.
	\item Dominio de valores.
	\item Normalización entre entidades.
\end{itemize}

%% Manejadores de BD existentes
\begin{table}[H]
	\centering 
	\caption{Ejemplos de distintos manejadores de base de datos
		relacionales y sus características 
		\cite{db:gestores:tics}.}
	\begin{tabular*}{0.9\textwidth}{@{\extracolsep{\fill}} |l|l|}
		\hline
		$\textbf{Sistema gestor}$ & $\textbf{Descripción}$
		\\\hline \hline
		\parbox[t]{3cm}{Microsoft Access} & 
		\parbox[t]{10.2cm}{
		\begin{itemize}
		\item Propietaria de Microsoft.
		\item Usada para la definición y manipulación de bases de datos.
		\item Entorno de programación a través del administrador Visual 
			Basic.
		\end{itemize}
		} 
		\\\hline
		\parbox[t]{3cm}{MySQL} & \parbox[t]{10.2cm}{
		\begin{itemize}		
		\item Veloz al realizar operaciones. 
		\item Bajo costo en requerimientos de operaciones.
		\item Fácil de configurar e instalar.
		\end{itemize}
		} 
		\\\hline
		\parbox[t]{3cm}{Oracle} & \parbox[t]{10.2cm}{
		\begin{itemize}
		\item Desarrollada por Oracle Corporation. 
		\item Incorpora el lenguaje PL-SQL (Programming 
			language-Structured Query Language).
		\item Escalable.
		\item Estabilidad.
		\item Multiplataforma.
		\end{itemize}
		} 
		\\\hline
	\end{tabular*}
\end{table}

%---------- Base de datos semiestructuradas ----------
%
\textbf{Base de datos basadas en modelo semiestructurado}
\label{sub:bdRelacionales}
%% Información general
Se consideran como una especie de bases de datos, pero no pueden, ya tienen 
ausencia de un esquema de representación de datos \cite{db:ciencias:computacion}. \\
Además, la existencia de bases de datos semiestructuradas se debe a la necesidad 
de intercambio de información, e integración de datos entre aplicaciones 
empresariales de distinta arquitectura. También como resultado entre ese 
intercambio se requiere hacer browsing o navegación del documento para realizar 
búsquedas de datos almacenados. \\

Características de bases de datos semiestructuradas
\begin{itemize}
	\item No se respeta un esquema en particular.
	\item Rápida evolución de la estructura de la información.
	\item El tipado de los datos puede estar cambiando constantemente, es 
		decir, son  débilmente tipados.
	\item La inserción de nuevos datos pueden no respetar la estructura de 
		datos almacenada previamente.
\end{itemize}

%% Manejadores de BD existentes
\begin{table}[H]
	\centering 
	\caption{Ejemplos de base de datos semiestructuradas y sus 
		características \cite{db:ciencias:computacion}.}
	\begin{tabular*}{0.9\textwidth}{@{\extracolsep{\fill}} |l|l|}
		\hline
		$\textbf{Sistema}$ & $\textbf{Descripción}$
		\\\hline \hline
		\parbox[t]{3cm}{XML (Extensible Markub Language)} & 
		\parbox[t]{10.2cm}{
		\begin{itemize}
		\item Metalenguaje que nos permite definir lenguajes de marcado 
			adecuados a usos determinados. 
		\item Es un lenguaje parecido a HTML(Hypertext Markub Language).
		\item XML es personalizable y es ampliable o extensible. 
		\item Se puede incluir todo tipo de información incluyendo 
			información no estructurada como datos de un archivo. 
		\item Se representan los datos con algún modelo ya sea de tipo 
			grafo (árboles, grafos dirigidos, etc). 
		\item Se parte de listas de asociación con etiqueta valor.
		\end{itemize}
		} \\\hline
	\end{tabular*}
\end{table}

%---------- Base de datos no relacionales ----------
%
%\label{sub:bdNoRelacionales}
\textbf{Base de datos basadas en el modelo orientado a objetos} \\
%% Información general
Esta clase de bases de datos hacen referencia a algo llamado "Bases de datos 
NoSQL (Not Only Structured Query Language)", con esto no quiere decir que no 
usen un lenguaje de consultas SQL, más bien se dice así, ya que no hacen uso de 
características que usan las bases de datos relacionales, y entre ellas está que 
dentro del lenguaje de consultas permiten hacer concatenaciones u operaciones de 
conjuntos entre tablas, así llamado JOIN (juntar por su traducción en inglés) 
\cite{inv_storage_data}.

%% Características
\textbf{Características de las bases de datos no relacionales} \\
Todo sistema gestor de bases de datos, pero en específico para estos, deben 
cumplir los siguientes requerimientos \cite{inv_storage_data}:

Se debe de cumplir el teorema CAP (Consistencia, Disponibilidad y Partición), 
como se ve a continuación:
\begin{itemize}
	\item Cualquier nodo de la base de datos, pueden ver los mismos datos en 
		cualquier instante de tiempo.
	\item Cada petición recibe una respuesta acerca de si se tuvo éxito o 
		no.
	\item El sistema continúa funcionando a pesar de la falta de mensajes.
	\item Conectividad. Los datos están cada vez más entrelazados y 
		conectados. 
	\item Desestructuración cada vez mayor.
	\item Arquitectura. Empleo en aplicaciones orientadas a servicios. Se 
		usa una arquitectura distribuida.
\end{itemize}

Algunos ejemplos de manejadores de los distintos tipos de bases de datos son 
como se tienen en la siguiente tabla \ref{tabla:bd:manejadores:oop}:

%% Manejadores de BD existentes
\begin{table}[H]
	\centering 
	\caption{Ejemplos de distintos manejadores de base de datos no 
		relacionales y sus características 
		\cite{inv_storage_data, bd_nosql_tipos}.}
	\label{tabla:bd:manejadores:oop}
	\begin{tabular*}{0.9\textwidth}{@{\extracolsep{\fill}} |c|l|l|}
		\hline
		$\textbf{Tipo de base de datos}$ & $\textbf{Sistema Gestor}$ &
		$\textbf{Descripción}$
		\\\hline \hline
		\parbox[t]{4cm}{Orientadas a Clave-Valor} &
		\parbox[t]{3cm}{Dynamite, Voldemort, Tokyo} & 
		\parbox[t]{5.5cm}{
		\begin{itemize}
		\item Orientadas a clave-valor (o bien llamadas hash tables).
		\item Funcionalidad fácil de manejar.
		\item Cada clave está identificada con un único valor .
		\item No existen colisiones de datos. 
		\item La información está habitualmente guardada en formato 
			binario. 
		\item Son rápidas para realizar operaciones de lectura y 
			escritura.
		\end{itemize}
		} \\\hline
		\parbox[t]{4cm}{Orientadas a Documentos} &
		\parbox[t]{3cm}{CouchDB, MongoDB} &
		\parbox[t]{5.5cm}{
		\begin{itemize}
		\item Se almacenan en documentos con formatos ya sea en JSON, 
			XML.
		\item Búsquedas por clave-valor.
		\end{itemize}
		} \\\hline
		\parbox[t]{4cm}{Orientadas a Grafos} &
		\parbox[t]{3cm}{Neo4j, InfoGrid, Virtuoso} &
		\parbox[t]{5.5cm}{
		\begin{itemize}
		\item La información es representada como nodos en un grafo y 
			sus relaciones como aristas que unen los nodos. 
		\item Se usa la teoría de grafos para  realizar búsquedas 
			avanzadas.
		\end{itemize}
		} \\\hline
		\parbox[t]{4cm}{Orientadas a Objetos} &
		\parbox[t]{3cm}{Zope, Gemstone, Db4o, Oracle
		(objeto-relacional)} &
		\parbox[t]{5.5cm}{
		\begin{itemize}
		\item La información puede ser modelada a cómo se modela en una 
			aplicación de usuario.
		\item Incorporación de objetos personalizables por parte del 
			usuario.
		\item Código reutilizable.
		\end{itemize}
		} \\\hline
	\end{tabular*}
\end{table}

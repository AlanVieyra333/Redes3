%===============================================================================
% Capa de presentación.
% 

\subsection{Capa de presentación}

En el caso de un sitio web se referiría al diseño de la propia web, tanto su 
aspecto visual (colores, imágenes, tipografía empleada, posicionamiento de los 
distintos bloques de contenido dentro de las distintas páginas, etc.), como de 
la estructuración de los contenidos en diversas secciones y apartados enlazables 
a través de un menú con las distintas opciones disponibles 
\cite{webapp:def:arqui:web}. Aquí entrarían 
distintas disciplinas como las del diseño gráfico, la usabilidad, la experiencia 
de usuario (UX), la interacción usuario-máquina, los mapas del sitio o mapas 
web, etc., así como distintos términos como HTML5, CSS, DOM, JavaScript (JS), 
AJAX, estándares web, etc. Existen diferentes frameworks (cuya traducción 
aproximada sería “marco de trabajo”) que permiten desarrollar aplicaciones web 
de manera rápida y sencilla, por ejemplo: \\

%---------- Angular ----------
\label{sub:angular}
\subsubsection{Angular}
Está basado en el modelo vista controlador (MVC) construido con TypeScript con 
el objetivo de hacer JS más ágil. Está respaldada por Google y presenta una 
arquitectura basada en componentes. sus principales beneficios son 
\cite{webapp:five:javascript:frameworks}:
\begin{itemize}
	\item Producción rápida de código.
	\item Pruebas fáciles en cualquier parte de la aplicación (Pruebas 
		unitarias).
	\item Transferencia de datos bidireccional (interfaz-servidor).
\end{itemize}

%---------- React ----------
\label{sub:react}
\subsubsection{React}
Es una librería de JS para el desarrollo de interfaces de usuario. Está 
respaldada por Facebook y presenta una arquitectura basada en componentes. Se 
puede integrar fácilmente a cualquier arquitectura. Proporciona un gran aumento 
de rendimiento debido al uso de un DOM (por sus siglas en inglés Document Object 
Model) virtual \cite{webapp_react}.

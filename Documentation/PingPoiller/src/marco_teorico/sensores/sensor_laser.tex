%===============================================================================
% Sección: Sensores láser.
%

Este tipo de sensor detecta la posición del objeto. Esto se logra mediante el 
uso de un sistema de triangulación o uno de medición de tiempo.

\subsubsection{Sistema de Triangulación}
\label{sec:sensores:l:sistema_triangulacion}

El cambio en la distancia hasta el objeto afecta la posición de la luz 
concentrada en el elemento de detección CMOS. Esta información se utiliza para 
detectar la posición del objeto 
\cite{sensores:proximidad:clasificacion:principios}.

%%Figura 1.7: Sensor CMOS modelo reflectivo.
\begin{figure}[H]
	\begin{center}
		\includegraphics[width=.6\textwidth]
		{images/doc/img_sensor_cmos_modelo}
		\caption{Sensor CMOS modelo reflectivo.}
		\label{sensor:cmos}
	\end{center}
\end{figure}

El láser emite un rayo láser hacia el objeto como se muestra en la Figura 
\ref{sensor:cmos}
La luz reflejada por el objeto es concentrada por el lente del receptor, 
formando una imagen sobre el elemento receptor de luz. Cuando la distancia 
cambia, la luz concentrada se refleja en un ángulo diferente, y la posición de 
la imagen cambia correspondientemente.

\subsubsection{Sistema de Medición de tiempo}
\label{sec:sensores:l:medicion_tiempo}

La distancia se mide con base al tiempo que el rayo láser emitido tarda en 
retornar al sensor, tras incidir en el objeto. La detección no se ve afectada 
por el estado de la superficie del objeto.

%%Figura 1.8: Modelo de tiempo de vuelo
\begin{figure}[H]
	\begin{center}
		\includegraphics[width=.6\textwidth]
		{images/doc/img_modelo_tiempo_vuelo}
		\caption{Modelo de tiempo de vuelo.}
	\end{center}
\end{figure}

En la figura, el sensor detecta el tiempo (T) transcurrido hasta que se recibe 
el rayo láser reflejado, para calcular la distancia (Y). La fórmula de cálculo 
es: 2Y (distancia de ida y vuelta) = C (velocidad de la luz) × T (tiempo hasta 
que se recibe la luz reflejada).

\subsubsection{Sensores láser de tipo de reconocimiento de posición}
\label{sec:sensores:l:recog_posicion}

Algunos modelos pueden medir la distancia y la posición con una precisión mayor 
que un sensor simple.

\textbf{Modelo de barrera CCD} \\
Se emite un láser desde un transmisor, que luego es recibido por un elemento 
receptor de luz CCD. El área donde se interrumpe el láser se identifica 
claramente en el CCD. Este modelo se puede utilizar para detectar la posición en 
línea o medir el diámetro exterior de un objeto 
\cite{sensores:proximidad:clasificacion:principios}.

%%Figura 1.9: Serie IG
\begin{figure}[H]
	\begin{center}
		\includegraphics[width=.4\textwidth]{images/doc/img_serie_ig}
		\caption{Serie IG.}
	\end{center}
\end{figure}

\textbf{Modelo reflectivo CMOS de alta precisión} \\
La luz reflejada se recibe en el elemento receptor de luz CMOS, y la posición se 
determina con base en el principio de triangulación. Este modelo puede 
transmitir la información de altura con una señal analógica 
\cite{sensores:proximidad:clasificacion:principios}.

%%Figura 1.10: Serie IL/IA
\begin{figure}[H]
	\begin{center}
		\includegraphics[width=.4\textwidth]{images/doc/img_serie_il_ia}
		\caption{Serie IL/IA.}
	\end{center}
\end{figure}

. 
\\ \\

%===============================================================================
% Sección: Sensores ultrasónicos.
%

Los sensores ultrasónicos  cubren situaciones de aplicación de automatización 
industrial. Pueden detectar una gran variedad de materiales, no se ven afectados 
por superficies problemáticas y presentan una gran inmunidad frente a las 
influencias medioambientales. Sea cual sea su tarea, ya se trate de manipulación 
de materiales, equipos móviles, alimentación y bebidas, medición de nivel de 
llenado o detección en entradas y puertas, los sensores ultrasónicos aportan 
soluciones para los requisitos de aplicación más diversos 
\cite{sensores:proximidad:ultrasonicos:aplicaciones}. \\

El cabezal emite una onda ultrasónica y recibe la onda reflejada que retorna 
desde el objeto. Los sensores ultrasónicos miden la distancia al objeto contando 
el tiempo entre la emisión y la recepción. \\

%% imagen: Representación de Sensor Ultrasónico
\begin{figure}[H]
	\begin{center}
		\includegraphics[width=.5\textwidth]
		{images/doc/img_sensor_ultrasonido}
		\caption{Serie IG.}
	\end{center}
\end{figure}

Un sensor óptico tiene un transmisor y receptor, mientras que un sensor 
ultrasónico utiliza un elemento ultrasónico único, tanto para la emisión como la 
recepción. En un sensor ultrasónico de modelo reflectivo, un solo oscilador 
emite y recibe las ondas ultrasónicas, alternativamente. Esto permite la 
miniaturización del cabezal del sensor 
\cite{sensores:proximidad:ultrasonicos:significado}.


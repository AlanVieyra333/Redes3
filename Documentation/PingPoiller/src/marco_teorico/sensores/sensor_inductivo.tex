%===============================================================================
% Sección: Sensores inductivos.
%

Este tipo de sensores pueden detectar objetos metálicos que se acercan al 
sensor, sin tener contacto físico con los mismos. Los sensores de proximidad 
inductivos se clasifican de acuerdo con su principio de funcionamiento: el tipo 
de oscilación de alta frecuencia que utiliza la inducción electromagnética; el 
tipo magnético que emplea un imán; y el tipo de capacitancia que aprovecha los 
cambios en la capacidad eléctrica 
\cite{sensores:proximidad:inductivos:fundamentos}. \\

\subsubsection{Características}
\label{sec:sensores:capi:caracteristicas}

\begin{itemize}
	\item \textbf{Detección de metales}: \\
		Los sensores de proximidad inductivos sólo pueden detectar 
	objetos metálicos. No detectan objetos no metálicos, tales como 
	plástico, madera, papel y cerámica. A diferencia de los sensores 
	fotoeléctricos, esto permite que un sensor de proximidad inductivos 
	pueda detectar un objeto de metal a través de plástico opaco.

	\item \textbf{Resistencia ambiental}: \\
		Los sensores de proximidad inductivos son duraderos. Por 
	ejemplo, todos los modelos de cabezal KEYENCE satisfacen los requisitos 
	IP67 sellando el interior con material de relleno o mediante otras 
	medidas \cite{sensores:proximidad:inductivos:fundamentos}.

\end{itemize}

\textbf{Potección IP} \\
Grado de protección IP está regulado por la norma internacional IEC 60529.
El nivel de protección al polvo y al agua se mide en niveles, según la 
resistencia que tenga el dispositivo electrónico a la entrada de estas 
sustancias se le otorga un nivel, con ejemplos como IP56, IP67 o IP68, dónde el 
primer número hace referencia al polvo (cuanto mayor, mejor), y el segundo hace 
referencia al agua (cuanto mayor, mejor) 
\cite{sensores:proximidad:IP:tecnorium}. \\

\textbf{Niveles de protección al polvo} \\
El primer dígito hace referencia a resistencia de polvo y tiene seis niveles:

\begin{itemize}
	\item Nivel 0: no tiene ningún tipo de resistencia al polvo.
	\item Nivel 1: no debe entrar un objeto de forma esférica de 50 
		milímetros de diámetro.
	\item Nivel 2: el dispositivo electrónico está protegido ante la entrada 
		de partículas sólidas de al menos 12,5 mm.
	\item Nivel 3: protegido ante partículas (otra vez, esféricas) de 2,5mm.
	\item Nivel 4: resistencia ante objetos de 1mm de diámetro.
	\item Nivel 5: indica que el teléfono móvil (o cualquier otro 
		dispositivo electrónico) no se verá afectado por la entrada de 
		polvo, aunque permite cierta entrada del mismo.
	\item Nivel 6: es el nivel máximo y garantiza que el terminal es 100\% 
		estanco, por lo que no entrará polvo en ninguna circunstancia.
\end{itemize}

\textbf{Niveles de protección al agua} \\
El agua es un aspecto muy delicado. El agua es fuente de vida, pero resulta 
letal para cualquier dispositivo que se alimente de electrones. El estándar IP 
define 9 grados de protección ante el agua, que tiene en cuenta la presión, la 
cantidad de agua y la forma en la que incide un chorro.

\begin{itemize}
	\item Nivel 0: el terminal no cuenta con protección ante el agua.
	\item Nivel 1: el terminal debe resistir el goteo de agua lanzada desde 
		20 centímetros de agua durante 10 minutos.
	\item Nivel 2: igual que el nivel 1, pero los chorros de agua se 
		lanzarán rotando el chorro de agua 15º.
	\item Nivel 3: pasamos del goteo al agua nebulizada, o lanzada con un 
		spray. Debe resistir el agua nebulizada de 11 litros durante 5 
		minutos.
	\item Nivel 4: el terminal debe aguantar ser proyectado con un chorro de 
		agua. Chorro que tendrá una presión de entre 80-100 kN por metro 
		cuadrado y que durará 5 minutos.
	\item Nivel 5: igual que el nivel 4, pero debe soportar 12,5 litros por 
		minuto, a una presión de 30kN/m2 y con una boquilla de 6,3mm, 
		resistiendo la entrada de agua durante al menos 3 minutos.
	\item Nivel 6: igual que los dos niveles anteriores, pero con 100 litros 
		por minuto y una presión de 100kN/m2 y boquilla de 12,6mm.
	\item Nivel 7: este es el nivel de protección mínimo para que un móvil 
		sea sumergible. El terminal debe soportar la inmersión completa 
		en agua durante 30 minutos a un metro de profundidad sin que 
		llegue ni una sola gota de agua a sus circuitos internos.
	\item Nivel 8: se define como el soporte para una inmersión completa y 
		continua, superior al nivel 7. Junto a la especificación IP68, 
		el fabricante debe indicar la profundidad máxima, generalmente 
		de al menos 1,5 metros y el tiempo, del al menos media hora.
	\item Nivel 9: el terminal resistirá la entrada de agua en condiciones 
		extremas. En concreto, debe ser resistente a la inmersión en 
		agua a presión, de entre 80 y 100 bares y a una temperatura de 
		80ºC.
\end{itemize}

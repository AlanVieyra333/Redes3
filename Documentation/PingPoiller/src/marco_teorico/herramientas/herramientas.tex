%%==============================================================================
% Herramientas de manejo del drone
%

\section{Herramientas de software para control del drone}

Existen herramientas que permiten el control del drone, ya que nos ofrecen 
algunas interfaces de comunicación o APIs (Application Program Interface por sus 
siglas en inglés) que interactúan ya sea desde una aplicación de computadora o 
un programa remoto que se conectará al sistema del drone a través de diferentes 
medios (WIFI, conexión serial, etc.) permitiendo enviarle peticiones de control 
con la finalidad de que se logre: despegar, aterrizar, moverse a la derecha, 
moverse a la izquierda, moverse hacia abajo, moverse hacia arriba, rotar en 
sentido de las manecillas del reloj, rotar en sentido contrario a las manecillas 
del reloj; logrando así la manipulación del mismo. \\

%% Node JS
%
\subsection{Node JavaScript}

Node JavaScript o NodeJS es una poderosa plataforma de desarrollo de código 
basado en el lenguaje de programación JavaScript, el cual fue construido en la 
versión 8 del motor de Javascript del navegador Chrome de la empresa Google 
\cite{node_js_tutorial}. \\
Principalmente usado para el desarrollo de aplicaciones web ágiles en cuanto a 
operaciones de Entrada/Salida, por ejemplo: sitios de flujo de videos, 
aplicaciones simples, aplicaciones de geolocalización o uso de mapas, entre 
otras. \\
Node es un software de código libre, y es usado en miles de aplicaciones 
actuales en el mercado del internet. \\

\begin{figure}[hbtp!]
	\begin{center} 
		\includegraphics[width=.4\textwidth]{images/doc/img_node_logo}
		\caption{Logotipo de NodeJS.}
	\end{center}
\end{figure}

%% Usos 
El ecosistema de Node es NPM (Node Package Manager por su abreviación en inglés, 
o Administrador de paquetes de node), ya que es un conjunto de librerías de 
código abierto en el mundo \cite{npm_program}. \\

%%Paquetes-Aplicaciones
\textbf{Librerías disponibles online} \\
Existen diversos paquetes que nos pueden ayudar a manipular el drone, por 
ejemplo \cite{npm_program}:

\begin{itemize}
	\item node-bebop
	\item ar-drone
	\item ardrone-autonomy
	\item cyclonJS
	\item Johnny-five
	\item node-parrot-drone
\end{itemize}

Estos paquetes se pueden instalar usando la herramienta npm para descargarlos 
del repositorio original de node. \\
Cada uno tiene un conjunto de ejemplos, así como las carpetas respectivas donde 
se encuentra su código fuente, y son utilizados para el manejo de drones 
específicos dependiendo de su modelo como se muestra en la tabla 
\ref{paq:npm:program}.

%% Tabla de paquetes
\begin{table}[H]
	\centering 
	\caption{Principales paquetes en NPM para el control de drones 
		\cite{npm_program}.}
	\label{paq:npm:drone}
	\begin{tabular*}{0.9\textwidth}{@{\extracolsep{\fill}} |c|l|c|}
		\hline
		$\textbf{Nombre del paquete}$ & $\textbf{Descripción}$ &
		$\textbf{Última actualización}$
		\\\hline \hline
		\parbox[t]{4cm}{node-bebop} &
		\parbox[t]{3cm}{
		Biblioteca de JavaScript Node.js para Parrot Bebop que permite 
		enviar y recibir datos via WiFi.
		} & 
		\parbox[t]{5.5cm}{2016} \\\hline
		\parbox[t]{4cm}{ar-drone} &
		\parbox[t]{3cm}{Implementación de los protocolos de red 
		utilizados por Parrot AR Drone 2.0
		} &
		\parbox[t]{5.5cm}{2015} \\\hline
		\parbox[t]{4cm}{ardrone-autonomy} &
		\parbox[t]{3cm}{Constructor de bloques para vuelos autónomos en 
		el drone modelo AR} &
		\parbox[t]{5.5cm}{2014} \\\hline
		\parbox[t]{4cm}{cylon} &
		\parbox[t]{3cm}{Framework de programación de robótica para 
		arduino
		} &
		\parbox[t]{5.5cm}{2016} \\\hline
		\parbox[t]{4cm}{Johnny-five} &
		\parbox[t]{3cm}{Framework de programación de robótica para 
		arduino
		} &
		\parbox[t]{5.5cm}{2018} \\\hline
		\parbox[t]{4cm}{node-parrot-drone} &
		\parbox[t]{3cm}{Implementación del SDK de Parrot (fabricante de 
		drones).
		} &
		\parbox[t]{5.5cm}{2017} \\\hline
		\parbox[t]{4cm}{mavlink{\_}ardupilotmega{\_}v1.0} &
		\parbox[t]{3cm}{Implementation del protocolo MAVLink (por sus 
		siglas en inglés Micro Air Vehicle Link).
		} &
		\parbox[t]{5.5cm}{2013} \\\hline
	\end{tabular*}
\end{table}

%% SDK
%
\subsection{SDK}
%%Usos
El SDK, por sus siglas en inglés, Software Development Kit, es un kit de 
desarrollo que nos ofrece el fabricante, el cual nos permite introducir código y 
manipular el drone desde un nuevo programa fuente construido para los sistemas 
operativos Android, Unix o iOS. Dependerá de cada fabricante el proporcionar el 
SDK que permite manipular el drone, ya que, cada modelo tiene sus propias 
características y no todos los drones tienen en su arquitectura el soporte 
necesario para realizar la programación \cite{sdk_program_drone}. \\
La forma de controlar el drone es conectarse al drone, recibir flujos de datos, 
tales como videos, archivos en general, enviar y empezar un nuevo plan de vuelo, 
actualizar el estado del drone. \\

Existen aplicaciones hechas con un SDK como:
\begin{itemize}
	\item FreeFlight 2, 3, pro, etc. 
	\item Sphinx
	\item SkyController
	\item SkyController 2
\end{itemize}

%%Documentación
Éstas son específicas de la familia Parrot, particularmente para los drones de 
los modelos \cite{sdk_program_drone}:

\begin{itemize}
	\item Bebop 2
	\item Bebop
	\item Jumping Sumo
	\item Jumping Race
	\item MiniDrone Rolling Spider
	\item Airbone Cargo
	\item Airbone Night
	\item Swing 
	\item Mambo
\end{itemize}

\begin{table}[H]
	\centering 
	\caption{Características que ofrece el SDK para Parrot drone 
		 \cite{sdk_program_drone}.}
	\begin{tabular*}{0.9\textwidth}{@{\extracolsep{\fill}} |l|l|}
		\hline
		$\textbf{Característica}$ & $\textbf{Descripción}$
		\\\hline
		\parbox[t]{3cm}{Configurar un controlador de dispositivo} & 
		\parbox[t]{10.2cm}{Es un objeto principal que permitirá 
		instanciar un objeto de comunicación entre la aplicación y el 
		drone.
		} \\
		\hline
		\parbox[t]{3cm}{Comando de despegue} & \parbox[t]{10.2cm}{
		Permite enviar el comando de despegue, para realizar esta 
		acción, el usuario necesita asegurarse que el estatus de vuelo 
		debe estar en posición de aterrizaje. Después de haber 
		despegado, ahora se pueden enviar comandos de pilotaje del 
		drone.
		} \\
		\hline
		\parbox[t]{3cm}{Comando de Aterrizaje} & \parbox[t]{10.2cm}{
		Permite enviar un comando de aterrizaje, pero primero 
		necesitamos asegurarnos de que el sistema está en posición de 
		vuelo. Ayudan a realizar el control de drone realizando 
		operaciones como: rotar en el eje de las abscisas, rotar en el 
		eje de las Y, asenso y descenso (Eje de la Z). Estas operaciones 
		las permite hacer por medio de banderas, y agregando los rangos 
		que permite ser manipulado.}  
		\\\hline
		\parbox[t]{3cm}{Flujo de video} & \parbox[t]{10.2cm}{Esta 
		operacion se realiza en drones que tienen la posibilidad de 
		grabar videos con la cámara que tiene integrada, ya sea al 
		frente o abajo del drone. Sólo se necesita enviar una petición 
		al drone, para que se active la opción de grabación, para así 
		poder recibir un flujo de información una vez que se solicite el 
		detenimiento del video.}
		\\\hline
		\parbox[t]{3cm}{Tomar fotos} & \parbox[t]{10.2cm}{Al igual que 
		la operación anterior, se tiene que configurar una opción que 
		será enviada al drone, para que active la cámara en opción de 
		tomar fotografías. Además, se pueden descargar las fotos una vez 
		tomadas, ya que se guardan los archivos internos del drone.}
		\\\hline
	\end{tabular*}
\end{table}

%% Ardupilot
%
\subsection{Ardupilot}
%% Usos
Ardupilot el más avanzado, completamente documentado y software confiable de 
código abierto disponible. Tardó en desarrollarse durante 5 años, ya que se 
colaboró con investigadores científicos de la computación, ingenieros. \\
Instalado en más de 1,000,000 de vehículos a lo largo del mundo, con 
analizadores de datos, herramientas de simulación. La parte de código abierto 
está en rápida evolución. Finalmente, dado que el código fuente es abierto, se 
puede auditar para garantizar el cumplimiento de los requerimientos de seguridad 
y confidencialidad \cite{ardupilot_dev}. \\
Ardupilot también ofrece un kit de desarrollo ya específico para poder usar las 
tarjetas que proporciona de fábrica, como son: PX4 o autopilot, ardupilot, APM, 
además de ofrecer un control remoto en conjunto con todas las piezas necesarias 
para la construcción de un drone.

%% MAVLink
\subsection{MAVLink}
MavLink es un protocolo de comunicación para MAV (Micro Aerial Vehicles) para todo tipo de aviones no tripulados (tanto aéreos como terrestres). \\
El paquete MAVLink es básicamente una secuencia de bytes codificados y enviados a través de un transductor (serie USB, radio frecuencia, WiFi, GPRS, etc.). Mediante la codificación se ordena la información en un estructura de datos de manera inteligente añadiendo un checksums (suma de control), número de secuencia y se envía a través del canal byte a byte \cite{mavlink}. \\

Se realiza el envío de paquetes para poder decodificarlos y obtener información como estado del drone, batería, configuraciones, calibración, GPS, coordenadas y el estado actual de un vuelo.

%% Algoritmos para evitar obstáculos
\subsection{Algoritmos para evitar obstáculos}
Situaciones en las que un conjunto de drones se encuentran realizando una misión alrededor de la misma zona, en las que es necesario apoyar al sistema de control y guiado del dron para garantizar que no colisionen entre ellos ni con el entorno mientras realizan su misión. \\

A continuación se tratan algoritmos, computacionalmente sencillos, que puedan realizar las computadoras de UAVs de no muy alto coste, con los sensores que contengan a bordo, de manera autónoma y descentralizada, y garanticen una navegación sin colisiones \cite{the:algoritmos_evadir_obstaculos}.

\subsubsection{Bubble}
Se debe considerar que los drones constan de un anillo de sensores de distancia (infrasonidos, infrarrojos, láser...) que cubren un abanico de 180 delante del mismo. Como se muestra en la figura \ref{fig:algoritmo_burbuja} \cite{the:algoritmos_evadir_obstaculos}.\\

\fig[0.4]{img_algoritmo_burbuja}{algoritmo_burbuja}{Rango de cobertura de los sensores.}

Con las medidas de éstos, se define una burbuja de sensibilidad, de la siguiente forma:\\
\begin{center}
	$bubble boundary[i]=Ki*V*delta_t$
\end{center}

Siendo $V$ la velocidad del dron, $delta_t$ el intervalo de tiempo entre las sucesivas medidas del sensor, y Ki una constante escaladora. El algoritmo va continuamente midiendo y comprobando si hay obstáculos dentro de la burbuja, funcionando de la siguiente manera:\\

El dron se dirige directamente hacia su objetivo, hasta que encuentra uno o más obstáculos dentro de la burbuja
sensible. En ese momento, calcula un “ángulo de rebordeo” en la dirección en la que haya menor densidad de
obstáculos, y se dirige en dicha dirección hasta que deje atrás al obstáculo y el objetivo sea visible, dirigiéndose
hacia el objetivo, hasta que se encuentre otro obstáculo.


\subsubsection{Campo de fuerza virtual}
Cada obstáculo, ejerce una fuerza virtual repulsiva sobre el UAV, y el objetivo ejerce una fuerza virtual atractiva sobre el mismo, por lo que el UAV se moverá en la dirección resultante de cada una de estas fuerzas virtuales, de manera que tiende a esquivar a los obstáculos y acercarse al objetivo. Las fuerzas repulsivas y atractivas van multiplicadas por unas constantes que sirven para ajustar el comportamiento del robot \cite{the:algoritmos_evadir_obstaculos}.\\
La constante que multiplica al vector de repulsión, significará lo conservador o temerario que será nuestro UAV. Si la constante es muy grande, el vector de fuerzas virtuales de repulsión afectará más, y el robot será más conservador, por el contrario si esta constante adquiere un valor pequeño, nuestro robot se acercará más a los obstáculos.\\
La constante que multiplica al vector de fuerza virtual de atracción, indicará cuánto le afectan los obstáculos a nuestro UAV. Si tiene un valor muy grande, apenas se inmutará con los obstáculos.

\subsubsection{Optimal Reciprocal Collision Avoidance}
Se tiene en cuenta que cada UAV tiene una velocidad máxima, posteriormente se verá para qué es necesaria, así como una velocidad óptima o preferida, que la hemos definido como la que se dirige hacia el objetivo \cite{the:algoritmos_evadir_obstaculos}.\\
Se puede definir geométricamente la velocidad $ORCA_t$ $A\|B$ de la siguiente manera:\\
Se asume que tanto A como B se mueven a su velocidad óptima o preferida V$_A$$_o$$_p$$_t$ y V$_B$$_o$$_p$$_t$ ,supongamos que se encuentran camino de una colisión,es decir $V_A$-$V_B$ E VO.
Se define en este punto “u” como la distancia $A\|B$ mínima desde el punto de la velocidad relativa de los UAV ($V_A$ - $V_B$) hasta la frontera del cono de la Velocidad Obstáculo, y se define $"n"$ como el vector unitario que apunta en ésta dirección.\\
Matemáticamente:\\
\begin{center}
u = (argmin $_v$ $_E$ $_d$ $_V$$_O$$_t$ $_A$$\|$$_B$ $\|$$\|$ v – (V$_A$$_o$$_p$$_t$ - V$_B$$_o$$_p$$_t$ $\|$$\|$) - (V$_A$$_o$$_p$$_t$ - V$_B$$_o$$_p$$_t$).
\end{center}


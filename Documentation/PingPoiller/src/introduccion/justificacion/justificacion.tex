%===============================================================================
% Justificación del proyecto.
% Autor(es): Landa Aguirre Rafael
% Fecha: 08 - Mar - 2018
%

\section{Justificación}

De acuerdo con un artículo en  excélsior, publicado el día 23 de Septiembre del 2016, algunas empresas como Amazon, UPS (United Parcel Service), Zookal, DHL, Google, Waterstone, Facebook  ya están implementando el uso de drones para entregas a domicilio, sin embargo, eśtos requieren de una infraestructura costosa para su control. \\
El prototipo propuesto para el propósito del presente Trabajo Terminal es la manipulación de un vehículo aéreo no tripulado que logre trasladarse de un lugar a otro de manera autónoma, con ayuda de sensores y una controladora de vuelo capaz de tomar decisiones de vuelo que le permitirán evitar colisiones inesperadas.
También será monitoreado el vehículo a través de internet en una aplicación web que le estará informando al cliente sobre el estado del mismo cada cierto periodo de tiempo, independientemente de que tan lejos se encuentre el vehículo del cliente.

Asi mismo,los resultados del proyecto ayudarán a determinar la viabilidad de su uso en las áreas de seguridad, agricultura, paquetería a domicilio, transporte, entre otras. \\

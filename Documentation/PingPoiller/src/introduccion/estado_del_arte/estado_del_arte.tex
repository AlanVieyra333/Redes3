% Estado del arte.
% Autor(es): Landa Aguirre Rafael
%
% Fecha: 23 - Feb - 2018.
%

\section{Estado del Arte}

Usando herramientas como la internet se buscaron prototipos o sistemas r
ealizados por empresas comerciales ó universidades que han logrado dar a conocer 
al público los diseños de drones. Con esto se comprueba que hay, en otros 
países, diseños más avanzados debido a la gran inversión que se realiza, y 
además los diferentes dispositivos que adaptan al drone, tales como sensores, 
procesadores más rápidos, los cuáles hacen que la autonomía del drone sea más 
eficiente.

\begin{table}[H]
	\centering 
	\caption{Ejemplos de prototipos previamente diseñados en algún proyecto 
		para alguna organización.}
	\begin{tabular*}{0.9\textwidth}{@{\extracolsep{\fill}} |c|l|l|}
		\hline
		$\textbf{Nombre}$ & $\textbf{Tipo}$ &
		$\textbf{Características}$
		\\\hline \hline
		\parbox[t]{4cm}{Drone autónomo de Boeing} &
		\parbox[t]{3cm}{Vehículo desarrollado y presentado para la 
		empresa Boeing } & 
		\parbox[t]{5.5cm}{Prototipo desarrollado en octubre del año 
		2015. Se trata de un drone de ocho motores completamente 
		eléctrico y que utiliza baterías desarrolladas por las propia 
		Boeing y diseñado con la idea de construir plataformas de carga 
		autónoma de gran envergadura: con capacidad para cargar con 
		entre 120 y 230 kg, un alcance de entre 15 y 30 km, cambiar la 
		manera en el que se entregan mercancías.
		} \\\hline
		\parbox[t]{4cm}{Sistema de control que permite el vuelo autonomo 
		de drones} &
		\parbox[t]{3cm}{Sistema de  control desarrollado por la 
		universidad de Alicante.
		} &
		\parbox[t]{5.5cm}{Este sistema permite que el vehículo puede 
		alternar entre distintos planes de vuelo o definir los 
		desplazamientos óptimos en función del entorno y la información 
		recibida por sus sensores.  \\
		Se usa el protocolo MavLink para la comunicación entre el drone 
		y este sistema e incorpora sensores que son compatibles con USB, 
		I2C, ó RS232.  \\
		Existe una comunicación con el exterior a una base externa, la 
		cual permite enviar datos de vuelo, e información recogida por 
		los sensores. \\
		Además, incorpora en su diseño misiones peligrosas en las que el 
		tiempo de respuesta es bastante mayor ó entornos distantes en la 
		que la comunicación se puede interrumpir.
		} \\\hline
	\end{tabular*}
\end{table}

%% Referencias:
%% https://sgitt-otri.ua.es/es/empresa/documentos/ot-1502-drones.pdf
%% http://www.microsiervos.com/archivo/drones/dron-autonomo-boeing-capaz-transportar-230-kg.html
%% http://www.elmundo.es/economia/2017/05/04/590b6961e2704e34398b4618.html
.
\\ \\

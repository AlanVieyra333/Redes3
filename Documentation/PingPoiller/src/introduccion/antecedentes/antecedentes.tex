%===============================================================================
% Sección introductoria del reporte.
% Autor(es): 
%	Landa Aguirre Rafael
% Fecha: 08 - Mar - 2018
%

\section{Antecedentes}

%% Antecedentes históricos (generales)
%
\subsection{Desarrollo de drones en el Mundo}
Durante los últimos siglos, la tecnología se ha desarrollado a un ritmo 
trepidante. Desde el desarrollo de las primeras computadoras, las 
radio-computadoras, las redes de computadoras, han contribuido a la invención de 
los vehículos aéreos no tripulados (Drone por su nombre en inglés) 
\cite{vehiculos_aereos_lat}. \\
En el transcurso de la Primera Guerra Mundial, el progreso de la aviación 
convencional avanzó rápidamente, pero se veía frenada debido a la falta de 
desarrollo tecnológico. \\
Elmer Ambrose Sperry logró dar solución a diversos problemas, creando una nave 
no tripulada y brindó experiencias exitosas en cuanto al giroscopio para 
aplicaciones marítimas que llevó a crear un giro estabilizador para un avión en 
1909 \cite{origen_drones, elmer_sperry}. \\
La primera mitad del siglo XX, EE.UU. era líder en el mercado de vehículos 
aéreos no tripulados, ya que se financiaban desarrollos por parte del ejército y 
la marina pero, además, se fueron dando a conocer más avances tecnológicos ya 
que se incorporaron nuevos elementos al vehículo, estaban siendo equipados con 
cámaras de mayor resolución, para reconocimiento de territorio enemigo. \\
A finales del siglo XX, fue cuando los drones comenzaron a ser operados vía 
radio-control, obteniendo más características de su autonomía de vuelo. \\

%% Antecedentes en Latinoamérica (general)
%
\subsection{Desarrollo de drones en Latinoamérica}
En América Latina se comienza el desarrollo de drones hasta después del siglo 
XX, ya que anteriormente el campo estaba en pleno desarrollo por potencias 
militares, principalmente EE.UU., Gran Bretaña, Francia, Alemania, por 
cuestiones de defensa nacional. \\
En 2011, el Consejo de Defensa Sudamericano (UNASUR) acordó la creación de un 
grupo de trabajo que estudie el desarrollo y producción de vehículos aéreos no 
tripulados en tal región \cite{vehiculos_aereos_lat}. \\
La policía mexicana, ha sido dotada de nuevos vehículos aéreos no tripulados, ya 
que ha tenido un aumento significativo de operaciones tácticas, particularmente 
contra la lucha del crimen organizado, como los cárteles de droga 
\cite{vehiculos_aereos_lat}. \\
Varios desarrollos de vehículos como los presentados por la empresa mexicana 
Hydra Technologies para la policía mexicana, son mostrados en la siguiente 
tabla:

\begin{table}[H]
	\centering 
	\caption{Ejemplos de distintos manejadores de base de datos
		relacionales y sus características \cite{vehiculos_aereos_lat}.}
	\begin{tabular*}{0.93\textwidth}{@{\extracolsep{\fill}} |c|l|}
		\hline
		$\textbf{Nombre de Proyecto}$ & $\textbf{Características}$ \\
		\hline \hline
		\parbox[t]{3.1cm}{HYDRA TECHNOLOGIES S4-EHECATL} & 
		\parbox[t]{10.2cm}{Presentado en el 2007, este sistema táctico de 
		vigilancia aérea no tripulado tiene una autonomía de hasta 8 
		horas, con un radio de acción de hasta 100 kilómetros, a 
		altitudes de hasta 4500 metros y utilizando para sus misiones 
		cámaras de video de 25 megapíxeles y sistemas FLIR.
		} \\\hline
		\parbox[t]{3.1cm}{HYDRA TECHNOLOGIES E1-GAVILAN} & 
		\parbox[t]{10.2cm}{
		Diseñado para vigilancia y monitoreo urbano. Presentado en el 
		año 2008. En la actualidad dos de ellos se encuentran 
		operativos. Prototipo que tiene autonomía de 75 minutos, con 
		radio de acción de 10 kilómetros, maneja altitud de 400 metros, 
		hace uso de cámaras de video de alta resolución para poder 
		realizar sus misiones de vuelo.
		} \\\hline
		\parbox[t]{3.1cm}{HYDRA TECHNOLOGIES E2-COLIBRI} & 
		\parbox[t]{10.2cm}{
		El E2 tiene una autonomía de 30 minutos, con un radio de acción 
		de hasta 2 kilómetros, a altitudes de 200 metros y utilizando 
		para sus misiones cámaras de video de 25 megapíxeles y sistemas 
		FLIR.
		} \\\hline
		\parbox[t]{3.1cm}{ROTOMOTION SR-200} & 
		\parbox[t]{10.2cm}{
		Autocopter diseñado por la empresa estadounidense Romotion. 
		Posee las siguientes características: 2.7 metros de largo por 76 
		centímetros de ancho, peso de hasta 25 kilogramos, velocidades 
		de hasta 60 kilómetros por hora. Posee cámaras de 20 
		megapíxeles, día/noche y de alta resolución.
		}
		\\\hline
		\parbox[t]{3.1cm}{VANT T1-T1-T3} & 
		\parbox[t]{10.2cm}{El proyecto comenzó a finales del 2010, con 
		la participación de científicos del Instituto de Investigación y 
		Desarrollo de la Secretaría de Marina. Tendrán como misiones el 
		patrullaje del litoral marítimo de esta nación, así como la 
		vigilancia y protección de su infraestructura petrolera. VANT 
		táctico de 2.5 metros de ancho por 1.72 de largo, con un alcance 
		de 6 kilómetros y una autonomía de 80 minutos y un MINI-VANT, de 
		1.78 metros de ancho por 1.33 de largo, con una autonomía de solo 
		30 minutos y un alcance de 2 kilómetros.
		}
		\\\hline
	\end{tabular*}
\end{table}

Desde sus orígenes militares, los drones tienen la peculiaridad de no tener 
intervención humana en cuanto al pilotaje del vehículo, por lo que está 
controlado por un control externo o sistema electrónico externo que decide en 
cada momento cuál es el siguiente paso que seguir en la trayectoria del aparato. 
Para poder aportar maniobrabilidad en el vehículo, existen dispositivos llamados 
sensores que toman datos o con ayuda de un sistema de vuelo, ya que, para 
dirigirse hacia algún lugar, las decisiones necesitan ser tomadas en tiempo real 
\cite{vehiculos_aereos_no_trip_apps}.
La posibilidad de transmisión de imágenes o datos a una base de datos en tiempo 
real ha significado un gran avance que permite incorporar un mayor número de 
sensores que permite controlar agentes de vuelo, tales como: velocímetros, GPS, 
altímetros, giroscopios, entro otros \cite{vehiculos_aereos_no_trip_apps}.
Los componentes básicos de un vehículo de esta naturaleza son las siguientes 
\cite{vehiculos_aereos_no_trip_apps, app_veh_no_tripulados_hidr}:

\begin{itemize}
	\item \textbf{Control de autopiloto}: Sistema electromecánico usado para 
controlar el vuelo del drone en forma autónoma, como por ejemplo detección de 
altura mínima de vuelo para que no permita el choque contra el suelo.
	\item \textbf{Fuselaje}: Es el lugar donde se montan los dispositivos 
externos al drone para actividades de monitoreo, control, geoposicionamiento, 
soporte de cámaras, y equipos auxiliares.
	\item \textbf{Carga útil}: Incluye sensores pasivos, tales como cámaras 
espectrales, o sensores activos, como el sistema LIDIAR (Detección y rango de 
luz).
	\item \textbf{Subsistema de comunicación}: La mayoría de los drones 
cuentan con enlaces inalámbricos para comunicación con la estación terrestre y 
WiFi para compartir datos.
	\item \textbf{Estación de control terrestre}: Para monitoreo del control 
de vuelo en tiempo real, se usa un monitor de despliegue, el cual muestra los 
datos telemétricos, geoposicionamiento, estudio de la señal de GPS, tiempo de 
vuelo, nivel de carga de baterías.
	\item \textbf{Dispositivos de despegue y aterrizaje}: Los drones 
requieren de dispositivos especiales para el control de aterrizaje y despegue, 
como un lanzador hidráulico o redes para el aterrizaje.
\end{itemize}

Algunas de las aplicaciones comerciales que se les da a los drones actualmente 
son:

\begin{itemize}
	\item \textbf{Seguridad}: Vigilancia en zonas de acceso restringido, 
donde el drone tiene la tarea de monitorear la presencia de personal no 
autorizado \cite{morelos_cinco_for_security, drones_empresas_sec_vigi}.
	\item \textbf{Congestión vehicular}: Monitorización del 
congestionamiento vehicular, para elaborar reportes de tránsito y a su vez, 
transmitirlos en tiempo real a estaciones de control para determinar el tráfico 
posible en una determinada hora, y elaborar un plan de descongestionamiento 
vehicular \cite{african_drones_johanesburg_cong}.
	\item \textbf{Agricultura}: Monitorización de cultivos, para conocer el 
estado de los plantíos y si es que existe alguna plaga en los mismos, utilizando 
reconocimiento de patrones y análisis de imágenes, apoyándose en los diversos 
sensores implementados en el drone \cite{drones_for_agricultura}.
	\item \textbf{Logística}: Envíos menores, Amazon quiere enviar en 
EE.UU. los paquetes a la puerta de casa a través de drones 
\cite{amazon_prime_air}; en Alemania 
está el “Dönercopter”, un pequeño drone que despacha kebabs. En Australia, las 
compañías Zookal y Flirtey los usan para despachar libros, mientras que, en 
Londres, la cadena de restaurantes Yo Sushi prueba drones que llevan las 
bandejas desde la cocina directamente a las mesas de sus clientes fuera del 
local \cite{amazon_prime_air}. Sin embargo, esta tecnología carece de la 
confiabilidad en los clientes, ya que los sistemas actuales aún no garantizan la 
seguridad necesaria, como por ejemplo Amazon, que asegura que no pondrá en 
marcha sus drones repartidores “hasta que seamos capaces de demostrar seguridad 
de las operaciones”. Para evitar colisiones inesperadas, estos drones deberán 
estar equipados con sensores y de un software especial que sea capaz de 
aterrizar en zonas seguras \cite{func_amazon_prime_air}.
\end{itemize}

Con esto se puede observar la importancia del uso de drones dentro de diferentes actividades.
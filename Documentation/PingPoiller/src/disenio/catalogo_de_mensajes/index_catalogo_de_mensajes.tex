% Indice de catalogo de mensajes.
% Autor(es): Landa Aguirre Rafael
% 	     Alan Vieyra Fernando
%	     Olvera Pérez Víctor David
% Fecha: 26 - Abr - 2018
%

\section{Mensajes del sistema}
\cdtLabel{apendice:Mensajes}{}

%===========================  MSGX ==================================
%\begin{mensaje}{MSGX}{}{}
%	\item[Canal:] 
%	\item[Propósito:] 
%	\item[Redacción:]
%	\item[Parámetros:] 
%	\begin{itemize}
%		\item 
%	\end{itemize}
%	\item[Ejemplo:]  
%	\item[Referenciado por: ]
%\end{mensaje}

%===========================  MSG1 ==================================
\begin{mensaje}{MSG1}{Operación exitosa}{Informativo}
	\item[Canal:] Sistema
	\item[Propósito:] Notificar al actor que la operación solicitada al sistema se llevó a cabo exitosamente.
	\item[Redacción:] $<ARTICULO>$ + $<OPERACION>$ se llevo a cabo correctamente.
	\item[Parámetros:] \cdtEmpty
	\begin{itemize}
		\item ARTICULO: Es la parte de la oración que se ocupa de expresar el género (el/la).
		\item OPERACION: Actividad que el actor debe realizar.
	\end{itemize}
	\item[Ejemplo:] La solicitud de vuelo se llevo a cabo correctamente.
\end{mensaje}

%============================== MSG2 =================================
\begin{mensaje}{MSG2}{Operación fallida}{Error}
	\item[Canal:] Sistema
	\item[Propósito:] Notificar al actor que la operación solicitada al sistema no pudo llevarse a cabo.
	\item[Redacción:] Error al intentar $<OPERACION>$. Intentelo mas tarde.
	\item[Parámetros:] OPERACION: Operación que no se pudo llevar a cabo.
	\item[Ejemplo:] Error al intentar registrar al usuario en el sistema. Intentelo mas tarde.
\end{mensaje}

%===========================  MSG3 ==================================
\begin{mensaje}{MSG3}{Elementos no disponibles}{Error}
	\item[Canal:] Sistema
	\item[Propósito:] Notificar al actor que no existen elementos en el sistema por mostrar.
	\item[Redacción:] ¡No hay $<ELEMENTOS>$ para mostrarte!
	\item[Parámetros:] ELEMENTOS: Información de la cual se requiere para concluir un proceso
	\item[Ejemplo:] ¡No hay registros de vuelos para mostrarte!
\end{mensaje}

%============================== MSG4 =================================
\begin{mensaje}{MSG4}{Elemento no disponible}{Error}
	\item[Canal:] Sistema
	\item[Propósito:] Notificar al actor que el elemento que se desea ver no existe o no está disponible en el sistema.
	\item[Redacción:] ¡$<ELEMENTO>$ no disponible aún!
	\item[Parámetros:] ELEMENTO:Información de un registro en específico de la cual se requiere para concluir un proceso.
	\item[Ejemplo:] ¡Drone no disponible aún!
\end{mensaje}

%============================== MSG5 =================================
\begin{mensaje}{MSG5}{Duplicado de información}{Error}
	\item[Canal:] Sistema
	\item[Propósito:] Notificar al actor que ya existe un elemento con la misma información que desea ingresar.
	\item[Redacción:] ¡Ya existe $<DETERMINANTE>$ + $<ELEMENTO>$ con el mismo $<CAMPO>$!
	\item[Parámetros:] \cdtEmpty
	\begin{itemize}
		\item DETERMINANTE: Es la parte de la oración que se ocupa de expresar el género (un/una).
		\item ELEMENTO: Elemento que esta duplicado.
		\item CAMPO: Campo que está siendo duplicado.
	\end{itemize}
	\item[Ejemplo:]¡Ya existe un usuario con el misma nombre!
\end{mensaje}

%============================== MSG6 =================================
\begin{mensaje}{MSG6}{Falta dato obligatorio}{Error}
	\item[Canal:] Sistema
	\item[Propósito:] Notificar al actor la omisión de algún dato obligatorio por ingresar.
	\item[Redacción:] Campo obligatorio.
\end{mensaje}

%============================== MSG7 =================================
\begin{mensaje}{MSG7}{Formato de nombre incorrecto}{Error}
	\item[Canal:] Sistema
	\item[Propósito:] Notificar al actor que el nombre ingresado no cumple con el tipo de dato definido en el diccionario de datos.
	\item[Redacción:] Nombre inválido, verifique que el nombre ingresado por el usuario contenga caracteres del alfabeto incluyendo acentos y la letra "ñ$"$  .
\end{mensaje}

%============================== MSG8 =================================
\begin{mensaje}{MSG8}{Formato de nombre de usuario incorrecto}{Error}
	\item[Canal:] Sistema
	\item[Propósito:] Notificar al actor que el nombre de usuario ingresado no cumple con el tipo de dato definido en el diccionario de datos.
	\item[Redacción:] Nombre de usuario inválido, verifique que 
	contenga entre 8 y 20 caracteres alfanuméricos.
\end{mensaje}

%============================== MSG9 =================================
\begin{mensaje}{MSG9}{Formato de correo incorrecto}{Error}
	\item[Canal:] Sistema
	\item[Propósito:] Notificar al actor que el correo ingresado no cumple con el tipo de dato definido en el diccionario de datos.
	\item[Redacción:] Correo inválido, verifique que el correo debe tener el caracter "@$"$ y ".$"$, además de contar con caracteres alfanuméricos.
\end{mensaje}

%============================== MSG10 =================================
\begin{mensaje}{MSG10}{Formato de contraseña incorrecto}{Error}
\item[Canal:] Sistema
\item[Propósito:] Notificar al actor que la contraseña ingresado no cumple con el tipo de dato definido en el diccionario de datos.
\item[Redacción:] Contraseña inválido, verifique que contenga entre 8 y 50 caracteres alfanuméricos..
\end{mensaje}

%============================== MSG11 =================================
\begin{mensaje}{MSG11}{Datos de usuario inválidos}{Error}
	\item[Canal:] Sistema
	\item[Propósito:] Notificar al actor que los datos de inicio de sesión que ha introducido no corresponden a un usuario registrado en el sistema.
	\item[Redacción:] Credenciales inválidas.
\end{mensaje}

%===========================  MSG12 ==================================
\begin{mensaje}{MSG12}{Eliminar elemento}{Confirmación}
	\item[Canal:] Sistema
	\item[Propósito:] Solicitar la confirmación del actor para la eliminación de un elemento.
	\item[Redacción:] ¿Desea eliminar $<ARTICULO>$ + $<ELEMENTO>$ + $<SELECCION>$?
	\item[Parámetros:] \cdtEmpty
	\begin{itemize}
		\item ARTICULO: Es la parte de la oración que se ocupa de expresar el género (masculino/femenino).
		\item ELEMENTO: Es el elemento que se requiere eliminar.
		\item SELECCION:Dependiendo del género en la oración se ocupa, seleccionado(masculino)  ó seleccionada(femenino).
	\end{itemize}
	\item[Ejemplo:] ¿Desea eliminar el historial seleccionado?
\end{mensaje}

%===========================  MSG13 ==================================
\begin{mensaje}{MSG13}{Envío de correo existoso}{Informativo}
	\item[Canal:] Sistema
	\item[Propósito:] Notificar al actor que se le ha enviado un correo electrónico.
	\item[Redacción:] Se ha enviado un correo electrónico a: $<CORREO>$ + $<ASUNTO>$.
	\item[Parámetros:] \cdtEmpty
	\begin{itemize}
		\item CORREO: Correo electrónico al cual se le envió el mensaje.
		\item AUSNTO: Breve explicación del mensaje enviado.
	\end{itemize}
	\item[Ejemplo:] Se ha enviado un correo electrónico a: usuario@email.com con las instrucciones para restaurar tu contraseña.
\end{mensaje}

%===========================  MSG14 ==================================
\begin{mensaje}{MSG14}{Correo: Restauración de contraseña}{Notificación}
	\item[Canal:] Correo.
	\item[Propósito:] Permitir al usuario restaurar su contraseña.
	\item[Redacción:] \cdtEmpty
	Hola $<NOMBRE>$!
	
	Ha solicitado el cambio de su contraseña. Para restaurarla dirijase al siguiente enlace en un plazo no mayor a 48 horas.
	$<URL>$
	
	Por su atención, gracias.
	\item[Parámetros:] \cdtEmpty 	
	\begin{itemize}
		\item $<NOMBRE$: Nombre del usuario.
		\item $<URL>$: Enlace generado automáticamente con expiración de un día para permitir al usuario restablecer su contraseña.
	\end{itemize}
	\item[Ejemplo:] \cdtEmpty
	
	Hola Juan Carlos Pérez Acosta!
	
	Ha solicitado el cambio de su contraseña. Para restaurarla dirijase al siguiente enlace en un plazo no mayor a 48 horas.
	
	www.mcflight.com/restauracion?id=kjrDAIf834FDSv213ZzP
	
	Por su atención, gracias.
	
\end{mensaje}

%============================== MSG15 =================================
\begin{mensaje}{MSG15}{Confirmación de contraseña incorrecta}{Error}
	\item[Canal:] Sistema
	\item[Propósito:] Notificar al actor que la confirmación de contraseña que ha introducido no coincide con la contraseña introducida.
	\item[Redacción:] La confirmación de su contraseña no coincide con la contraseña introducida.
\end{mensaje}
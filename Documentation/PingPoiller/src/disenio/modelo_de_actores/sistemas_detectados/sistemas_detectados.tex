%===============================================================================
% Sección: Sistemas detectados.
%

La figura \ref{img:dis:sistemas:detectados} muestra los sistemas involucrados en 
el proyecto.

\begin{figure}[H]
	\begin{center}
		\includegraphics[width=.5\textwidth]
		{images/doc/img_sistemas_detectados}
		\caption{Sistemas involucrados.}
		\label{img:dis:sistemas:detectados}
	\end{center}
\end{figure}

%========== Responsable del Drone ======%
\begin{actor}{Drone}{Drone}{Sistema que se comunicará con la aplicación web para intercambiar información referente al vuelo.}
	\item[Responsabilidades:] \cdtEmpty
	\begin{itemize}
		\item Enviar posición geográfica actual.
		\item Enviar su altura respecto a su posición inicial.
		\item Enviar su nivel de batería.
		\item Enviar su velocidad.
		\item Enviar el estado del drone (en vuelo, aterrizado).
		\item Recibir la posición geográfica de destino.
	\end{itemize}
	\item[Perfil:] Sistema autónomo.
	\item{Cantidad:} Para fines de este prototipo se requiere el uso de un solo drone.
	\item{Estatus:} En revisión.
\end{actor}

%========== Responsable del Servidor ======%
\begin{actor}{Servidor}{Servidor}{El servidor es un actor en el que tendrá comunicación con el drone, con el fin de estar al pendiente de la información que recibe por parte del drone, y además para enviar datos de GPS por parte del cliente.}
	\item[Responsabilidades:] \cdtEmpty
	\begin{itemize}
		\item Enviar la posición geográfica de destino.
		\item Recibir datos generales del drone.
		\item Mostrar información relevante del vuelo.
	\end{itemize}
	\item[Perfil:] Desarrollado en python usando diversos frameworks tales como DJango y Angular.
	\item{Cantidad:} Para la interacción con varios clientes y un drone, se requiere el uso de un solo servidor.
	\item{Estatus:} En revisión.
\end{actor}


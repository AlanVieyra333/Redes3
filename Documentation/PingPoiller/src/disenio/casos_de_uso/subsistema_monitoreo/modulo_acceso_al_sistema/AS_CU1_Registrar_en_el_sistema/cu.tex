% Indice de Caso de uso: Registrar en el sistema.
% Autor(es): Landa Aguirre Rafael
% 	     Alan Vieyra Fernando
%	     Olvera Pérez Víctor David
% Fecha: 04 - May - 2018
%

\begin{UseCase}{AS-CU1}{Registrar en el sistema}{
		Permite al \refElem{Cliente} registrarse en el sistema para tener acceso a éste, permitiéndole solicitar al drone que realice un vuelo.
	}
	\UCccitem{Versión}{0.1}
	\UCccsection{Datos para el control Interno}
	\UCccitem{Elaboró}{Alan Fernando Rincón Vieyra}
	\UCccitem{Revisó}{\TODO}
	\UCccitem{Estatus}{Por revisar}
	\UCccitem{Fecha del último estatus}{07 de Mayo de 2018}
	\UCccitem{Observaciones}{Ninguna}
	\UCsection{Atributos}

	\UCitem{Actor}{\refElem{Cliente}}

	\UCitem{Propósito}{Registrar usuarios en el sistema.}

	\UCitem{Entradas}{%
		\begin{Titemize}
			\Titem \refElem{Cliente.nombre}.
			\Titem \refElem{Usuario.nombreUsuario}.
			\Titem \refElem{Cliente.correoElectronico}.
			\Titem \refElem{Usuario.contrasenia}.
		\end{Titemize}
	}

	\UCitem{Origen}{%
		\begin{Titemize}
			\Titem \ioEscribir
		\end{Titemize}
	}

	\UCitem{Salidas}{%
		\begin{Titemize}
			\Titem \refIdElem{MSG1}
			\Titem \refIdElem{MSG5}
			\Titem \refIdElem{MSG6}
			\Titem \refIdElem{MSG7}
			\Titem \refIdElem{MSG8}
			\Titem \refIdElem{MSG9}
			\Titem \refIdElem{MSG10}
			\Titem \refIdElem{MSG15}
		\end{Titemize}	
	}

	\UCitem{Destino}{Pantalla}

	\UCitem{Precondiciones}{%
		Ninguna.
	}

	\UCitem{Postcondiciones}{%
		El usuario queda registrado en el sistema.
	}

	\UCitem{Reglas de Negocio}{%
		\begin{Titemize}
			\Titem \refIdElem{RN-MON-N01}
			\Titem \refIdElem{RN-MON-N02}
			\Titem \refIdElem{RN-MON-N03}
			\Titem \refIdElem{RN-MON-N04}
			\Titem \refIdElem{RN-MON-N05}
			%\Titem \refIdElem{RN-MON-N06}
			%\Titem \refIdElem{RN-MON-N07}
			%\Titem \refIdElem{RN-MON-N08}
			%\Titem \refIdElem{RN-MON-N09}
			%\Titem \refIdElem{RN-MON-N10}

			\Titem \refIdElem{RN-MON-N11}
			\Titem \refIdElem{RN-MON-N12}
			\Titem \refIdElem{RN-MON-N13}
		\end{Titemize}
	}

	\UCitem{Errores}{%
		\begin{Titemize}
			\Titem \UCerr{Uno}{Cuando el usuario no introduce todos los campos marcados con asterisco,}{el sistema muestra el mensaje \refIdElem{MSG6} debajo de cada campo faltante y regresa al paso \ref{AS-CU1:introducirDatos} de la trayectoria principal.}
			
			\Titem \UCerr{Dos}{Cuando se introduce un nombre, nombre de usuario, correo o contraseña inválida,}{el sistema muestra el mensaje \refIdElem{MSG7}, \refIdElem{MSG8}, \refIdElem{MSG9} o \refIdElem{MSG10} respectivamente y regresa al paso \ref{AS-CU1:introducirDatos} de la trayectoria principal.}
		 
			\Titem \UCerr{Tres}{Cuando se introduce un nombre de usuario o correo que ya fue registrado previamente,}{el sistema muestra el mensaje \refIdElem{MSG5} con los parametros:
			\begin{itemize}
				\item $DETERMINANTE$: 'un'.
				\item $ELEMENTO$: 'usuario'.
				\item $CAMPO$: 'nombre de usuario' / 'correo'.
			\end{itemize}
			y regresa al paso \ref{AS-CU1:introducirDatos} de la trayectoria principal.}
			
			\Titem \UCerr{Cuatro}{Cuando se introduce una contraseña que no coincide con la confirmación de la misma,}{el sistema muestra el mensaje \refIdElem{MSG15} y regresa al paso \ref{AS-CU1:introducirDatos} de la trayectoria principal.}
			
			\Titem \UCerr{Cinco}{Cuando no es posible registrar al usuario,}{el sistema muestra el mensaje \refIdElem{MSG2} con el parámetro:
			\begin{itemize}
				\item $OPERACION$: 'registrar al usuario en el sistema'.
			\end{itemize}
			y regresa al paso \ref{AS-CU1:introducirDatos} de la trayectoria principal.}
		\end{Titemize}				
	}

	\UCitem{Viene de}{Ninguno.}

	\UCitem{Disparadores}{Requiere hacer uso del sistema.} 

	\UCitem{Condiciones de Término}{%
		Se muestra el mensaje \refIdElem{MSG1} indicando que el usuario ha quedado registrado en el sistema.}

	\UCitem{Efectos Colaterales}{El usuario registrado puede acceder al sistema.}
	
	\UCitem{Datos sensibles}{
		\begin{Titemize}
			\Titem \refElem{Cliente.nombre}.
			\Titem \refElem{Cliente.correoElectronico}.
		\end{Titemize}
	}
\end{UseCase}


%Trayectoria Principal : Happy Path

\begin{UCtrayectoria}

	\UCpaso [\UCactor] Presiona el botón \IUbutton{Registrarse} en la pantalla \refIdElem{AS-IU1}.

	\UCpaso Muestra la pantalla \refIdElem{AS-IU2}.

	\UCpaso [\UCactor] \label{AS-CU1:introducirDatos}Introduce su nombre, nombre de usuario, correo electrónico, contraseña y la confirmación de su contraseña.
	
	\UCpaso [\UCactor] Presiona el botón \IUbutton{Registrar}. \refTray{A}
	
	\UCpaso Verifica que no falte información en los campos marcados con asterisco, con base a la regla de negocio \refIdElem{RN-MON-N01}. \refErr{Uno}

	\UCpaso Verifica que los datos introducidos sean válidos, con base a las reglas de negocio \refIdElem{RN-MON-N02}, \refIdElem{RN-MON-N03}, \refIdElem{RN-MON-N04} y \refIdElem{RN-MON-N05}. \refErr{Dos}

	\UCpaso Verifica que el nombre de usuario o el correo no estén registrados en el sistema, con base en la regla de negocio \refIdElem{RN-MON-N11}. \refErr{Tres}
	
	\UCpaso Verifica que la contraseña y la confirmación de contraseña sean identicas, con base en la regla de negocio \refIdElem{RN-MON-N13}. \refErr{Cuatro}

	\UCpaso Registra los datos del usuraio en el sistema. \refErr{Cinco}

	\UCpaso Muestra el mensaje \refIdElem{MSG1} en la pantalla \refIdElem{GV-IU1} con los siguientes parámetros:
	\begin{itemize}
		\item $ARTICULO$: 'El'.
		\item $OPERACION$: 'registro'.
	\end{itemize}
\end{UCtrayectoria}


%-------------------------- Trayectoria Alternativa A ---------------------------------

\begin{UCtrayectoriaA}[Termina caso de uso]{A}{Se requiere cancelar la operación}

	\UCpaso [\UCactor] 	Da clic sobre la imagen principal del menú.

	\UCpaso Muestra la pantalla \refIdElem{AS-IU1}.
\end{UCtrayectoriaA}

% Caso de uso: Restablecer contraseña.
% Autor(es): Landa Aguirre Rafael
% 	     Alan Vieyra Fernando
%	     Olvera Pérez Víctor David
% Fecha: 04 - May - 2018
%

\begin{UseCase}{AS-CU2.1}{Restablecer contraseña}{
		Permite al \refElem{Cliente} realizar un proceso de recuperación 
		de contraseña cuanto éste lo requiera o en una situación de 
		olvido.
	}
	\UCccitem{Versión}{0.1}
	\UCccsection{Datos para el control Interno}	
	\UCccitem{Elaboró}{
		\begin{Titemize}
			\Titem Alan Fernando Rincón Vieyra.
			\Titem Landa Aguirre Rafael.
		\end{Titemize}
	}
	\UCccitem{Revisó}{}
	\UCccitem{Estatus}{\TODO{Por modificar.}}
	\UCccitem{Fecha del último estatus}{11 de Mayo de 2018}
	\UCccitem{Observaciones}{Ninguna.}
	\UCsection{Atributos}
	% V.0.1.
	\UCitem{Actor}{\refElem{Cliente}}
	% V.0.1.
	\UCitem{Propósito}{
		Recuperar el acceso al sistema o solicitar un cambio de 
		contraseña.
	}
	% V.0.1.
	\UCitem{Entradas}{%
		\begin{Titemize}
			\Titem \refElem{Cliente.correoElectronico}.
			\Titem \refElem{Usuario.contrasenia}.
		\end{Titemize}
	}
	% V.0.1.
	\UCitem{Origen}{
		\begin{Titemize}
			\Titem \ioEscribir
		\end{Titemize}
	}
	% V.0.1.
	\UCitem{Salidas}{%
		\begin{Titemize}
			\Titem \refIdElem{MSG1}.
			\Titem \refIdElem{MSG6}.
			\Titem \refIdElem{MSG9}.
			\Titem \refIdElem{MSG10}.
		\end{Titemize}	
	}
	% V.0.1.
	\UCitem{Destino}{Pantalla}
	% V.0.1.
	\UCitem{Precondiciones}{Haber estado previamente registrado.}
	% V.0.1.
	\UCitem{Postcondiciones}{%
		\begin{Titemize}
			\Titem Mostrar mensaje de confirmación \refIdElem{MSG1}.
			\Titem Haber enviado correo electrónico de recuperación.
		\end{Titemize}
	}
	% V.0.1.
	\UCitem{Reglas de Negocio}{%
		\begin{Titemize}
			\Titem \refIdElem{RN-MON-N04}.
			\Titem \refIdElem{RN-MON-N05}.
			%% Agregar RN de confimar contraseña y contraseña
		\end{Titemize}
	}
	% V.0.1.
	\UCitem{Errores}{%
		\begin{Titemize}
			\Titem \UCerr{Uno}{Cuando el usuario no introduce todos 
			los campos marcados con asterisco,}{el sistema muestra 
			el mensaje \refIdElem{MSG6} y regresa al paso 
			\ref{AS-CU21:introducirCorreoElectronico} de la 
			trayectoria principal.}
			
			\Titem \UCerr{Dos}{Cuando el campo correo electrónico no 
			cumple con el formato de dato solicitado,}{se muestra el 
			mensaje \refIdElem{MSG9} y regresa al paso 
			\ref{AS-CU21:introducirCorreoElectronico} de la 
			trayectoria principal.}
			
			\Titem \UCerr{Tres}{Cuando la nueva contraseña no 
			cumple con el formato de dato solicitado,}{se muestra el 
			mensaje \refIdElem{MSG10} y regresa al paso 
			\ref{AS-CU21:introducirContrasenia} de la trayectoria 
			principal.}
			
			\Titem \UCerr{Cuatro}{Cuando el campo confirmar nueva
			contraseña no cumple con el formato de dato solicitado,}
			{se muestra el mensaje \refIdElem{MSG10} y regresa al 
			paso \ref{AS-CU21:introducirConfirmarContrasenia} de 
			la trayectoria 
			principal.}
			\Titem \UCerr{Cinco}{Cuando los campos nueva contraseña 
			y confirmar nueva contraseña no coinciden }{se muestra 
			el mensaje \refIdElem{MSG15} y regresa al paso 
			\ref{AS-CU21:introducirContrasenia} de la trayectoria 
			principal.}
		\end{Titemize}				
	}
	% V.0.1.
	\UCitem{Viene de}{\refIdElem{AS-CU1}}
	% V.0.1.
	\UCitem{Disparadores}{
		Requiere solicitar un cambio de datos de acceso	para ingresar el 
		sistema.
	} 
	% V.0.1.
	\UCitem{Condiciones de Término}{%
		\begin{Titemize}
			\Titem Se muestra el mensaje \refIdElem{MSG1} indicando 
			que se han actualizado sus datos de acceso al sistema.
			\Titem El actor \refElem{Cliente} verifica el correo 
			electrónico recibido.
		\end{Titemize}
	}
	% V.0.1.
	\UCitem{Efectos Colaterales}{
		\begin{Titemize}
			\Titem El usuario registrado puede acceder al sistema.
			\Titem Actualización de la información de registro.
		\end{Titemize}
	}
	\UCitem{Datos sensibles}{
		\begin{Titemize}
			\Titem \refElem{Cliente.correoElectronico}.
			\Titem \refElem{Usuario.contrasenia}.
		\end{Titemize}
	}
\end{UseCase}

%Trayectoria Principal : Happy Path

\begin{UCtrayectoria}
	% V.0.1.
	\UCpaso [\UCactor] Presiona el enlance \underline{\textit{“¿Ha olvidado 
	su contraseña?”}} de la pantalla \refIdElem{AS-IU3}.
	% V.0.1.
	\UCpaso Muestra la pantalla \refIdElem{AS-IU4}.
	% V.0.1.
	\UCpaso [\UCactor] \label{AS-CU21:introducirCorreoElectronico} Introducir 
	información del campo: correo electrónico.
	% V.0.1.
	\UCpaso [\UCactor] \label{AS-CU21:introducirContrasenia} Introducir 
	información del campo: nueva contraseña.
	% V.0.1.
	\UCpaso [\UCactor] \label{AS-CU21:introducirConfirmarContrasenia} 
	Introducir información del campo: confirmar nueva contraseña.
	% V.0.1.
	\UCpaso [\UCactor] Presiona el botón \IUbutton{Enviar solicitud}.
	% V.0.1.
	\UCpaso Verifica que no existan campos obligatorios vacíos. \refErr{Uno}
	% V.0.1.
	\UCpaso Verifica si los campos del usuario cumplen con los formatos 
	requeridos de acuerdo a las reglas de negocio \refIdElem{RN-MON-N04}, 
	\refIdElem{RN-MON-N05}. \refErr{Dos} \refErr{Tres} \refErr{Cuatro}
	% V.0.1.
	\UCpaso Recibir campos: correo electrónico, nueva contraseña.
	% V.0.1.
	\UCpaso Verifica si los campos: nueva contraseña y confirmar nueva 
	contraseña coinciden. \refErr{Cinco} %% Agregar RN respectiva
	% V.0.1.
	\UCpaso Verifica si el campo correo electronico está previamente 
	registrado en el sistema. \refTray{A}
	% V.0.1.
	\UCpaso Actualizar el campo contraseña en base de datos.
	% V.0.1.
	\UCpaso Enviar correo electrónico con la información de contraseña 
	actualizada. \refTray{B}
	% V.0.1.
	\UCpaso Muestra el mensaje: \refIdElem{MSG1} en la pantalla 
	\refIdElem{AS-IU4}, con los siguientes parámetros:
	
	\begin{itemize}
		\item $ARTICULO$: 'La'.
		\item $OPERACION$: 'actualizar'.
	\end{itemize}
	% V.0.1.
	\UCpaso Muestra el mensaje: \refIdElem{MSG1} en la pantalla 
	\refIdElem{AS-IU4}, con los siguientes parámetros:
	
	\begin{itemize}
		\item $ARTICULO$: 'La'.
		\item $OPERACION$: 'enviar correo'.
	\end{itemize}
	
	%Un correo electrónico de 
	%recuperación se ha enviado a: $<<correo-del-usuario>>$ exitosamente. 
	%Favor de verificarlo.
	
	% V.0.1.
	\UCpaso [\UCactor] Abre una nueva pantalla en el navegador.
	% V.0.1.
	\UCpaso [\UCactor] Accede a su correo electrónico.
	% V.0.1.
	\UCpaso [\UCactor] Verifica el correo electrónico recibido.
\end{UCtrayectoria}


%-------------------------- Trayectoria Alternativa A ---------------------------------

\begin{UCtrayectoriaA}{A}{Correo electrónico no registrado}
	% V.0.1.
	\UCpaso Muestra el mensaje: \refIdElem{MSG2} en pantalla 
	\refIdElem{AS-IU4}, con el siguiente parámetro:
	\begin{itemize}
		\item $OPERACION$: 'correo electrónico inexistente'.
	\end{itemize}
	
	%Su correo electrónico no ha 
	%sido registrado en el sistema. Favor de verificarlo.
	
	% V.0.1.
	\UCpaso Regresar al paso \ref{AS-CU21:introducirCorreoElectronico} de 
	la trayectoria principal.
\end{UCtrayectoriaA}

%-------------------------- Trayectoria Alternativa B ---------------------------------

\begin{UCtrayectoriaA}[Termina caso de uso]{B}{Error al enviar correo electrónico}
	% V.0.1.
	\UCpaso Muestra el mensaje de error: \refIdElem{MSG2} en pantalla 
	\refIdElem{AS-IU4}, con el siguiente parámetro:
	\begin{itemize}
		\item $OPERACION$: 'enviar correo electrónico'.
	\end{itemize}
	
	%Su correo 
	%electrónico no ha sido enviado. Favor de intentarlo más tarde.
\end{UCtrayectoriaA}


% Caso de uso: Editar datos personales.
% Autor(es): Landa Aguirre Rafael
% 	     Alan Vieyra Fernando
%	     Olvera Pérez Víctor David
% Fecha: 12 - May - 2018
%

\begin{UseCase}{AS-CU3.2}{Editar perfil}{
		Permite al \refElem{Cliente} actualizar o modificar en caso de 
		que lo requiera sus datos de cuenta de usuario, en el que podrá
		modificar su correo electrónico, nombre de usuario, contraseña.
	}
	\UCccitem{Versión}{0.1}
	\UCccsection{Datos para el control Interno}
	\UCccitem{Elaboró}{
		\begin{Titemize}
			\Titem Alan Fernando Rincón Vieyra.
			\Titem Rafael Landa Aguirre.
		\end{Titemize}
	}
	\UCccitem{Revisó}{----}
	\UCccitem{Estatus}{Revisado.}
	\UCccitem{Fecha del último estatus}{11 de Mayo de 2018}
	\UCccitem{Observaciones}{Ninguna.}
	\UCsection{Atributos}
	% V.0.1.
	\UCitem{Actor}{\refElem{Cliente}}
	% V.0.1.
	\UCitem{Propósito}{
		Modificar sus datos de perfil, con el fin de que el usuario 
		actualice su información en el momento que requiera.
	}
	% V.0.1.
	\UCitem{Entradas}{%
		\begin{Titemize}
			\Titem \refElem{Usuario.nombreUsuario}.
			\Titem \refElem{Cliente.correoElectronico}.
			\Titem \refElem{Usuario.contrasenia}.	
		\end{Titemize}
	}
	% V.0.1.
	\UCitem{Origen}{
		\begin{Titemize}
			\Titem \ioEscribir
		\end{Titemize}
	}
	% V.0.1.
	\UCitem{Salidas}{Ninguno.}
	% V.0.1.
	\UCitem{Destino}{Pantalla.}
	% V.0.1.
	\UCitem{Precondiciones}{%
		\begin{Titemize}
			\Titem Haber estado registrado en el sistema.
			\Titem Haber iniciado sesión.
		\end{Titemize}
	}
	% V.0.1.
	\UCitem{Postcondiciones}{
		Actualización de la información de acceso.
	}
	% V.0.1.
	\UCitem{Reglas de Negocio}{%
		\begin{Titemize}
			\Titem \refIdElem{RN-MON-N01}
			\Titem \refIdElem{RN-MON-N03}
			\Titem \refIdElem{RN-MON-N04}
			\Titem \refIdElem{RN-MON-N05}
			\Titem \refIdElem{RN-MON-N13}
		\end{Titemize}
	}
	% V.0.1.
	\UCitem{Errores}{%
		\begin{Titemize}
			\Titem \UCerr{Uno}{Cuando el usuario no introduce todos 
			los campos marcados con asterisco,}{el sistema muestra 
			el mensaje \refIdElem{MSG6} y regresa al paso 
			\ref{AS-CU3:introducirNombreDeUsuario} de la trayectoria 
			principal.}
			
			\Titem \UCerr{Dos}{Cuando el campo nombre de usuario no 
			cumple con el formato de dato solicitado,}{se muestra el 
			mensaje \refIdElem{MSG8} y regresa al paso 
			\ref{AS-CU3:introducirNombreDeUsuario} de la 
			trayectoria principal.}
			
			\Titem \UCerr{Tres}{Cuando el campo correo electrónico no 
			cumple con el formato de dato solicitado,}{se muestra el 
			mensaje \refIdElem{MSG9} y regresa al paso 
			\ref{AS-CU3:introducirCorreoElectronico} de la 
			trayectoria principal.}
			
			\Titem \UCerr{Cuatro}{Cuando la contraseña no 
			cumple con el formato de dato solicitado,}{se muestra el 
			mensaje \refIdElem{MSG10} y regresa al paso 
			\ref{AS-CU3:introducirContrasenia} de la trayectoria 
			principal.}
			
			\Titem \UCerr{Cinco}{Cuando el campo confirmar 
			contraseña no cumple con el formato de dato solicitado,}
			{se muestra el mensaje \refIdElem{MSG10} y regresa al 
			paso \ref{AS-CU3:introducirConfirmarContrasenia} de 
			la trayectoria 
			principal.}
		
			\Titem \UCerr{Seis}{Cuando se introduce una contraseña 
			que no coincide con la confirmación de la misma,}{el 
			sistema muestra el mensaje \refIdElem{MSG15} y regresa 
			al paso \ref{AS-CU21:introducirContrasenia} de la 
			trayectoria principal.}
		\end{Titemize}				
	}
	% V.0.1.
	\UCitem{Viene de}{Ninguno.}
	% V.0.1.
	\UCitem{Disparadores}{%
		Requiere actualizar sus datos de acceso.
	}
	% V.0.1.
	\UCitem{Condiciones de Término}{%
		Se muestra el mensaje \refIdElem{MSG1} indicando 
		que se han actualizado sus datos de acceso al sistema.
	}
	% V.0.1.
	\UCitem{Efectos Colaterales}{Actualización de información de acceso al 
		sistema.
	}
	\UCitem{Datos sensibles}{
		\begin{Titemize}
			\Titem \refElem{Usuario.nombreUsuario}.
			\Titem \refElem{Cliente.correoElectronico}.
			\Titem \refElem{Usuario.contrasenia}.	
		\end{Titemize}
	}
\end{UseCase}

%Trayectoria Principal : Happy Path

\begin{UCtrayectoria}
	% V.0.1.
	\UCpaso [\UCactor] Presiona el botón \IUbutton{Ver perfil} de la 
	pantalla sesión del usuario: \refIdElem{GV-IU1}, \refIdElem{GV-IU2}, 
	\refIdElem{GH-IU1}.
	% V.0.1.
	\UCpaso Muestra la pantalla \refIdElem{AS-IU5}.
	% V.0.1.
	\UCpaso [\UCactor] \label{AS-CU3:introducirNombreDeUsuario} 
	Introducir información del campo correoElectronico.
	% V.0.1.
	\UCpaso [\UCactor] \label{AS-CU3:introducirCorreoElectronico} 
	Introducir información del campo correoElectronico.
	% V.0.1.
	\UCpaso [\UCactor] \label{AS-CU3:introducirContrasenia} Introducir 
	información del campo contraseña.
	% V.0.1.
	\UCpaso [\UCactor] \label{AS-CU3:introducirConfirmarContrasenia} 
	Introducir información del campo confirmar contraseña.
	% V.0.1.
	\UCpaso [\UCactor] Presiona el botón \IUbutton{Actualizar información}.
	% V.0.1.
	\UCpaso Verifica que no falte información en los campos marcados con 
	asterisco, con base a la regla de negocio \refIdElem{RN-MON-N01}. 
	\refErr{Uno}
	% V.0.1.
	\UCpaso Verifica si los campos del usuario cumplen con los formatos 
	requeridos de acuerdo a las reglas de negocio \refIdElem{RN-MON-N03}, 
	\refIdElem{RN-MON-N04}, \refIdElem{RN-MON-N05} y 
	\refIdElem{RN-MON-N13}. \refErr{Dos} 
	\refErr{Tres} \refErr{Cuatro} \refErr{Cinco}
	% V.0.1.
	\UCpaso Verifica que la contraseña y la confirmación de contraseña sean 
	identicas, de acuerdo a la regla de negocio: \refIdElem{RN-MON-N13}. 
	\refErr{Seis}
	% V.0.1.
	\UCpaso Recibir campos: correo electrónico, nombre de usuario, 
	contraseña.
	% V.0.1.
	\UCpaso Obtener usuario por correo electrónico.
	% V.0.1.
	\UCpaso Verificar que no exista duplicado de información de acuerdo a 
	las reglas de negocio: \refIdElem{RN-MON-N11}. \refTray{A}
	% V.0.1.
	\UCpaso \label{AS-CU3:actualizarCampos} Actualizar los campos: 
	nombre de usuario, correo electrónico, contraseña en base de datos.
	% V.0.1.
	\UCpaso Muestra el mensaje: \refIdElem{MSG1} en la pantalla 
	\refIdElem{AS-IU5}, con los siguientes parámetros:
	
	\begin{itemize}
		\item $ARTICULO$: 'La'.
		\item $OPERACION$: 'actualización de datos de usuario'.
	\end{itemize}
\end{UCtrayectoria}


%-------------------------- Trayectoria Alternativa A ---------------------------------

\begin{UCtrayectoriaA}{A}{No existe duplicado de información}
	% V.0.1.
	\UCpaso Verificar si no existe correo electrónico duplicado de acuerdo a 
	la regla de negocio: \refIdElem{RN-MON-N11}. \refTray{B}
	% V.0.1.
	\UCpaso \label{AS-CU3A:noDuplicados} Verificar si no existe nombre de 
	usuario duplicado de acuerdo a la regla de negocio: 
	\refIdElem{RN-MON-N12}. \refTray{C}
	% V.0.1.
	\UCpaso Continúa en el paso \ref{AS-CU3:actualizarCampos} de la 
	trayectoria principal.
\end{UCtrayectoriaA}

%-------------------------- Trayectoria Alternativa B --------------------------------- 

\begin{UCtrayectoriaA}{B}{Duplicado de correo electrónico}
	% V.0.1.
	\UCpaso Muestra el mensaje \refIdElem{MSG2}en pantalla 
	\refIdElem{AS-IU5}, con el siguiente parámetro:
	\begin{itemize}
		\item $OPERACION$: 'verificación de correo electrónico'.
	\end{itemize}
	
	%Existe un correo electronico
	%previamente registrado. Favor de verificar correo electrónico.
	% V.0.1.
	\UCpaso Continúa con el paso \ref{AS-CU3A:noDuplicados} de la 
	trayectoria alterna \textbf{A}.
\end{UCtrayectoriaA}

%-------------------------- Trayectoria Alternativa C --------------------------------- 

\begin{UCtrayectoriaA}[Termina caso de uso]{C}{Duplicado de nombre de usuario}
	% V.0.1.
	\UCpaso Muestra el mensaje \refIdElem{MSG2} en pantalla 
	\refIdElem{AS-IU5}, con el siguiente parámetro:
	\begin{itemize}
		\item $OPERACION$: 'verificación de nombre de usuario'.
	\end{itemize}
	
	%Existe un nombre de usuario
	%previamente registrado. Favor de verificar nombre de usuario.
\end{UCtrayectoriaA}

% Indice de Caso de uso: Enviar información.
% Autor(es): Landa Aguirre Rafael
% 	     Alan Vieyra Fernando
%	     Olvera Pérez Víctor David
% Fecha: 04 - May - 2018
%

\begin{UseCase}{CD-CU1}{Enviar información}{
		Permite al \refElem{Drone} comunicarse con la aplicación web 
		permitiendo enviar información cada periodo de tiempo, datos
		como: altura relativa, posición geográfica, estado de batería, 
		velocidad relativa, orientación magnética, estado del 
		\refElem{Drone} (en caso de haber ocurrido alguna falla).
	}
	\UCccitem{Versión}{0.1}
	\UCccsection{Datos para el control Interno}
	\UCccitem{Elaboró}{
		\begin{Titemize}
			\Titem Alan Fernando Rincón Vieyra.
			\Titem Rafael Landa Aguirre.
		\end{Titemize}
	}
	\UCccitem{Revisó}{----}
	\UCccitem{Estatus}{Revisado.}
	\UCccitem{Fecha del último estatus}{16 de Mayo de 2018}
	\UCccitem{Observaciones}{Ninguna.}
	\UCsection{Atributos}
	% V.0.1.
	\UCitem{Actor}{\refElem{Drone}}
	% V.0.1.
	\UCitem{Propósito}{
		Enviar información de \refElem{Drone} de acuerdo a la regla de 
		negocio \refIdElem{RN-MON-N07}, para que la aplicación
		web muestre información al \refElem{Cliente}.
	}
	% V.0.1.
	\UCitem{Entradas}{%
		\begin{Titemize}
			\Titem \refElem{Drone.posicion}.
			\Titem \refElem{Drone.altura}.
			\Titem \refElem{Drone.nivelBateria}.
			\Titem \refElem{Drone.velocidad}.
			\Titem \refElem{Drone.orientacion}.
			\Titem \refElem{Drone.estado}.	
		\end{Titemize}	
	}
	% V.0.1.
	\UCitem{Origen}{Ninguno.}
	% V.0.1.
	\UCitem{Salidas}{Ninguno.}
	% V.0.1.
	\UCitem{Destino}{Ninguno.}
	% V.0.1.
	\UCitem{Precondiciones}{%
		\begin{Titemize}
			\Titem \refElem{Drone} deberá haber estado encendido.
			\Titem El protocolo Mavlink deberá estar configurado en 
			modi activo.
			\Titem El enlace via GSM deberá estar activo.
		\end{Titemize}
	}
	% V.0.1.
	\UCitem{Postcondiciones}{Ninguno.}
	% V.0.1.
	\UCitem{Reglas de Negocio}{\refIdElem{RN-MON-N07}}
	% V.0.1.
	\UCitem{Errores}{%
		\begin{Titemize}
			\Titem \UCerr{Uno}{Cuando no existe conexión via GSM, }{
			la aplicación web no recibe paquetes de información y se
			tomará como error de conexión.}
		\end{Titemize}	
	}
	% V.0.1.
	\UCitem{Viene de}{Ninguno.}
	% V.0.1.
	\UCitem{Disparadores}{%
		\begin{Titemize}
			\Titem Requiere el envío de paquetes de información para 
			actualizar su posición geográfica, así como informar 
			acerca del estado del drone y de otras variables a 
			monitorear.
		\end{Titemize}
	} 
	% V.0.1.
	\UCitem{Condiciones de Término}{%
		Hasta que el drone llegue a su punto destino.
	}
	% V.0.1.
	\UCitem{Efectos Colaterales}{La aplicación web puede no obtener conexión
		con el \refElem{Drone}.
	}
	\UCitem{Datos sensibles}{
		\begin{Titemize}
			\Titem Altura relativa a la tierra.
			\Titem Posición geográfica.
			\Titem Estado de batería.
			\Titem Velocidad relativa.
			\Titem Orientación magnética.
			\Titem Estado del drone.
		\end{Titemize}
	}
\end{UseCase}


%Trayectoria Principal : Happy Path

\begin{UCtrayectoria}
	% V.0.1.
	\UCpaso [\UCactor] Iniciar estado de encendido.
	% V.0.1.
	\UCpaso [\UCactor] Iniciar despegue.
	% V.0.1.
	\UCpaso Iniciar estructura de bytes.
	% V.0.1.
	\UCpaso Iniciar comunicación via GSM.
	% V.0.1.
	\UCpaso \label{CD-CU1:posicionGeografica} Para obtener la posición 
	geográfica, incluye \textbf{CD-CU1.1}
	% V.0.1.
	\UCpaso Para obtener la altura, incluye \textbf{CD-CU1.2}
	% V.0.1.
	\UCpaso Para obtener el estado de batería, incluye \textbf{CD-CU1.3}
	% V.0.1.
	\UCpaso Para obtener la velocidad resultante, incluye \textbf{CD-CU1.4}
	% V.0.1.
	\UCpaso Para obtener la orientación, incluye \textbf{CD-CU1.5}
	% V.0.1.
	\UCpaso Para obtener el estado del drone, incluye \textbf{CD-CU1.6}
	% V.0.1.
	\UCpaso Actualizar estructura de bytes con la información de los pasos 
	anteriores desde el paso \ref{CD-CU1:posicionGeografica}.
	% V.0.1.
	\UCpaso Adjuntar estructura como cadena o mensaje nuevo.
	% V.0.1.
	\UCpaso	Enviar comando de envío de cadena o mensaje nuevo. \refErr{Uno}
\end{UCtrayectoria}


%-------------------------- Trayectoria Alternativa A ---------------------------------

%\begin{UCtrayectoriaA}{A}{Se requiere seleccionar todas las unidades académicas }
	% V.0.1 TODO.
%	\UCpaso [\UCactor] 	Presiona el botón \IUbutton{Seleccionar Todo} de la pantalla \refIdElem{IN-DAE-UI2.1}.
	% V.0.1 TODO.
%	\UCpaso Continúa en el paso \ref{IN-DAE-CU2.1:sel}  de la trayectoria principal.
%\end{UCtrayectoriaA}

%-------------------------- Trayectoria Alternativa F --------------------------------- 

%\begin{UCtrayectoriaA}[Termina caso de uso]{F}{Se requiere Cancelar la operación}
	% V.0.1 TODO.
%	\UCpaso [\UCactor] 	Presiona el botón \IUbutton{Cancelar} de la pantalla \refIdElem{IN-DAE-UI2.1}.
	% V.0.1 TODO.
%	\UCpaso Muestra la pantalla \refIdElem{IN-DAE-UI2} o \refIdElem{IN-DAE-UI2a} 
%\end{UCtrayectoriaA}

%-------------------------- Trayectoria Alternativa G --------------------------------- 

%\begin{UCtrayectoriaA}{G}{Se requiere agregar una Unidad Académica}
	% V.0.1 TODO.
%	\UCpaso [\UCactor]  Selecciona el acrónimo de Unidad Académica que se requiere agregar a la asociación del calendario. 
	% V.0.1 TODO.
%	\UCpaso Sigue en el paso \ref{IN-DAE-CU2.1:sel}  de la trayectoria principal.
%\end{UCtrayectoriaA}


%\subsubsection{Puntos de extensión}

%\UCExtensionPoint{Gestionar Calendario Escolar}{El actor requiere Gestionar Calendario Escolar}{ Paso \ref{DAE-IN-CU1:acp} de la Trayectoria Principal}{\refIdElem{IN-DAE-CU2}}

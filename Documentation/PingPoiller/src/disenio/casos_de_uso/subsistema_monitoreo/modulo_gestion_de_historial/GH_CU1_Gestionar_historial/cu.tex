% Indice de trayectorias: Gestión de historial.
% Autor(es): Landa Aguirre Rafael
% 	     Alan Vieyra Fernando
%	     Olvera Pérez Víctor David
% Fecha: 04 - May - 2018
%

\begin{UseCase}{GH-CU1}{Gestionar historial}{
		Permite al \refElem{Cliente} puede gestionar el historial de 
		solicitudes que a realizado en un periodo de tiempo, permitiendo
		visualización individual de cada vuelo, eliminación de 
		solicitudes realizadas previamente.
	}
	\UCccitem{Versión}{0.1}
	\UCccsection{Datos para el control Interno}
	\UCccitem{Elaboró}{
		\begin{Titemize}
			\Titem Alan Fernando Rincón Vieyra.
			\Titem Rafael Landa Aguirre.
		\end{Titemize}
	}
	\UCccitem{Revisó}{----}
	\UCccitem{Estatus}{Revisado.}
	\UCccitem{Fecha del último estatus}{16 de Mayo de 2018}
	\UCccitem{Observaciones}{Ninguna.}
	\UCsection{Atributos}
	% V.0.1.
	\UCitem{Actor}{\refElem{Cliente}.}
	% V.0.1.
	\UCitem{Propósito}{
		Gestionar el historial de \refElem{Vuelo} de solicitudes realizadas previamente, permitiendo realizar las siguientes acciones: \cdtEmpty
		
		\begin{itemize}
			\item Consultar lista de solicitudes por usuario.
			\item Consultar por solicitud.
			\item Eliminar registro de la lista de solicitudes.
		\end{itemize}
	}
	% V.0.1.
	\UCitem{Entradas}{%
		Ninguno.	
	}
	% V.0.1.
	\UCitem{Origen}{%
		\begin{Titemize}
			\Titem \ioSeleccionar
		\end{Titemize}
	}
	% V.0.1.
	\UCitem{Salidas}{%
		\begin{Titemize}
			\Titem Lista de solicitudes de \refElem{Vuelo} organizadas por fecha desde las más recientes.
			\Titem \refIdElem{MSG3}
		\end{Titemize}
	}
	% V.0.1.
	\UCitem{Destino}{Pantalla.}
	% V.0.1.
	\UCitem{Precondiciones}{
		\begin{Titemize}
			\Titem Existen solicitudes de vuelo previemente 
			realizadas.
			\Titem Haber iniciado una nueva sesión dentro de la 
			aplicación.
		\end{Titemize}
	}
	% V.0.1.
	\UCitem{Postcondiciones}{%
		\begin{Titemize}
			\Titem Mostrar lista de solicitudes.
		\end{Titemize}
	}
	% V.0.1.
	\UCitem{Reglas de Negocio}{%
		\begin{Titemize}
			\Titem \refIdElem{RN-MON-N08}.
		\end{Titemize}
	}
	% V.0.1.
	\UCitem{Errores}{Ninguno.
		%\begin{Titemize}
		%	\Titem \UCerr{Uno}{Cuando no existen elementos o 
		%	registros,}{el sistema muestra \refIdElem{MSG3} con el 
		%	siguiente parámetro: $ELEMENTO:$ 'no hay registros de 
		%	solicitudes de vuelo'.}
		%\end{Titemize}				
	}
	% V.0.1.
	\UCitem{Viene de}{Ninguno.}
	% V.0.1.
	\UCitem{Disparadores}{%
		\begin{Titemize}
			\Titem Requiere consultar el histórico de registros de 
			vuelo.
			\Titem Requiere visualizar información de cada solicitud 
			que haya realizado.
		\end{Titemize}
	}
	% V.0.1.
	\UCitem{Condiciones de Término}{
		Hasta que el usuario termine su sesión activa.
	}
	% V.0.1.
	\UCitem{Efectos Colaterales}{Ninguno.}
	\UCitem{Datos sensibles}{Ninguno.}
\end{UseCase}


%Trayectoria Principal : Happy Path

\begin{UCtrayectoria}
	% V.0.1.
	\UCpaso [\UCactor] Presiona la pestaña o botón 
	\IUbutton{Ver Historial de vuelos} del menú principal.
	% V.0.1.
	\UCpaso Muestra la pantalla \refIdElem{GH-IU1}.
	% V.0.1.
	\UCpaso \label{GH_CU1:opcion1} Consultar registros de vuelos en base de 
	datos. \refTray{A}
	% V.0.1.
	\UCpaso Mostrar lista de los resultados obtenidos desde base de datos.
	% V.0.1.
	\UCpaso En caso de que el \refElem{Cliente} requiera visualizar o 
	consultar una solicitud, ver \textbf{GH-CU11}.
\end{UCtrayectoria}


%-------------------------- Trayectoria Alternativa A ---------------------------------

\begin{UCtrayectoriaA}{A}{No existen elementos disponibles}
	% V.0.1 TODO.
	\UCpaso Muestra el mensaje \refIdElem{MSG3}, con los siguientes 
	parámetros:
	\begin{itemize}
		\item $ELEMENTOS:$ 'solicitudes registradas'
	\end{itemize}
	% V.0.1 TODO.
	\UCpaso Continúa en el paso \ref{GH_CU1:opcion1}  de la trayectoria 
	principal.
\end{UCtrayectoriaA}

%-------------------------- Trayectoria Alternativa F --------------------------------- 

%\begin{UCtrayectoriaA}[Termina caso de uso]{F}{Se requiere Cancelar la operación}
	% V.0.1 TODO.
%	\UCpaso [\UCactor] 	Presiona el botón \IUbutton{Cancelar} de la pantalla \refIdElem{IN-DAE-UI2.1}.
	% V.0.1 TODO.
%	\UCpaso Muestra la pantalla \refIdElem{IN-DAE-UI2} o \refIdElem{IN-DAE-UI2a} 
%\end{UCtrayectoriaA}

%-------------------------- Trayectoria Alternativa G --------------------------------- 

%\begin{UCtrayectoriaA}{G}{Se requiere agregar una Unidad Académica}
	% V.0.1 TODO.
%	\UCpaso [\UCactor]  Selecciona el acrónimo de Unidad Académica que se requiere agregar a la asociación del calendario. 
	% V.0.1 TODO.
%	\UCpaso Sigue en el paso \ref{IN-DAE-CU2.1:sel}  de la trayectoria principal.
%\end{UCtrayectoriaA}


%\subsubsection{Puntos de extensión}

%\UCExtensionPoint{Gestionar Calendario Escolar}{El actor requiere Gestionar Calendario Escolar}{ Paso \ref{DAE-IN-CU1:acp} de la Trayectoria Principal}{\refIdElem{IN-DAE-CU2}}

% Trayectorias de Caso de uso: realizar de movimiento.
% Autor(es): Landa Aguirre Rafael
% 	     Alan Vieyra Fernando
%	     Olvera Pérez Víctor David
% Fecha: 04 - May - 2018
%

\begin{UseCase}{CM-CU1}{Realizar movimiento}{
		Permite al \refElem{Drone} poder realizar un movimiento en 
		alguna dirección (Arriba, Abajo, Adelante, Izquierda, Derecha, 
		Girar izquierda, Girar derecha).
	}
	\UCccitem{Versión}{0.1}
	\UCccsection{Datos para el control Interno}
	\UCccitem{Elaboró}{Rafael Landa Aguirre}
	\UCccitem{Supervisó}{-----}
	\UCccitem{Estatus}{\TODO[Por revisar]}
	\UCccitem{Fecha del último estatus}{11 de Mayo de 2018}
	\UCccitem{Observaciones}{\TODO[Por revisar]}
	\UCsection{Atributos}
	% V.0.1.
	\UCitem{Actor}{\refElem{Drone}}
	% V.0.1.
	\UCitem{Propósito}{Realizar un movimiento en alguna dirección: 
			Arriba, Abajo, Izquierda, derecha, Girar derecha, Girar 
			izquierda.}
	% V.0.1.
	\UCitem{Entradas}{Ninguna.}
	% V.0.1.
	\UCitem{Origen}{Ninguno.}
	% V.0.1.
	\UCitem{Salidas}{Ninguno.}
	% V.0.1.
	\UCitem{Destino}{Ninguno.}
	% V.0.1.
	\UCitem{Precondiciones}{%
		\begin{Titemize}
			\Titem Estado de encendido.
			\Titem GPS en estado de activo.
			\Titem Comunicación con el módulo de proximidad.
		\end{Titemize}
	}
	% V.0.1.
	\UCitem{Postcondiciones}{El drone tendrá que haberse movido en alguna 
			dirección: Arriba, Abajo, Adelante, Izquierda, Derecha, 
			Girar izquierda, Girar derecha.}
	% V.0.1.
	\UCitem{Reglas de Negocio}{Ninguno.}
	% V.0.1.
	\UCitem{Errores}{%
		\begin{Titemize}
			\Titem \UCerr{Uno}{Cuando no existe conexión con el 
			sensor,}{el \refElem{Drone} responderá con un 
			aterrizaje de precaución.}
		\end{Titemize}				
	}
	% V.0.1.
	\UCitem{Viene de}{Ninguno}
	% V.0.1 TODO.
	\UCitem{Disparadores}{Requiere de movimiento para llegar a su punto 
	destino.} 
	% V.0.1 TODO.
	\UCitem{Condiciones de Término}{%
		Hasta que el drone haya llegado a su objetivo final.}
	% V.0.1 TODO.
	\UCitem{Efectos Colaterales}{El drone podrá moverse hasta que detecte 
obstáculos, y modificará su dirección de movimiento cuando se requiera.}
	\UCitem{Datos sensibles}{Distancia}
\end{UseCase}


%Trayectoria Principal : Happy Path

\begin{UCtrayectoria}
	% V.0.1.
	\UCpaso [\UCactor] El drone comienza su despegue.
	% V.0.1.
	\UCpaso [\UCactor] El drone detecta comunicación módulo de proximidad.	
	% V.0.1.
	\UCpaso Si el drone detecta un objeto arriba de este, ver \refIdElem{CM-CU1.1}.
	% V.0.1.
	\UCpaso Si el drone detecta un objeto abajo, ver \refIdElem{CM-CU1.2}.
	% V.0.1.
	\UCpaso Si el drone detecta un objeto adelante, ver \refIdElem{CM-CU1.3}.
	% V.0.1.
	\UCpaso Si el drone detecta un objeto atras, ver \refIdElem{CM-CU1.4}.
	% V.0.1.
	\UCpaso Si el drone tiene que girar a la izquierda, ver \refIdElem{CM-CU1.5}.
	% V.0.1.
	\UCpaso Si el drone tiene que girar a la derecha, ver \refIdElem{CM-CU1.6}.
	% V.0.1.
	\UCpaso Si el drone tiene detenerse, ver \refIdElem{CM-CU1.7}.
\end{UCtrayectoria}

%-------------------------- Trayectoria Alternativa A ---------------------------------

%\begin{UCtrayectoriaA}{A}{Se requiere seleccionar todas las unidades académicas }
	% V.0.1 TODO.
%	\UCpaso [\UCactor] 	Presiona el botón \IUbutton{Seleccionar Todo} de la pantalla \refIdElem{IN-DAE-UI2.1}.
	% V.0.1 TODO.
%	\UCpaso Continúa en el paso \ref{IN-DAE-CU2.1:sel}  de la trayectoria principal.
%\end{UCtrayectoriaA}

%-------------------------- Trayectoria Alternativa F --------------------------------- 

%\begin{UCtrayectoriaA}[Termina caso de uso]{F}{Se requiere Cancelar la operación}
	% V.0.1 TODO.
%	\UCpaso [\UCactor] 	Presiona el botón \IUbutton{Cancelar} de la pantalla \refIdElem{IN-DAE-UI2.1}.
	% V.0.1 TODO.
%	\UCpaso Muestra la pantalla \refIdElem{IN-DAE-UI2} o \refIdElem{IN-DAE-UI2a} 
%\end{UCtrayectoriaA}

%-------------------------- Trayectoria Alternativa G --------------------------------- 

%\begin{UCtrayectoriaA}{G}{Se requiere agregar una Unidad Académica}
	% V.0.1 TODO.
%	\UCpaso [\UCactor]  Selecciona el acrónimo de Unidad Académica que se requiere agregar a la asociación del calendario. 
	% V.0.1 TODO.
%	\UCpaso Sigue en el paso \ref{IN-DAE-CU2.1:sel}  de la trayectoria principal.
%\end{UCtrayectoriaA}


\subsubsection{Puntos de extensión}

\UCExtensionPoint{Move hacia arriba}{El actor requiere moverse hacia arriba en caso de detectar un obstáculo por abajo de éste}{ Paso 3 de la Trayectoria Principal}{\refIdElem{CM-CU1.1}}

\UCExtensionPoint{Mover hacia abajo}{El actor moverse hacia abajo en caso de detectar un obstáculo por arriba de éste}{ Paso 4 de la Trayectoria Principal}{\refIdElem{CM-CU1.2}}

\UCExtensionPoint{Mover hacia adelante}{El actor requiere moverse hacia adelante en caso de detectar un obstáculo detras de éste o que no exista un obstáculo por delante}{ Paso 5 de la Trayectoria Principal}{\refIdElem{CM-CU1.3}}

\UCExtensionPoint{Mover hacia atras}{El actor requiere moverse hacia atras en caso de detectar un obstáculo por delante de éste.}{ Paso 6 de la Trayectoria Principal}{\refIdElem{CM-CU1.4}}

\UCExtensionPoint{Girar a la izquierda}{El actor requiere girar a la izquierda en caso de cambiar de orientación}{ Paso 7 de la Trayectoria Principal}{\refIdElem{CM-CU1.5}}

\UCExtensionPoint{Girar a la derecha}{El actor requiere girar a la izquierda en caso de cambiar de orientación}{ Paso 8 de la Trayectoria Principal}{\refIdElem{CM-CU1.6}}

\UCExtensionPoint{Detener}{El actor requiere detenerse en caso de haber llegado a su destino o detectar la presencia de un obstáculo}{ Paso 9 de la Trayectoria Principal}{\refIdElem{CM-CU1.7}}

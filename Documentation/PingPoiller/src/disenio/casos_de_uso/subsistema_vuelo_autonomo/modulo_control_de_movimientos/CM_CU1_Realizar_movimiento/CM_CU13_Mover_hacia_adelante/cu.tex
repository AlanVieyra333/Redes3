% Trayectorias de Caso de uso: Mover hacia adelante.
% Autor(es): Landa Aguirre Rafael
% 	     Alan Vieyra Fernando
%	     Olvera Pérez Víctor David
% Fecha: 04 - May - 2018
%

\begin{UseCase}{CM-CU1.3}{Mover hacia adelante}{
		Permite al \refElem{Drone} moverse en dirección hacia adelante, 
		en caso moverse de un lugar a otro como movimiento por defecto
		al iniciar su trayectoria.
	}
	\UCccitem{Versión}{0.1}
	\UCccsection{Datos para el control Interno}
	\UCccitem{Elaboró}{Rafael Landa Aguirre}
	\UCccitem{Supervisó}{-----}
	\UCccitem{Estatus}{\TODO[Por revisar]}
	\UCccitem{Fecha del último estatus}{11 de Mayo de 2018}
	\UCccitem{Observaciones}{\TODO[Por revisar]}
	\UCsection{Atributos}
	% V.0.1.
	\UCitem{Actor}{\refElem{Drone}}
	% V.0.1.
	\UCitem{Propósito}{Realizar el control del movimiento del 
			\refElem{Drone} en dirección hacia adelante, con el 
			objetivo llegar a su punto destino.
	}
	% V.0.1.
	\UCitem{Entradas}{Ninguno.}
	% V.0.1.
	\UCitem{Origen}{Ninguno.}
	% V.0.1.
	\UCitem{Salidas}{Ninguno.}
	% V.0.1.
	\UCitem{Destino}{Ninguno.}
	% V.0.1.
	\UCitem{Precondiciones}{%
		\begin{Titemize}
			\Titem Estado de encendido.
			\Titem Exista conexión con el módulo de proximidad.
		\end{Titemize}
	}
	% V.0.1.
	\UCitem{Postcondiciones}{Realizar el movimiento del \refElem{Drone} en 
			dirección hacia adelante.}
	% V.0.1.
	\UCitem{Reglas de Negocio}{Ninguno.}
	% V.0.1.
	\UCitem{Errores}{%
		\begin{Titemize}
			\Titem \UCerr{Uno}{Cuando no existe conexión con el 
			sensor,}{el \refElem{Drone} responderá con un 
			aterrizaje de precaución.}
		\end{Titemize}
	}
	% V.0.1.
	\UCitem{Viene de}{\refIdElem{CM-CU1}}
	% V.0.1.
	\UCitem{Disparadores}{%
		\begin{Titemize}
			\Titem Requiere de movimiento hacia arriba para esquivar 
			un obstáculo en caso de encontrarlo por debajo del 
			\refElem{Drone}.
			\Titem Requiere de movimiento para llegar a su punto 
			destino.
		\end{Titemize}
	} 
	% V.0.1.
	\UCitem{Condiciones de Término}{%
		Hasta la detección de un obstáculo en la dirección de su 
		movimiento.
	}
	% V.0.1.
	\UCitem{Efectos Colaterales}{No pueda complir con el objetivo de llegar 
		a su punto destino.
	}
	\UCitem{Datos sensibles}{Ninguno.}
\end{UseCase}


%Trayectoria Principal : Happy Path

\begin{UCtrayectoria}
	% V.0.1.
	\UCpaso [\UCactor] Detectar comunicación con el módulo de proximidad.
		\refErr{Uno}
	% V.0.1.
	\UCpaso Obtener estado de PWM de movimiento vertical (yaw), movimiento 
	horizontal adelante (pitch), movimiento horizontal lateral (roll), 
	y acelerador (throttle).
	% V.0.1.
	\UCpaso Incrementar unitariamente estado de PWM para el pitch, a partir 
	del valor inicial del rango de valores (1101 - 1901).
	% V.0.1.
	\UCpaso Detectar si existe un objeto, ver \refIdElem{CM-CU1.7}. Termina 
	caso de uso. En caso de no existir proseguir al paso 2.
\end{UCtrayectoria}


%-------------------------- Trayectoria Alternativa A ---------------------------------

%\begin{UCtrayectoriaA}{A}{Se requiere seleccionar todas las unidades académicas }
	% V.0.1 TODO.
%	\UCpaso [\UCactor] 	Presiona el botón \IUbutton{Seleccionar Todo} de la pantalla \refIdElem{IN-DAE-UI2.1}.
	% V.0.1 TODO.
%	\UCpaso Continúa en el paso \ref{IN-DAE-CU2.1:sel}  de la trayectoria principal.
%\end{UCtrayectoriaA}

%-------------------------- Trayectoria Alternativa F --------------------------------- 

%\begin{UCtrayectoriaA}[Termina caso de uso]{F}{Se requiere Cancelar la operación}
	% V.0.1 TODO.
%	\UCpaso [\UCactor] 	Presiona el botón \IUbutton{Cancelar} de la pantalla \refIdElem{IN-DAE-UI2.1}.
	% V.0.1 TODO.
%	\UCpaso Muestra la pantalla \refIdElem{IN-DAE-UI2} o \refIdElem{IN-DAE-UI2a} 
%\end{UCtrayectoriaA}

%-------------------------- Trayectoria Alternativa G --------------------------------- 

%\begin{UCtrayectoriaA}{G}{Se requiere agregar una Unidad Académica}
	% V.0.1 TODO.
%	\UCpaso [\UCactor]  Selecciona el acrónimo de Unidad Académica que se requiere agregar a la asociación del calendario. 
	% V.0.1 TODO.
%	\UCpaso Sigue en el paso \ref{IN-DAE-CU2.1:sel}  de la trayectoria principal.
%\end{UCtrayectoriaA}

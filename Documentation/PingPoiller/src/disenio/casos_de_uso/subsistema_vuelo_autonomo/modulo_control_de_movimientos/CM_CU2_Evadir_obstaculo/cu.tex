% Caso de uso: Evadir obstáculo.
% Autor(es): Landa Aguirre Rafael
% 	     Alan Vieyra Fernando
%	     Olvera Pérez Víctor David
% Fecha: 17 - May - 2018
%

\begin{UseCase}{CM-CU2}{Evadir obstáculo}{
		Permite al \refElem{Drone} detectar la presencia de un obstáculo 
		en la dirección a la que se mueve, con el fin de moverse en 
		una dirección que evite el choque, y además, con el objetivo de 
		llegar a un obstáculo.
	}
	\UCccitem{Versión}{0.1}
	\UCccsection{Datos para el control Interno}
	\UCccitem{Elaboró}{Rafael Landa Aguirre}
	\UCccitem{Supervisó}{-----}
	\UCccitem{Estatus}{\TODO[Por revisar]}
	\UCccitem{Fecha del último estatus}{09 de Mayo de 2018}
	\UCccitem{Observaciones}{\TODO[Por revisar]}
	\UCsection{Atributos}
	% V.0.1.
	\UCitem{Actor}{
		\begin{Titemize}
			\Titem \refElem{Drone}
		\end{Titemize}
	}
	% V.0.1.
	\UCitem{Propósito}{
		\begin{Titemize}
			\Titem Realizar el control del movimiento del 
			\refElem{Drone} en dirección hacia arriba, con el fin de 
			evitar un obstáculo o para empezar su despegue de vuelo.
		\end{Titemize}
	}
	% V.0.1.
	\UCitem{Entradas}{%
		\begin{Titemize}
			\Titem Distancia.	
		\end{Titemize}	
	}
	% V.0.1.
	\UCitem{Origen}{%
		\begin{Titemize}
			\Titem Ninguno.
		\end{Titemize}
	}
	% V.0.1.
	\UCitem{Salidas}{%
		\begin{Titemize}
			\Titem Ninguno.
		\end{Titemize}	
	}
	% V.0.1.
	\UCitem{Destino}{Ninguno.}
	% V.0.1.
	\UCitem{Precondiciones}{%
		\begin{Titemize}
			\Titem Estado de encendido.
			\Titem Exista conexión con el módulo de proximidad.
		\end{Titemize}
	}
	% V.0.1.
	\UCitem{Postcondiciones}{%
		\begin{Titemize}
			\Titem Realizar el movimiento del \refElem{Drone}.
		\end{Titemize}
	}
	% V.0.1.
	\UCitem{Reglas de Negocio}{Ninguno.}
	% V.0.1.
	\UCitem{Errores}{%
		\begin{Titemize}
			\Titem \UCerr{Uno}{Cuando no existe conexión con el 
			sensor,}{el \refElem{Drone} responderá con un 
			aterrizaje de precaución.}
		\end{Titemize}				
	}
	% V.0.1.
	\UCitem{Viene de}{Ninguno.}
	% V.0.1.
	\UCitem{Disparadores}{%
		\begin{Titemize}
			\Titem Requiere la detección de un obstáculo en un rango 
			programado, para evitar un choque.
		\end{Titemize}
	} 
	% V.0.1.
	\UCitem{Condiciones de Término}{%
		Hasta que el drone haya llegado a su objetivo final.}
	% V.0.1.
	\UCitem{Efectos Colaterales}{Ninguno.}
	\UCitem{Datos sensibles}{Ninguno.}
\end{UseCase}


%Trayectoria Principal : Happy Path

\begin{UCtrayectoria}
	% V.0.1.
	\UCpaso [\UCactor] El drone comienza su despegue.
	% V.0.1.
	\UCpaso [\UCactor] El drone detecta comunicación módulo de proximidad.
		\refErr{Uno}
	% V.0.1.
	\UCpaso \label{CM-CU2:cu1_7} Véase CM-CU1.7.
	% V.0.1.
	\UCpaso Obtener información del módulo de proximidad, ver 
		\refIdElem{PR-CU1}.
	% V.0.1.
	\UCpaso Verifica si el drone ha llegado a su punto destino. En tal caso, 
	pasar el paso \ref{CM-CU2:detectaObstaculo} de la trayectoria principal.
	% V.0.1.
	\UCpaso \label{CM-CU2:detectaObstaculo} Si el drone detecta un 
	obstáculo, ver \refIdElem{CM-CU1}. En caso contrario, regresar el paso 
	\ref{CM-CU2:cu1_7} de la trayectoria principal.
	% V.0.1.
	\UCpaso Realiza el movimiento de aterrizaje del drone.
\end{UCtrayectoria}


%-------------------------- Trayectoria Alternativa A ---------------------------------

%\begin{UCtrayectoriaA}{A}{Se requiere seleccionar todas las unidades académicas }
	% V.0.1 TODO.
%	\UCpaso [\UCactor] 	Presiona el botón \IUbutton{Seleccionar Todo} de la pantalla \refIdElem{IN-DAE-UI2.1}.
	% V.0.1 TODO.
%	\UCpaso Continúa en el paso \ref{IN-DAE-CU2.1:sel}  de la trayectoria principal.
%\end{UCtrayectoriaA}

%-------------------------- Trayectoria Alternativa F --------------------------------- 

%\begin{UCtrayectoriaA}[Termina caso de uso]{F}{Se requiere Cancelar la operación}
	% V.0.1 TODO.
%	\UCpaso [\UCactor] 	Presiona el botón \IUbutton{Cancelar} de la pantalla \refIdElem{IN-DAE-UI2.1}.
	% V.0.1 TODO.
%	\UCpaso Muestra la pantalla \refIdElem{IN-DAE-UI2} o \refIdElem{IN-DAE-UI2a} 
%\end{UCtrayectoriaA}

%-------------------------- Trayectoria Alternativa G --------------------------------- 

%\begin{UCtrayectoriaA}{G}{Se requiere agregar una Unidad Académica}
	% V.0.1 TODO.
%	\UCpaso [\UCactor]  Selecciona el acrónimo de Unidad Académica que se requiere agregar a la asociación del calendario. 
	% V.0.1 TODO.
%	\UCpaso Sigue en el paso \ref{IN-DAE-CU2.1:sel}  de la trayectoria principal.
%\end{UCtrayectoriaA}


%\subsubsection{Puntos de extensión}

%\UCExtensionPoint{Gestionar Calendario Escolar}{El actor requiere Gestionar Calendario Escolar}{ Paso \ref{DAE-IN-CU1:acp} de la Trayectoria Principal}{\refIdElem{IN-DAE-CU2}}

% Trayectorias de recibir información del sensor de proximidad
% Autor(es): Landa Aguirre Rafael
% 	     Alan Vieyra Fernando
%	     Olvera Pérez Víctor David
% Fecha: 04 - May - 2018
%

\begin{UseCase}{PR-CU1}{Recibir información del sensor de proximidad}{
		Permite al \refElem{Drone} obtener lectura de información de la 
		existencia de un objeto brindada por el sensor o sensores de 
		proximidad.
	}
	\UCccitem{Versión}{0.1}
	\UCccsection{Datos para el control Interno}
	\UCccitem{Elaboró}{Rafael Landa Aguirre}
	\UCccitem{Supervisó}{-----}
	\UCccitem{Estatus}{\TODO[Por revisar]}
	\UCccitem{Fecha del último estatus}{09 de Mayo de 2018}
	\UCccitem{Observaciones}{\TODO[Por revisar]}
	\UCsection{Atributos}
	% V.0.1.
	\UCitem{Actor}{\refElem{Drone}}
	% V.0.1.
	\UCitem{Propósito}{Recibir información del sensor(es) de proximidad.}
	% V.0.1.
	\UCitem{Entradas}{Distancia del sensor de proximidad.}
	% V.0.1.
	\UCitem{Origen}{Ninguno.}
	% V.0.1.
	\UCitem{Salidas}{Información de distancia.}
	% V.0.1.
	\UCitem{Destino}{Ninguno.}
	% V.0.1.
	\UCitem{Precondiciones}{Debe existir comunicación con el sensor de 
			proximidad.}
	% V.0.1.
	\UCitem{Postcondiciones}{Ninguno.}
	% V.0.1.
	\UCitem{Reglas de Negocio}{\refIdElem{RN-VA-N01}}
	% V.0.1.
	\UCitem{Errores}{%
		\begin{Titemize}
			\Titem \UCerr{Uno}{Cuando no existe conexión con el 
			sensor,}{ el \refElem{Drone} inmediatamente responderá 
			con un aterrizaje de precaución.}
		\end{Titemize}				
	}
	% V.0.1.
	\UCitem{Viene de}{Ninguno}
	% V.0.1.
	\UCitem{Disparadores}{El drone requiere recibir información del sensor 
			de proximidad para detectar la existencia de objetos.} 
	% V.0.1.
	\UCitem{Condiciones de Término}{
		El drone está en estado de aterrizaje.}
	% V.0.1.
	\UCitem{Efectos Colaterales}{El drone podrá hacer uso de la información 
	de detección de un objeto, para llevar a cabo toma de decisiones.}
	\UCitem{Datos sensibles}{Distancia.}
\end{UseCase}


%Trayectoria Principal : Happy Path

\begin{UCtrayectoria}
	% V.0.1.
	\UCpaso [\UCactor] Lectura de paquetes vía Mavlink.
	% V.0.1.
	\UCpaso Inicializar puertos de entrada/salida del microcontrolador.	
	% V.0.1.
	\UCpaso Para cada conexión a un sensor de proximidad, dense los pasos 
	siguientes:
	% V.0.1.
	\UCpaso Iniciar el puerto de conexión con el pin “trigger” del sensor en 
	estado de bajo o 0 lógico.
	% V.0.1.
	\UCpaso Enviar señal de retardo en estado de alto o 1 lógico de 10 us.
	% V.0.1.
	\UCpaso Esperar el arribo de respuesta de la señal enviada. \refErr{Uno}
	% V.0.1.
	\UCpaso Calcular la distancia del basandose en el tiempo de espera de la 
	señal dada en \refIdElem{RN-VA-N01}.
	% V.0.1.
	\UCpaso Empaquetar la distancia calculada en un paquete mavlink.
	% V.0.1.
	\UCpaso Enviar paquete mavlink al drone via serial USB.
	% V.0.1.
	\UCpaso Mientras el drone esté en estado de vuelo, regresar al paso 4.
\end{UCtrayectoria}


%-------------------------- Trayectoria Alternativa A ---------------------------------

%\begin{UCtrayectoriaA}{A}{Se requiere seleccionar todas las unidades académicas }
	% V.0.1 TODO.
%	\UCpaso [\UCactor] 	Presiona el botón \IUbutton{Seleccionar Todo} de la pantalla \refIdElem{IN-DAE-UI2.1}.
	% V.0.1 TODO.
%	\UCpaso Continúa en el paso \ref{IN-DAE-CU2.1:sel}  de la trayectoria principal.
%\end{UCtrayectoriaA}

%-------------------------- Trayectoria Alternativa F --------------------------------- 

%\begin{UCtrayectoriaA}[Termina caso de uso]{F}{Se requiere Cancelar la operación}
	% V.0.1 TODO.
%	\UCpaso [\UCactor] 	Presiona el botón \IUbutton{Cancelar} de la pantalla \refIdElem{IN-DAE-UI2.1}.
	% V.0.1 TODO.
%	\UCpaso Muestra la pantalla \refIdElem{IN-DAE-UI2} o \refIdElem{IN-DAE-UI2a} 
%\end{UCtrayectoriaA}

%-------------------------- Trayectoria Alternativa G --------------------------------- 

%\begin{UCtrayectoriaA}{G}{Se requiere agregar una Unidad Académica}
	% V.0.1 TODO.
%	\UCpaso [\UCactor]  Selecciona el acrónimo de Unidad Académica que se requiere agregar a la asociación del calendario. 
	% V.0.1 TODO.
%	\UCpaso Sigue en el paso \ref{IN-DAE-CU2.1:sel}  de la trayectoria principal.
%\end{UCtrayectoriaA}

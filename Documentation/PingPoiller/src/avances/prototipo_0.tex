\subsection{Prototipo 0: Control de drone manualmente}

\subsubsection{Objetivo}
Manipular de manera manual el drone \textbf{Bebop 2} de Parrot usando el firmware \textbf{Ardupilot} versión 3.5.5 con un control remoto (joystick).

\subsubsection{Descripción}
El drone deberá conectarse vía WI-FI hacia una computadora usando el programa Mission Planner, mismo que debe reconocer el mando joytick conectado a la computadora. Una vez hecha la conexión se deberá manipular el drone con el joystick, alevándolo sobre el suelo a una distancia de al menos 1 metro permaneciendo en su posición durante al menos 3 segundos.

\subsubsection{Resultados}

\begin{prueba}[{\bf Subsistema:} Vuelo Autónomo, {\bf Módulo:} Control de movimientos, {\bf Caso de uso:} CM-CU1 Realizar movimiento]{CM-CU1-P1}{Despegue del drone}
	\Pitem{Elevación de drone}
	\pregunta{¿Se armaron los motores del drone?}{X}{}{Ninguna.}
	\pregunta{¿El drone se elevó?}{X}{}{Ninguna.}
	\pregunta{¿El drone se mantuvo en su posición durante al menos 3 segundos?}{X}{}{Ninguna.}
\end{prueba}
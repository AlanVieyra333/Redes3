\subsection{Prototipo 1: Control de drone automáticamente}

\subsubsection{Objetivo}
Manipular de manera autónoma el drone \textbf{Bebop 2} de Parrot usando el firmware \textbf{Ardupilot} versión 3.5.5 logrando elevarlo por lo menos un metro.

\subsubsection{Descripción}
El drone deberá elevarse automáticamente sobre el suelo a una distancia de al menos 1 metro permaneciendo en su posición durante al menos 3 segundos.

\subsubsection{Resultados}
En la figura \ref{fig:test1} se muestra el drone en el suela esperando captar el GPS e iniciar a elevarse.

En la figura \ref{fig:test1.1} se muestra el drone a una distancia de 2 metros sobre el suelo.

\fig{test/img_prueba_1_autonomia_1}{test1}{Drone esperando la señal GPS.}

\fig{test/img_prueba_1_autonomia_2}{test1.1}{Drone en despegue.}

\begin{prueba}[{\bf Subsistema:} Vuelo Autónomo, {\bf Módulo:} Control de movimientos, {\bf Caso de uso:} CM-CU1 Realizar movimiento]{CM-CU1-P2}{Despegue del drone}
	\Pitem{Elevación de drone}
	\pregunta{¿Se armaron los motores del drone?}{X}{}{Ninguna.}
	\pregunta{¿El drone se elevó?}{X}{}{Ninguna.}
	\pregunta{¿El drone se mantuvo en su posición durante al menos 3 segundos?}{X}{}{Debido a errores de GPS, el drone tiende a moverse en dirección a la posición errónea.}
\end{prueba}

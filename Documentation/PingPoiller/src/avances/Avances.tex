\section{Desarrollo}
En este capitulo de muestran los pasos necesarios para el desarrollo del ping poller.

% =========================================================================================

\subsection{Configuración del entorno}

\subsubsection{Objetivo}
Instalar las herramientas necesarias para el desarrollo del ping poller.

\subsubsection{Descripción}
\begin{itemize}
	\item Instalar python 2.7.9\\
		\begin{lstlisting}[language=bash]
			$ apt-get update
			$ apt-get install gcc python=2.7.9-1 python-pip librrd-dev libpython-dev
		\end{lstlisting}
	\item Instalar rrdtool.\\
		\begin{lstlisting}[language=bash]
			$ pip install rrdtool
		\end{lstlisting}
\end{itemize}

\newpage
\begin{prueba}[{\bf Sistema:} Topología de red, {\bf Módulo:} Ping Poller]{P01-P1}{rrdtool instalado}
	\Pitem{Verificar paquetes instalados}
	\pregunta{¿Se instaló python 2.7.9?}{X}{}{Ninguna.}
	\pregunta{¿Se instaló pip?}{X}{}{Ninguna.}
	\pregunta{¿Se instaló rrdtool?}{X}{}{Me marcaba errores de haders y error en gcc. La solución fue reinstalar gcc.}
\end{prueba}

% =========================================================================================

\subsection{Realización del programa Ping Poller}

\subsubsection{Objetivo}
Realizar un programa en python que mande ping's a los dispositivos conectados en la red, y en caso de que la respuesta del dispositivo sobrepase el tiempo establecido, notificarlo via emal.

\subsubsection{Descripción}
\begin{itemize}
	\item {\bf ping.py} Programa encargado de realizar un ping a una ip específica y devuelve como resultado el tiempo de respuesta en milisegundos.\\
	\begin{lstlisting}[language=python]
		import subprocess, re
		
		def ping(hostname):
			ping_response = subprocess.Popen(
				['ping', '-c1', '-w1', hostname],
				stdout=subprocess.PIPE
			).stdout.read()
			
			# Verify packet recived.
			packets_recived = int(re.search(
				r'([\w|\s|.|\(|\)|\-|:|,\|=|\n]*)([0|1])( received)',
				ping_response).group(2))
			
			if (packets_recived == 1):
				time = int(re.search(
					r'([\w|\s|.|\(|\)|\-|:|,\|=|\n]*)(time=)([0-9]*)([.|0-9]*)( ms)',
					ping_response).group(3))
				return time
			else:
				return -1
	\end{lstlisting}
	
	\item {\bf rrdtool-db.py} Programa encargado de crear la base de datos round robin donde se almacenará el tiempo de retardo de los ping's realizados a una ip determinada.\\
		\begin{lstlisting}[language=python]
			#!/bin/python
			import sys, rrdtool, time
			
			tMinute = 60
			tHour = tMinute * 60
			tDay = tHour * 24
			#####################################################
			fname = 'ping.rrd'
			step = 1  # Time interval betwen every muestra. (1 sec)
			time_average = 1  # Time interval to average. (1 sec)
			hostname = sys.argv[1].replace('.', '_')
			#####################################################
			stime = int(time.time())
			duration = tHour
			
			# Un archivo round robin, que promedia las muestras de 5s durante una hora.
			ret = rrdtool.create(
				fname,
				'--start', str(stime),
				'--step', str(step),
				'DS:ping_%s:GAUGE:3:U:U' % hostname,
				'RRA:AVERAGE:0.5:%d:%d' % (time_average, duration / time_average),
			)
			
			if ret:
				print rrdtool.error()
		\end{lstlisting}
		Uso:\\
		\begin{lstlisting}[language=bash]
			$ ./rrdtool-db.py <IP>
		\end{lstlisting}
		
	\item {\bf rrdtool-ping.py} Programa encargado de actualizar la base de datos round robin con el tiempo de retardo de un ping realizado a una ip determinada.\\
		\begin{lstlisting}[language=python]
			#!/usr/bin/python
			import rrdtool, time, sys
			from ping import ping
			
			fname = 'ping.rrd'
			hostname = sys.argv[1]
			
			#######################################################
			
			ping_time = ping(hostname)

			# Se actualiza la bd con valores de [0, 1000] ms cada segundo.
			rrdtool.update(
				fname,
				'-t', 'ping_%s' % hostname.replace('.', '_'),
				'%d:%d' % (int(time.time()), ping_time)
			)
		\end{lstlisting}
		Uso:\\
		\begin{lstlisting}[language=bash]
			$ ./rrdtool-ping.py <IP>
		\end{lstlisting}

	\item {\bf rrdtool-draw.py} Programa encargado de crear una grafica del tiempo de retardo de los ping's realizados a una ip determinada durante los ultimos 10 minutos.\\
		\begin{lstlisting}[language=python]
			#!/usr/bin/python
			import rrdtool , time , random, sys
			
			tMinute = 60
			tHour = tMinute * 60
			tDay = tHour * 24
			##########################################################
			hostname = sys.argv[1].replace('.', '_')
			fname = 'ping.rrd'
			gfname = 'ping_' + hostname + '.png'
			duration = 10 * tMinute
			step = 1  # One second.
			time_average = 10  # Time interval to average. (10 sec)
			##########################################################
			dpoints = duration / step
			etime = int(time.time())
			stime = etime - dpoints
			
			rrdtool.graph(
				gfname,
				'--imgformat', 'PNG',
				'-w', '1100',
				'-h', '300',
				'--title', 'Pings',
				'--vertical-label', 'Tiempo (s)',
				'--lower-limit', '0',
				'--start', str(stime - 1),
				'--end', str(etime + 1),
				'DEF:rate=%s:ping_%s:AVERAGE' % (fname, hostname),
				'CDEF:ms=rate',
				'VDEF:msmax=ms,MAXIMUM',
				'VDEF:msavg=ms,AVERAGE',
				'VDEF:msmin=ms,MINIMUM',
				'LINE1:ms#FF0000:Retardo',
				r'GPRINT:msmax:Max\: %6.1lf ms',
				r'GPRINT:msavg:Prom\: %6.1lf ms',
				r'GPRINT:msmin:Min\: %6.1lf ms\l',
			)
		\end{lstlisting}
		Uso:\\
		\begin{lstlisting}[language=bash]
			$ ./rrdtool-draw.py <IP>
		\end{lstlisting}
\end{itemize}

\begin{prueba}[{\bf Sistema:} Topología de red, {\bf Módulo:} Ping Poller]{P01-P2}{Ping Poller}
	\Pitem{Verificar que se grafque correctamente.}
	\pregunta{¿Al ejecutar el programa rrdtool-db.py se creó el archivo ping.rrd?}{X}{}{Ninguna.}
	\pregunta{¿Se ejecuta correctamente el programa rrdtool-ping.py?}{X}{}{Ninguna.}
	\pregunta{¿Al ejecutar el programa rrdtool-draw.py se creó el archivo ping\_$<IP>$.png?}{X}{}{Ninguna.}
\end{prueba}

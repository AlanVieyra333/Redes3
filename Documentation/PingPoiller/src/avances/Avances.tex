\section{Desarrollo}
En este capitulo de muestran los pasos necesarios para el desarrollo del ping poller.

% =========================================================================================

\subsection{Configuración del entorno}

\subsubsection{Objetivo}
Instalar las herramientas necesarias para el desarrollo del ping poller.

\subsubsection{Descripción}
\begin{itemize}
	\item Instalar python 2.7.9\\
		\begin{lstlisting}[language=bash]
			#> apt-get update
			#> apt-get install gcc python=2.7.9-1 python-pip librrd-dev libpython-dev
		\end{lstlisting}
	\item Instalar rrdtool.\\
		\begin{lstlisting}[language=bash]
			#> pip install rrdtool
		\end{lstlisting}
\end{itemize}

\newpage
\begin{prueba}[{\bf Sistema:} Topología de red, {\bf Módulo:} Ping Poller]{P01-P1}{rrdtool instalado}
	\Pitem{Verificar paquetes instalados}
	\pregunta{¿Se instaló python 2.7.9?}{X}{}{Ninguna.}
	\pregunta{¿Se instaló pip?}{X}{}{Ninguna.}
	\pregunta{¿Se instaló rrdtool?}{X}{}{Me marcaba errores de haders y error en gcc. La solución fue reinstalar gcc.}
\end{prueba}

% =========================================================================================

\subsection{Realización del programa Ping Poller}

\subsubsection{Objetivo}
Realizar un programa en python que mande ping's a los dispositivos conectados en la red, y en caso de que la respuesta del dispositivo sobrepase el tiempo establecido, notificarlo via emal.

\subsubsection{Descripción}
\begin{itemize}
	\item {\bf ping.py} Programa encargado de realizar un ping a una ip específica y devuelve como resultado el tiempo de respuesta en milisegundos.\\
	\begin{lstlisting}[language=python]
		import subprocess, re
		
		def ping(hostname):
			ping_response = subprocess.Popen(
				['ping', '-c1', '-w1', hostname],
				stdout=subprocess.PIPE
			).stdout.read()
			
			# Verify packet recived.
			packets_recived = int(re.search(
				r'([\w|\s|.|\(|\)|\-|:|,\|=|\n]*)([0|1])( received)', ping_response)
			.group(2))
			
			if (packets_recived == 1):
				time = int(re.search(
					r'([\w|\s|.|\(|\)|\-|:|,\|=|\n]*)(time=)([0-9]*)( ms)', ping_response)
				.group(3))
				return time
			else:
				return -1
	\end{lstlisting}
	\item Instalar rrdtool.\\
	\begin{lstlisting}[language=bash]
	#> pip install rrdtool
	\end{lstlisting}
\end{itemize}

\newpage
\begin{prueba}[{\bf Sistema:} Topología de red, {\bf Módulo:} Ping Poller]{P01-P1}{rrdtool instalado}
	\Pitem{Verificar paquetes instalados}
	\pregunta{¿Se instaló python 2.7.9?}{X}{}{Ninguna.}
	\pregunta{¿Se instaló pip?}{X}{}{Ninguna.}
	\pregunta{¿Se instaló rrdtool?}{X}{}{Me marcaba errores de haders y error en gcc. La solución fue reinstalar gcc.}
\end{prueba}

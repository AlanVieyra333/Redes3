\section{Glosario de Términos}

% Glosario de términos.
%

\begin{itemize}
	\item \textbf{Algoritmo}. \\
 Es un conjunto prescrito de instrucciones o reglas bien definidas, ordenadas y 
finitas que permite llevar a cabo una actividad mediante pasos sucesivos que no 
generen dudas a quien deba hacer dicha actividad.

	\item \textbf{AJAX}. \\
Término que significa en inglés: Asyncronous JavaScript and XML que significa, 
JavaScript y XML asíncrono. Es una técnica de desarrollo web para crear 
aplicaciones interactivas o RIA (Rich Internet Applications). Estas aplicaciones 
se ejecutan del lado del cliente, es decir, en el navegador de los usuarios 
mientras se mantiene la comunicación asíncrona con el servidor en segundo plano.

	\item \textbf{Aplicación Web}.
 Aquellas herramientas que los usuarios pueden utilizar accediendo a un servidor 
web a través de internet o de una intranet mediante un navegador.

	\item \textbf{Applet}. \\
Componentes de programas realizados en lenguaje Java. Se pueden insertar en un 
contendor web para que sean ejecutadas.

	\item \textbf{Ardupilot}. \\
Sistema de código libre, un kit para vehículos aéreos no tripulados, capaz de 
controlar de forma autónoma:
\begin {itemize}
	\item Multimotores.
	\item Helicópteros.
	\item Submarinos.
	\item Rastreadores de antena.
\end{itemize}

	\item \textbf{Bytecodes}. \\
Es un código intermedio más abstracto que el código máquina. Habitualmente es 
tratado como un archivo binario que contiene un programa ejecutable similar a un 
módulo objeto, que es un archivo binario producido por el compilador cuyo 
contenido es el código objeto o código máquina.

	\item \textbf{CAN}. \\
Es un protocolo de comunicaciones, basado en una topología bus para la 
transmisión de mensajes en entornos distribuidos.

	\item \textbf{CSS}. \\
Es una hoja de estilos en cascada (\emph{Cascade-Style-Sheet}). Describe cómo 
los elementos HTML son mostrados en pantalla por el navegador.

	\item \textbf{C++}. \\
Es un lenguaje de programación diseñado a mediados de los años 1980 por Bjarne 
Stroustrup. En C++, la expresión $"C++"$ significa $"incremento de C"$ y se 
refiere a que C++ es una extensión de C. En ese sentido, desde el punto de vista 
de los lenguajes orientados a objetos, el C++ es un lenguaje de programación 
híbrido, ya que se pueden controlar mecanismos propios del lenguaje C así como 
tener el manejo de objetos.

	\item \textbf{Cliente/Servidor}. \\
Es un modelo de diseño de software en el que las tareas se reparten entre los p
roveedores de recursos o servicios, llamados servidores, y los demandantes, 
llamados clientes.

	\item \textbf{Drone}. \\
Es un vehículo aéreo sin tripulación humana (UAV).

	\item \textbf{DOM}. \\
Abreviación de Document-Object-Model, o bien, modelo de objeto de documento. 
Es esencialmente una interfaz de plataforma que proporciona un conjunto estándar 
de objetos para representar documentos HTML, XHTML y XML. A través del DOM, los 
programas pueden acceder y modificar el contenido, estructura y estilo de los 
documentos HTML y XML, que es para lo que se diseñó principalmente.

	\item \textbf{IIC}. \\
Abreviación de Inter-Integrated Circuit. Es un bus serie de datos desarrollado 
en 1982 por Philips Semiconductor.  Se utiliza principalmente internamente para 
la comunicación entre diferentes partes de un circuito, por ejemplo, entre un 
controlador y circuitos periféricos integrados.

	\item \textbf{HTML}. \\
Es un estándar que sirve de referencia del software que conecta con la 
elaboración de páginas web en sus diferentes versiones, define una estructura 
básica y un código (denominado código HTML) para la definición de contenido de 
una página web, como texto, imágenes, videos, juegos, entre otros. Éste código 
está a cargo de la W3C o bien World Wide Web Consortium.

	\item \textbf{Modulación de ancho de pulso (PWM)}. \\
Técnica que modifica el ciclo de trabajo de una señal periódica, usada para 
modificar la energía que se envía a una fuente.

	\item \textbf{UART}. \\
Por sus siglas en inglés Universal Asyncronous Receiver/Transmitter. Es un bus
serial usado para comunicación síncrona, compuesto de tres canales: RX, TX, 
Tierra (GND).

	\item \textbf{USB}. \\
Es un bus serial universal, hoy en día usado para la comunicación entre 
cualquier dispositivo: computadoras, periféricos, entre otros dispositivos. Éste
define los cables universales para proveer esa comunicación.

	\item \textbf{SDK}. \\
Es un conjunto de herramientas de software que permite al programador o 
desarrollador crear aplicaciones informáticas para un sistema en concreto.

\end{itemize}

.\\ \\
